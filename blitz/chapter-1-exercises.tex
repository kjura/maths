\documentclass{article}
\usepackage{amsmath}
\usepackage{amssymb}
\usepackage{amsthm}


\newtheorem*{conjecture}{Conjecture}
\newtheorem*{theorem}{Theorem}
\newtheorem{lemma}{Lemma}
\newtheorem*{proposition}{Proposition}


\makeatletter
\newcommand*{\rom}[1]{\expandafter\@slowromancap\romannumeral #1@}
\makeatother


\begin{document}

\section*{Exercise 1}

There are 11 letters in the word "MISSISSIPPI" in total. We have one M, four I's and S's and two P's. We have
eleven positions available for every letter. Moreover, we can sequentially
choose next rearrangments eliminating available positions until we exhaust all places for the letters. We have then:
\begin{equation*}
    \binom{11}{4} \cdot \binom{7}{4} \cdot \binom{3}{2} \cdot \binom{11}{4} = \frac{11!}{4! 4! 2! 1!} = 34650
\end{equation*}

To double check our reasoning, let's look at the problem from an equivalent perspective. There are \(11!\) possibilites
of rearranging 11 objects in a row where order matters. We can divide this number by respectively \(4!\), \(4!\), \(2!\) and \(1!\)
treating these factorials as a number of ways every letter could rearrange itself. By doing that we adjust for overcounting
and we end up with the right side of the above equation:
\begin{equation*}
    \frac{11!}{4! 4! 2! 1!} = 34650
\end{equation*}

\section*{Exercise 4}
\subsection*{(a)}
The total number of matches played:

\begin{equation*}
    \binom{n}{2} = \frac{n!}{2!(n - 2)!} = \frac{n(n-1)(n-2)!}{2!(n-2)!} = \frac{n(n-1)}{2}
\end{equation*}
we can do a quick sanity check to see that we are right. We know that there are \(n\) players so we want to choose
sets consisting of 2 elements from the set containing \(n\)-elements. On the other hand, we know that there are
\(n\) players who will play \(n-1\) matches and we divide this product by \(2\) to adjust for two possible
states of a given match: player A wins and player B loses or the opposite takes place.
To get the number of possible outcomes we put every possible match on the board and then count every state of 
a given match. Because there are two possibilites for a match we use Multiplication Rule to arrive to:
\begin{equation*}
    \underbrace{2 \cdot 2 \cdot 2 \cdot \dots }_\text{\(\binom{n}{2}\) times}
\end{equation*} 

So the total number of possible outcomes is \(2^{\binom{n}{2}}\)
\subsection*{(b)}
The total number of games played is \(\binom{n}{2} = \frac{n!}{2!(n - 2)!} = \frac{n(n-1)(n-2)!}{2!(n-2)!} = \frac{n(n-1)}{2}\).

We don't care about the order on the leaderboard, but order \emph{inside} the match
((W1, L2) is different than (L1, W2)).

\section*{Exercise 5}
\subsection*{(a)}
There are \(n\) rounds for \(2^n\) players taking place in the competition.

\subsection*{(b)}
We have \(2^n\) players taking place in the competition. Adjusting for the outcome of a given match (only one player can win)
we can divide the total number of players in a round by \(2\) to get the number of games played in a given round. Every next
round we need to start with a new number of remaining players arriving at:
\begin{equation*}
    \frac{2^n}{2} + \frac{2^{n-1}}{2} + \frac{2^{n-2}}{2} + \dots + \frac{2^1}{2} = 1 + \frac{1}{2} \sum_{i=1}^{n} 2^i
\end{equation*}

\subsection*{(c)}
There are \(2^n\) players starting the game. Looking at it from another angle, \(2^{n} - 1\) players must be eliminated
from the tournament (we subtract one because one player is the winner). This means that \(2^{n} - 1\) games
must be played to reveal the winner.
\newpage

\section*{Exercise 15}
\begin{center}
    STILL A DRAFT
\end{center}
Consider an experiment where we throw a coin. It can land either on heads or tails. We would like to throw it
\(n \geq 1\) times. We can consider \(n\) to be a number of experiments needed to estabilish a final sequence
of outcomes e.g \(HHTTHHTTH\) or \(TTTTTTTTTHHTT\) etc. Every throw there are two possibilites to happen 
so if we throw \(n\) times we get to have \(\underbrace{2 \cdot 2 \cdot 2 \cdot \dots}_\text{\(n\) times}\) possible
coin sequences. This is just applying Multiplication Rule so the final number of outcomes can be written as:
\begin{equation*}
    2^n
\end{equation*}
Now, consider tackling this problem from a different angle. Without loss of 
generality, instead of choosing over head or tails in a throw
let's count occurences of heads only. Then in the first throw we want to choose 0 heads from n throws, in the second
choose 1 heads from n throws, in the third choose 2 heads from n throws etc. Doing so we estabilish a pattern of using
binomial coefficient every next count summing previous possibilites of heads for n throws:
\begin{equation*}
    \binom{n}{0} + \binom{n}{1} + \binom{n}{2} + \binom{n}{3} + \dots 
\end{equation*}
We finally end up with:
\begin{equation*}
    \sum_{k=0}^{n} \binom{n}{k}
\end{equation*}
It is worth adding, that we do not lose information about tails, because we know that the final sequence have
\(n\) outcomes, knowing how many heads there was we simultaneously know the number of tails (and vice versa). We interpret
n choose 0 as "what is the number of outcomes without any heads landed".


\end{document}
