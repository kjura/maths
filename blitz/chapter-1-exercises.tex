\documentclass{article}
\usepackage{amsmath}
\usepackage{amssymb}
\usepackage{amsthm}


\newtheorem*{conjecture}{Conjecture}
\newtheorem*{theorem}{Theorem}
\newtheorem{lemma}{Lemma}
\newtheorem*{proposition}{Proposition}


\makeatletter
\newcommand*{\rom}[1]{\expandafter\@slowromancap\romannumeral #1@}
\makeatother


\begin{document}

\section{Chapter 1}
\subsection*{Counting}
\subsubsection*{Exercise 1}

There are 11 letters in the word "MISSISSIPPI" in total. We have one M, four I's and S's and two P's. We have
eleven positions available for every letter. Moreover, we can sequentially
choose next rearrangments eliminating available positions until we exhaust all places for the letters. We have then:
\begin{equation*}
    \binom{11}{4} \cdot \binom{7}{4} \cdot \binom{3}{2} \cdot \binom{11}{4} = \frac{11!}{4! 4! 2! 1!} = 34650
\end{equation*}

To double check our reasoning, let's look at the problem from an equivalent perspective. There are \(11!\) possibilites
of rearranging 11 objects in a row where order matters. We can divide this number by respectively \(4!\), \(4!\), \(2!\) and \(1!\)
treating these factorials as a number of ways every letter could rearrange itself. By doing that we adjust for overcounting
and we end up with the right side of the above equation:
\begin{equation*}
    \frac{11!}{4! 4! 2! 1!} = 34650
\end{equation*}

\end{document}
