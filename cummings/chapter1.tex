\documentclass{article}
\usepackage{amsmath}
\usepackage{amssymb}
\usepackage{amsthm}


\newtheorem*{conjecture}{Conjecture}

\begin{document}



\section*{Exercise 2.3}
\subsection*{a)}
Three examples for the property

\begin{itemize}
	\item 2 + 3 = 5 = 2(2) + 1
	\item 8 + 7 = 15 = 2(7) + 1
	\item 6 + 13 = 19 = 2(9) + 1
\end{itemize}

Let $m$ be an even integer and $n$ be an odd integer. By definition it means
that:
\begin{align*}
	m &= 2k, \quad \mbox{for some}\ k \in \mathbb{Z} \\	
	n &= 2l + 1, \ \mbox{for some}\ l \in \mathbb{Z}
\end{align*}

Then:
\begin{equation*}
	m + n = 2k + (2l + 1) = (2k + 2l) + 1 = 2(k+l) + 1
\end{equation*}
\noindent
Since $(k+l)$ is some integer, it follows from the definition of odd number
that $2(k+l) + 1$ is odd.

\subsection*{b)}


Three examples for the property

\begin{itemize}
	\item 2 * 2 = 4 = 2(2)
	\item 4 * 6 = 24 = 2(12)
	\item 8 * 10 = 80 = 2(40)
\end{itemize}

Let $m,n$ be even integers. By definition it means that:

\begin{align*}
	m &= 2k, \quad \mbox{for some}\ k \in \mathbb{Z} \\	
	n &= 2l, \quad \mbox{for some}\ l \in \mathbb{Z}
\end{align*}

Then:
\begin{equation*}
	m * n = 2k * 2l = 4kl = 2(2kl)
\end{equation*}

Since $2kl$ is some integer, it follows from the definition that $2(2kl)$ is
even.

\subsection*{c)}


Three examples for the property:

\begin{itemize}
	\item 3 * 3 = 9 = 2(4) + 1
	\item 7 * 11 = 77 = 2(38) + 1 
	\item 13 * 3 = 39 = 2(19) + 1
\end{itemize}

Let $m,n$ be odd integers. By definition it means that:


\begin{align*}
	m &= 2k + 1, \quad \mbox{for some}\ k \in \mathbb{Z} \\	
	n &= 2l + 1, \quad \mbox{for some}\ l \in \mathbb{Z}
\end{align*}

Then:
\begin{equation*}
	m * n = (2k+1) * (2l+1) = 4kl + 2k + 2l + 1 = 2(2kl + k) + (2l + 1) =
	2(k(2l+1)) + (2l+1)
\end{equation*}

%NOTE Be more explicit, about using the fact from the previous exercise

The left term is an even integer and the right term is an odd integer, it
follows then from the previous point (Exercise 2.1 (a)) that $2(k(2l+1)) +
(2l+1)$ is odd.


\subsection*{d)}


Three examples for the property:

\begin{itemize}
	\item 4 * 7 = 28 = 2(14)
	\item 3 * 2 = 6 = 2(3)
	\item 5 * 4 = 20 = 2(10)
\end{itemize}

Let $m$ be an even integer and $n$ be an odd integer. By definition we have:


\begin{align*}
	m &= 2k, \quad \mbox{for some}\ k \in \mathbb{Z} \\	
	n &= 2l + 1, \quad \mbox{for some}\ l \in \mathbb{Z}
\end{align*}

Then:
\begin{equation*}
	m * n = 2k * (2l + 1) = 4kl + 2k = 2(k(2l + 1))
\end{equation*}
\noindent
Since $k(2l + 1)$ is some integer, it follows from the definition that $2(k(2l + 1))$ is
even.


\subsection*{e)}


Three examples for the property:

\begin{itemize}
	\item $2^2=4=2(2)$
	\item $4^2 = 16 = 2(8)$
	\item $6^2 = 36 = 2(18)$
\end{itemize}


Let $n$ be an even integer. By definition we have:

\begin{align*}
	n &= 2k, \quad \mbox{for some}\ k \in \mathbb{Z}
\end{align*}

Then:

\begin{equation*}
	n^2 = 4k^2 = 4 * k * k = 2(2*k*k)
\end{equation*}

Since $2*k*k$ is some integer, by definition $2(2*k*k)$ is even.


\section*{Exercise 2.4}

\subsection*{a)}


Three examples for the property:

\begin{itemize}
	\item $-4 = (-1)4 = (-1) \cdot 2 \cdot 2 = 2 \cdot ((-1)\cdot 2)$
	\item $-8 = (-1) \cdot 8 = (-1) \cdot 2 \cdot 4 = 2 \cdot ((-1)\cdot 4)$
	\item $-12 = (-1) \cdot 12 = (-1) \cdot 2 \cdot 6 = 2 \cdot ((-1)\cdot
		6)$
\end{itemize}



Let $n$ be an even integer. By definition we have:

\begin{align*}
	n &= 2k, \quad \mbox{for some}\ k \in \mathbb{Z}
\end{align*}

Then:

\begin{equation*}
	-n = (-1)\cdot n = (-1) \cdot 2 \cdot k = 2 \cdot ((-1)\cdot k)
\end{equation*}

Since $(-1) \cdot k$ is some integer then by definition $2 \cdot ((-1) \cdot k)$ is even.


\subsection*{b)}


Three examples for the property:

\begin{itemize}
	\item $-5 = - 1 \cdot 5 = 2(-1 \cdot 2 -1) + 1$
	\item $-7 = - 1 \cdot 7 = 2(-1 \cdot 3 -1) + 1$ 
	\item $-9 = - 1 \cdot 9 = 2(-1 \cdot 4 - 1) + 1$  
\end{itemize}



Let $n$ be an odd integer. By definition we have:

\begin{align*}
	n &= 2k + 1, \quad \mbox{for some}\ k \in \mathbb{Z}
\end{align*}

Then:

\begin{equation*}
	-n = -2k - 1 = -2k + (1 - 2) = (-2k - 2) + 1 = 2(-k - 1) + 1
\end{equation*}

Since $(-k - 1)$ is some integer it follows from definition that   $2(-k - 1) + 1$
is odd.
\subsection*{c)}


Three examples for the property:

\begin{itemize}
	\item $(-1)^2 = (-1) \cdot (-1) = 1$
	
	\item $(-1)^4 = (-1) \cdot (-1) \cdot (-1) \cdot (-1) = 1 \cdot 1 =  1$

	\item $(-1)^6 = (-1) \cdot (-1) \cdot (-1) \cdot (-1) \cdot (-1) \cdot
		(-1) = 1 \cdot 1 \cdot 1  = 1$
\end{itemize}


Let $n$ be an even integer. By definition we have:

\begin{align*}
	n &= 2k, \quad \mbox{for some}\ k \in \mathbb{Z}
\end{align*}

Then:

\begin{equation*}
	(-1)^n = (-1)^{2k}
\end{equation*}

We use the fact that $\forall m,n \in \mathbb{Z}$ we have $(b^m)^n = b^{mn}$
(provided that the base is non-zero):

\begin{equation*}
	(-1)^n = (-1)^{2k} = (-1^2)^k = ((-1) \cdot (-1))^{k} = 1^k = 1
\end{equation*}
\noindent
In the last step we use the fact that raising $1$
to any power is still equal to $1$. 


\section*{Exercise 2.8}

\subsection*{(a)}

\begin{description}
    \item[Case 1: $n$ is even:]
	    Let $n$ be an even integer. This means that $n = 2k$ for some
	    integer $k$. We have:
	    \begin{equation*}
		n^2 + n = n(n+1) = (2k)(2k + 1) = 4k^{2} + 2k = 2(2k^2 + k)
	    \end{equation*}
    \item[Case 2: $n$ is odd:] Now, let n be an odd integer. This means that $n
	    = 2k + 1$ for some integer $k$. We have:
	    \begin{equation*}
		n^2 + n = n(n+1) = (2k +1)\big((2k + 1) + 1 \big) = 4k^2 + 6k + 2 = 2(2k^2 + 3k + 1)
	    \end{equation*}
\end{description}

It follows from the definition in both cases that $n^2 +n$ is even.


\subsection*{(b)}


\begin{description}
    \item[Case 1: $n$ is even:]
	    Let $n$ be an even integer. This means that $n = 2k$ for some
	    integer $k$. We have:
	    \begin{equation*}
		    3n^2 + 5n + 1 = 3\Big((2k)^2\Big) + 5(2k) + 1 = 12k^2 + 10k
		    + 1 = 2(6k^2 + 5k) + 1
	    \end{equation*}
    \item[Case 2: $n$ is odd:] Now, let n be an odd integer. This means that $n
	    = 2k + 1$ for some integer $k$. We have:
	    \begin{equation*}
		    3n^2 +5n + 1 = 3\Big((2k + 1)^2\Big) + 5(2k+1) + 1 
	    \end{equation*}
	    After some quick algebra we end up with:
	    \begin{equation*}
		   3n^2 + 5n + 1 = 12k^2 + 22k + 8 + 1 = 2(6k^2 + 11k + 4) + 1
	    \end{equation*}


\end{description}

It follows from the definition in both cases that $3n^2 +5n + 1$ is odd.


\subsection*{(c)}


\begin{description}
    \item[Case 1: $n$ is even:]
	    Let $n$ be an even integer. This means that $n = 2k$ for some
	    integer $k$. We have:
	    \begin{equation*}
		n^2 +3n - 6 = (2k)^2 + 3(2k) - 6 = 4k^2 + 6k - 6 = 2(2k^2 + 3k -
		3)
	    \end{equation*}
    \item[Case 2: $n$ is odd:] Now, let n be an odd integer. This means that $n
	    = 2k + 1$ for some integer $k$. We have:
	    \begin{equation*}
		    n^2 + 3n -6 = \Big((2k+1)^2\Big) + 3(2k+1) - 6 
	    \end{equation*}
	    After some quick algebra we end up with:
	    \begin{equation*}
		    n^2 +3n - 6 = 4k^2 + 10k - 2 = 2(2k^2 + 5k -1)
	    \end{equation*}
\end{description}

It follows from the definition in both cases that $n^2 +3n - 6$ is even.

\subsection*{(d)}

The following proof will consist of four cases where in the second and in the third
case we will show that the proposition does not hold (book error?):

\begin{description}
    \item[Case 1: $m$ is even and $n$ is even:]
	    Let $m, n$ be even integers. This means that $n = 2k$ for some
	    integer $k$ and $m = 2l$ for some integer $l$. We have:
	    \begin{equation*}
		7m -3n = 7(2l) - 3(2k) = 14l - 6k = 2(7l -3k)
	    \end{equation*}
    \item[Case 2: $m$ is odd and $n$ is even:]
	    Let $m$ be an odd integer and $n$ an even integer. This means that $n = 2k$ for some
	    integer $k$ and $m = 2l + 1$ for some integer $l$.
	    \begin{equation*}
		7m -3n = 7(2l + 1) - 3(2k) = 14l + 7 - 6k = 2(7l - 3k + 3) + 1
	    \end{equation*}
    \item[Case 3: $m$ is even and $n$ is odd:]	
	    Let $m$ be an even integer and $n$ be an odd integer. This means
	    that $n = 2k + 1$ for some
	    integer $k$ and $m = 2l$ for some integer $l$. We have:
	    \begin{equation*}
		7m -3n = 7(2l) - 3(2k+1) = 14l - 6k - 3 = 2(7l - 3k - 2) + 1
	    \end{equation*}
    \item[Case 4: $m$ is odd and $n$ is odd:]
	    Let $m, n$ be odd integers. This means that $n = 2k + 1$ for some
	    integer $k$ and $m = 2l + 1$ for some integer $l$. We have:
	    \begin{equation*}
		7m - 3n = 7(2l + 1) - 3(2k + 1) = 14l + 7 - 6k - 3 = 2(7l - 3k
		+ 2)
	    \end{equation*}
\end{description}

Only if $m,n$ are both even or odd the proposition holds. Otherwise, $7m - 3n$
is odd.



\section*{Exercise 2.9}

We write down some examples in the hope of seeing if there are any emerging
patterns:

\begin{itemize}
	\item $2 \cdot 2 = 4$
	\item $2 \cdot 3 = 6$
	\item $2 \cdot 4 = 8$
	\item $2 \cdot 5 = 10$
	\item $2 \cdot 6 = 12$
	\item $2 \cdot 7 = 14$
	\item $2 \cdot 8 = 16$
\end{itemize}


\begin{itemize}
	\item $3 \cdot 3 = 9$
	\item $3 \cdot 5 = 15$
	\item $3 \cdot 7 = 21$
	\item $3 \cdot 9 = 27$
	\item $3 \cdot 11 = 33$
\end{itemize}


From the first list it seems that whenever we multiply an even number by an odd
number or we multiply two even numbers the result is also even. If we multiply
two odd numbers then the outcome is odd. We make a following conjecture:

\begin{conjecture}
	If $m$ and $n$ are both even then $mn$ is even. If $m$ is even and $n$
	is odd then $mn$ is also even.
\end{conjecture}


\begin{description}
    \item[$m$ is even and $n$ is even:]
	    Let $m, n$ be even integers. This means that $n = 2k$ for some
	    integer $k$ and $m = 2l$ for some integer $l$. We have:
	    \begin{equation*}
		m \cdot n = 2l \cdot 2k = 2(l+k)
	\end{equation*}
    \item[$n$ is odd and $m$ is even:]
	    Let $m$ be an even integer and $n$ an odd integer. This means that
	    $n = 2k + 1$ for some
	    integer $k$ and $m = 2l$ for some integer $l$. We have:
	    \begin{equation*}
		    m \cdot n = 2l \cdot (2k + 1) = 4lk +2l = 2(2lk + l)
	    \end{equation*}
    \end{description}
\noindent
It follows from the definition that in both cases $m \cdot n$ is odd.


\section*{Exercise 2.11}

\subsection*{(a)}

Let $n$ be an integer. Observe that $n = 1 \cdot n$. We want to find some
integer $k$ such that $n = 1 \cdot k$. Let $k=n$. It follows that $1 \mid n$.

\subsection*{(b)}

Let $n$ be an integer. We want to find some integer $k$ such that $n = n \cdot
k$. Let $k=1$. It follows that $n \mid n$.

\subsection*{(c)}

Let $m, n, t$ be integers. We have $t = k(mn)$ for some integer $k$. We seek
some $l$ such that $t = l \cdot m$. Let $l=k \cdot n$. Then:
\begin{equation*}
	t = k(mn) = (kn)m = lm
\end{equation*}

It follows from the definition that $m \mid t$.


\subsection*{(d)}


Let $m, n, t$ be integers. Suppose also that $n \neq 0$. We have $tn = k(mn)$ for some integer $k$. We have:

\begin{equation*}
	tn = k(mn) = n(km) 
\end{equation*}

Dividing both sides by $n$ we get:


\begin{equation*}
	t = k \cdot m
\end{equation*}

It follows that $m \mid t$.



\section*{Exercise 2.12}

Let $m,n$ be positive real numbers and $m < n$. From our assumptions:

\begin{equation*}
	m < n \implies 0 < n - m
\end{equation*}

Since $n,m$ are both positive, it must be that $n + m > 0$. We use the fact
that if $a<b$ and $c$ is positive, then $ac < bc$. It follows then
that:

\begin{equation*}
	0 \cdot (n+m) < (n-m) \cdot (n+m) = n^2 - m^2
\end{equation*}

Which is just:
\begin{equation*}
	m^2 < n^2
\end{equation*}






\end{document}
