\documentclass{article}
\usepackage{amsmath}
\usepackage{amssymb}

\begin{document}


\section*{Exercise 2.1}
\subsection*{a)}
Three examples for the property

\begin{itemize}
	\item 2 + 3 = 5 = 2(2) + 1
	\item 8 + 7 = 15 = 2(7) + 1
	\item 6 + 13 = 19 = 2(9) + 1
\end{itemize}

Let $m$ be an even integer and $n$ be an odd integer. By definition it means
that:
\begin{align*}
	m &= 2k, \quad \mbox{for some}\ k \in \mathbb{Z} \\	
	n &= 2l + 1, \ \mbox{for some}\ l \in \mathbb{Z}
\end{align*}

Then:
\begin{equation*}
	m + n = 2k + (2l + 1) = (2k + 2l) + 1 = 2(k+l) + 1
\end{equation*}
\noindent
Since $(k+l)$ is some integer, it follows from the definition of odd number
that $2(k+l) + 1$ is odd.

\subsection*{b)}


Three examples for the property

\begin{itemize}
	\item 2 * 2 = 4 = 2(2)
	\item 4 * 6 = 24 = 2(12)
	\item 8 * 10 = 80 = 2(40)
\end{itemize}

Let $m,n$ be even integers. By definition it means that:

\begin{align*}
	m &= 2k, \quad \mbox{for some}\ k \in \mathbb{Z} \\	
	n &= 2l, \quad \mbox{for some}\ l \in \mathbb{Z}
\end{align*}

Then:
\begin{equation*}
	m * n = 2k * 2l = 4kl = 2(2kl)
\end{equation*}

Since $2kl$ is some integer, it follows from the definition that $2(2kl)$ is
even.

\subsection*{c)}


Three examples for the property:

\begin{itemize}
	\item 3 * 3 = 9 = 2(4) + 1
	\item 7 * 11 = 77 = 2(38) + 1 
	\item 13 * 3 = 39 = 2(19) + 1
\end{itemize}

Let $m,n$ be odd integers. By definition it means that:


\begin{align*}
	m &= 2k + 1, \quad \mbox{for some}\ k \in \mathbb{Z} \\	
	n &= 2l + 1, \quad \mbox{for some}\ l \in \mathbb{Z}
\end{align*}

Then:
\begin{equation*}
	m * n = (2k+1) * (2l+1) = 4kl + 2k + 2l + 1 = 2(2kl + k) + (2l + 1) =
	2(k(2l+1)) + (2l+1)
\end{equation*}

The left term is an even integer and the right term is an odd integer, it
follows then from the previous point (Exercise 2.1 (a)) that $2(k(2l+1)) +
(2l+1)$ is odd.


\subsection*{d)}


Three examples for the property:

\begin{itemize}
	\item 4 * 7 = 28 = 2(14)
	\item 3 * 2 = 6 = 2(3)
	\item 5 * 4 = 20 = 2(10)
\end{itemize}

Let $m$ be an even integer and $n$ be an odd integer. By definition we have:


\begin{align*}
	m &= 2k, \quad \mbox{for some}\ k \in \mathbb{Z} \\	
	n &= 2l + 1, \quad \mbox{for some}\ l \in \mathbb{Z}
\end{align*}

Then:
\begin{equation*}
	m * n = 2k * (2l + 1) = 4kl + 2k = 2(k(2l + 1))
\end{equation*}
\noindent
Since $k(2l + 1)$ is some integer, it follows from the definition that $2(k(2l + 1))$ is
even.


\subsection*{e)}


Three examples for the property:

\begin{itemize}
	\item $2^2=4=2(2)$
	\item $4^2 = 16 = 2(8)$
	\item $6^2 = 36 = 2(18)$
\end{itemize}


Let $n$ be an even integer. By definition we have:

\begin{align*}
	n &= 2k, \quad \mbox{for some}\ k \in \mathbb{Z}
\end{align*}

Then:

\begin{equation*}
	n^2 = 4k^2 = 4 * k * k = 2(2*k*k)
\end{equation*}

Since $2*k*k$ is some integer, by definition $2(2*k*k)$ is even.


\section*{Exercise 2.2}

\end{document}
