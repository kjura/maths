\documentclass{article}
\usepackage{amsmath}
\usepackage{amssymb}
\usepackage{amsthm}
% \usepackage{ dsfont }


\newtheorem*{conjecture}{Conjecture}
\newtheorem*{theorem}{Theorem}
\newtheorem{lemma}{Lemma}
\newtheorem*{proposition}{Proposition}


\makeatletter
\newcommand*{\rom}[1]{\expandafter\@slowromancap\romannumeral #1@}
\makeatother


\begin{document}


\section*{Exercise 18}

\begin{equation*}
    X \sim \mathcal{N}(\mu = 3, \sigma^2 = 16)
\end{equation*}

\begin{equation*}
    Z = \frac{X - 3}{4} \sim \mathcal{N}(0, 1)
\end{equation*}

\subsection*{(a)}

\begin{equation*}
    P(X < 7) = P(Z < \frac{7 - 3}{4}) =  P(Z < 1) = 
\end{equation*}
\begin{equation*}
    = \Phi(1) - \lim_{x \to -\infty}\Phi(\frac{x - 3}{4}) = \Phi(1) - 0 = 0.8413
\end{equation*}


\subsection*{(b)}

\begin{equation*}
    P(X > -2) = 1 - P(X \leq -2) = 1 - P(X < -2) = 
\end{equation*}
\begin{equation*}
    1 - P(Z < \frac{-2 - 3}{4}) = 1 - P(Z < \frac{-5}{4}) = 
\end{equation*}
\begin{equation*}
   = 1 - \Bigl(1 - \Phi(\frac{5}{4})\Bigr) = 1 - 1 + \Phi(\frac{5}{4}) = \Phi(\frac{5}{4}) = \Phi(1.25) = 0.8944
\end{equation*}

\subsection*{(c)}


\begin{equation*}
    TOASK!!TOASK!!TOASK!!TOASK!!TOASK!!TOASK!!
\end{equation*}
\begin{center}
    Ask about strictness of the inequality in the quantile function 
\end{center}
\begin{equation*}
    TOASK!!TOASK!!TOASK!!TOASK!!TOASK!!TOASK!!
\end{equation*}

\begin{equation*}
    \mbox{Find}  \ x  \ \mbox{such that } P(X > x) = 0.05
\end{equation*}

\begin{equation*}
    P(X > x) = 0.05
\end{equation*}
\begin{equation*}
    P(X > x) = 1 - P(X \leq x) = 1 - P(X < x) = 0.05
\end{equation*}
\begin{equation*}
    P(X < x) = 1 - 0.05 = 0.95
\end{equation*}

We will use the quantile function to find our \(x\). That is, we seek 
(not really sure about this part, is it \(>\) or \(<\) ? 0.95 or 0.05?):

\begin{equation*}
    F^{-1}(0.95) = \inf \{x : F(x) = P(X < x) > 0.95\}
\end{equation*}

Let us find the argument for \(F_{X}(x)\) such that \(F(x) = 0.95\). We will first standarize \(X\):

\begin{equation*}
    F(x) = P(X \leq x) =  P(X < x) = P(Z < \frac{x - 3}{4}) =
\end{equation*}
\begin{equation*}
    = P(Z < \frac{x - 3}{4}) =  \Phi(\frac{x - 3}{4})
\end{equation*}

Now we will set up an equation for out threshold value. We want to find an argument for \(\Phi\) such that it gives us
\(\Phi(x) = 0.95\). We use the Z-score table and find that \(\Phi(1.6) = 0.9452 \sim 0.95\) (note that we could also find other values
like 1.61 or 1.62 etc. but we are interested in the \(\inf\) due to the quantile function definition). Having this information
we can set up the neccesary equation for the argument we seek:

\begin{equation*}
    \frac{x - 3}{4} = 1.6 \implies x = 9.4
\end{equation*}

so finally:

\begin{equation*}
    \Phi(\frac{9.4 - 3}{4}) = 0.95 \implies P(X > x) = 0.5
\end{equation*}

for \(x = 9.4\)

\begin{equation*}
    !!!!!!!!!!!!!!!!!!!!!!!!!!!!!!!!!!!!!!!!!!
\end{equation*}
\begin{center}
    Check later again, precision seems to be off with respect to numpy/scipy
\end{center}
\begin{equation*}
    !!!!!!!!!!!!!!!!!!!!!!!!!!!!!!!!!!!!!!!!!!
\end{equation*}

\end{document}