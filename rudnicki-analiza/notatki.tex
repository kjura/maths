\documentclass{article}
\usepackage{amsmath}
\usepackage{amssymb}
\usepackage{amsthm}
\usepackage[T1]{fontenc}


\newtheorem*{conjecture}{Przypuszczenie}
\newtheorem*{theorem}{Twierdzenie}
\newtheorem{lemma}{Lemat}


\makeatletter
\newcommand*{\rom}[1]{\expandafter\@slowromancap\romannumeral #1@}
\makeatother


\begin{document}

\section{Wstęp do matematyki}
\subsection{Elementy logiki i teorii zbiorów}

\subsection{Funkcje i relacje}

\subsection{Zbiory liczbowe}
\subsubsection{Liczby wymierne i rzeczywiste}
\subsubsection{Sformułowanie aksjomatu ciągłości przy użyciu przekroju Dedekinda}
Sformułowanie aksjomatu ciągłości w inny sposób niż ten podany w książce. Można użyć do tego
przekroju Dedekinda (Rudnicki nie podaje tego explicite).

\begin{theorem}
    Jeżeli \(A\) i \(B\) są niepustymi podzbiorami \(\mathbb{R}\) takimi, że:
    \begin{equation}
        \mathbb{R} = A \cup B, \ \ A \cap B = \varnothing,
    \end{equation}
    \begin{equation}
        (x \in A \ \ \land \ \ y \in B) \implies x < y,
    \end{equation}
    to albo zbiór \(A\) ma element największy, albo zbiór \(B\) ma element najmniejszy.
\end{theorem}
\begin{proof}[Dowód]
    Najpierw warto napisać dla jakich sytuacji mamy podział dwóch zbiorów na cały przekrój liczb rzeczywistych.
    Suma tych dwóch zbiorów zadziała dla tychże dwóch sytuacji (domkniętość przedziałów):
    \begin{equation*}
        A=(-\infty, t), \ B=[t,\infty) \ \ \mbox{lub} \ \ A=(-\infty, t], \ B=(t,\infty)
    \end{equation*}
    Zbiór \(B\) jest ograniczony z dołu (elementy zbioru \(A\) są mniejsze, więc znajdziemy ograniczenie dla \(B\) z dołu), 
    więc ma kres dolny (aksjomat ciągłości). Niech \(b = \inf(B)\).
    Ponieważ dowolny element \(a \in A\) ogranicza zbiór \(B\) z dołu, więc \(a \leq b\). Jeżeli \(b \in A\), to z ostatniej
    nierówności wynika, że \(b\) jest elementem największym zbioru \(A\). Jeżeli \(b \in B\), to z definicji kresu dolnego
    \(b\) jest elementem najmniejszym zbioru \(B\). 
\end{proof}

\section{Przestrzenie metryczne II}
\subsection{Zbiory otwarte i domknięte}
\newpage
\subsubsection{Przykład 3: Kula jest zbiorem domkniętym}

Kula \(\overline{B}(x_0 , r)\) jest zbiorem domkniętym w przestrzeni metrycznej \((X, d)\).
Wystarczy pokazać, że zbiór \(G = X \setminus \overline{B}(x_0 , r)\)
jest otwarty. Niech \(x \in G\). Wtedy \(d(x, x_0) > r\). Przyjmijmy, że \(\delta = d(x, x_0) - r\). 
Założymy nie wprost, że istnieje przynajmniej jeden element \(e\)
w zbiorze \(B(x, \delta)\) który nie występuje w \(G\). Wtedy dla pewnego elementu \(e\) mamy:
\begin{equation*}
    e \in B(x , \delta)
\end{equation*}
\begin{equation*}
    e \in \overline{B}(x_0 , r)
\end{equation*}
Oznacza to, że pewien punkt \(e\) spełnia dwie nierówności:
\begin{equation*}
    d(e, x) < \delta
\end{equation*}
\begin{equation*}
    d(e, x_0) \leq r 
\end{equation*}
Dodając je stronami otrzymujemy:
\begin{equation*}
    d(e, x) + d(e, x_0) < \delta + r
\end{equation*}
Z nierówności trójkąta zauważamy że:
\begin{equation*}
    d(x, x_0) \leq d(e, x) + d(e, x_0) < \delta + r = d(x, x_0) - r + r = d(x, x_0)
\end{equation*}
\begin{equation*}
    d(x, x_0) < d(x, x_0)
\end{equation*}
Sprzeczność, a zatem, dla \(x \in G\) mamy \(B(x, \delta) \subset G\) więc zbiór \(G\) jest otwarty.

\subsubsection{Dowód twierdzenia 4.}

\begin{theorem}
    Wnętrze zbioru \(A\) jest największym zbiorem otwartym zawartym w \(A\), a domknięcie zbioru \(A\) jest najmniejszym
    zbiorem domkniętym zawierającym \(A\).
\end{theorem}
\begin{proof}[Dowód]
    Niech \(V\) będzie wnętrzem zbioru \(A\). Sprawdzamy, że \(V\) jest zbiorem otwartym. Żeby się o tym przekonać
    wystarczy pokazać, że \(K(x, \epsilon) \subset V\). Dla dowolnego punktu \(x \in V\)
    istnieje jego otoczenie \(K(x, \epsilon)\) zawarte w \(A\) (definicja punktu wewnętrznego). Wiemy, że kula jest zbiorem
    otwartym, więc dowolny punkt \(y \in K(x, \epsilon)\) jest punktem wewnętrzym zbioru \(K(x, \epsilon)\). Skoro \(y\)
    jest punktem wewnętrznym zbioru \(K(x, \epsilon)\) to istnieje otoczenie \(K(y, \delta) \subset K(x, \epsilon) \)
    (znowu, definicja punktu wewnętrznego). Mamy więc następującą sytuację:
    \begin{equation*}
        K(y, \delta) \subset K(x, \epsilon)
    \end{equation*}
    oraz
    \begin{equation*}
        K(x, \epsilon) \subset A
    \end{equation*}
    stąd mamy zatem:
    \begin{equation*}
        K(y, \delta) \subset A
    \end{equation*}
    więc \(y\) będący dowolnym punktem \(K(x, \epsilon)\) jest punktem wewnętrznym zbioru \(A\), inaczej, \(y \in V\).
    Pokazaliśmy, więc, że jeśli \(y \in K(x, \epsilon)\) to również \(y \in V\) a zatem \(V\) jest zbiorem otwartym.
    Teraz pokażemy, że \(V\) jest największym zbiorem otwartym zawartym w \(A\). Niech \(U\) będzie dowolnym zbiorem
    otwartym zawartym w \(A\). Wtedy dowolny punkt \(x \in U\) jest punktem wewnętrznym zbioru \(U\), ściślej mówiąc,
    dla dowolnego \(x\) w \(U\) istnieje otoczenie \(K(x, \epsilon)\) zawarte w \(U\). Mamy zatem:
    \begin{equation*}
        K(x, \epsilon) \subset U
    \end{equation*}
    \begin{equation*}
        U \subset A
    \end{equation*}
    tak więc:
    \begin{equation*}
        K(x, \epsilon) \subset A
    \end{equation*}
    a to oznacza, że \(x\) jest punktem wewnętrznym zbioru \(A\), czyli \(x \in V\). Ostatecznie zatem:
    \begin{equation*}
        x \in U \implies x \in V \iff U \subset V
    \end{equation*}
    i \(V\) jest największym zbiorem otwartym zawartym w \(A\).
    \newline
    Sprawdzamy, że \(\overline{A}\) jest najmniejszym zbiorem domkniętym zawierającym zbiór \(A\).
    Z definicji \(\overline{A}\) zbiór \(X \setminus \overline{A}\) składa się ze wszystkich punktów zewnętrznych
    zbioru \(A\). Wobec tego \(X \setminus \overline{A}\) jest wnętrzem zbioru \(X \setminus A\). Ponieważ wnętrze
    zbioru \(X \setminus A\) jest największym zbiorem otwartym zawartym w \(X \setminus A\), więc \(\overline{A}\)
    jest najmniejszym zbiorem domkniętym zawierającym zbiór \(A\).
\end{proof}

\subsubsection{Jeśli zbiór jest domknięty i ograniczony w \(\mathbb{R}\) to kresy górne i dolne są w nim zawarte}


\section{Granica i ciągłość funkcji}
\section{Własności funkcji ciągłych}




\end{document}
