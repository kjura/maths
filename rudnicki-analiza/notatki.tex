\documentclass{article}
\usepackage{amsmath}
\usepackage{amssymb}
\usepackage{amsthm}
\usepackage[T1]{fontenc}


\newtheorem*{conjecture}{Przypuszczenie}
\newtheorem*{theorem}{Twierdzenie}
\newtheorem{lemma}{Lemat}


\makeatletter
\newcommand*{\rom}[1]{\expandafter\@slowromancap\romannumeral #1@}
\makeatother


\DeclareMathOperator{\interior}{int}

\begin{document}
\tableofcontents


\section{Wstęp do matematyki}
\subsection{Elementy logiki i teorii zbiorów}

\subsection{Funkcje i relacje}

\subsection{Zbiory liczbowe}
\subsubsection{Liczby wymierne i rzeczywiste}
\subsubsection{Sformułowanie aksjomatu ciągłości przy użyciu przekroju Dedekinda}
Sformułowanie aksjomatu ciągłości w inny sposób niż ten podany w książce. Można użyć do tego
przekroju Dedekinda (Rudnicki nie podaje tego explicite).

\begin{theorem}
    Jeżeli \(A\) i \(B\) są niepustymi podzbiorami \(\mathbb{R}\) takimi, że:
    \begin{equation}
        \mathbb{R} = A \cup B, \ \ A \cap B = \varnothing,
    \end{equation}
    \begin{equation}
        (x \in A \ \ \land \ \ y \in B) \implies x < y,
    \end{equation}
    to albo zbiór \(A\) ma element największy, albo zbiór \(B\) ma element najmniejszy.
\end{theorem}
\begin{proof}[Dowód]
    Najpierw warto napisać dla jakich sytuacji mamy podział dwóch zbiorów na cały przekrój liczb rzeczywistych.
    Suma tych dwóch zbiorów zadziała dla tychże dwóch sytuacji (domkniętość przedziałów):
    \begin{equation*}
        A=(-\infty, t), \ B=[t,\infty) \ \ \mbox{lub} \ \ A=(-\infty, t], \ B=(t,\infty)
    \end{equation*}
    Zbiór \(B\) jest ograniczony z dołu (elementy zbioru \(A\) są mniejsze, więc znajdziemy ograniczenie dla \(B\) z dołu), 
    więc ma kres dolny (aksjomat ciągłości). Niech \(b = \inf(B)\).
    Ponieważ dowolny element \(a \in A\) ogranicza zbiór \(B\) z dołu, więc \(a \leq b\). Jeżeli \(b \in A\), to z ostatniej
    nierówności wynika, że \(b\) jest elementem największym zbioru \(A\). Jeżeli \(b \in B\), to z definicji kresu dolnego
    \(b\) jest elementem najmniejszym zbioru \(B\). 
\end{proof}

\section{Przestrzenie metryczne II}
\subsection{Zbiory otwarte i domknięte}
\subsubsection{Definicje punktów wewnętrznych, brzegowych i zewnętrznych}
Ustalmy pewną przestrzeń metryczną \((X, d)\). Niech \(A\) będzie podzbiorem tej przestrzeni. Punkt \(x \in A\) nazywamy \emph{punktem wewnętrznym}
zbioru \(A\), jeżeli istnieje takie otoczenie \(K(x, r)\) punktu \(x\), że \(K(x,r) \subset A\). Punkt \(x \in X\) nazywamy
\emph{punktem zewnętrznym} zbioru \(A\), jeżeli istnieje takie otoczenie \(K(x, r)\) punktu \(x\),
że \(K(x, r) \cap A = \emptyset\). Punkt \(x \in X\) nazywamy \emph{punktem brzegowym} zbioru \(A\), jeżeli nie jest punktem
wewnętrznym i nie jest punktem zewnętrznym zbioru \(A\). Z definicji punktu brzegowego wynika, że \(x\) jest punktem
brzegowym, jeżeli dla każdego otoczenia \(K(x, r)\) punktu \(x\) mamy:
\begin{equation*}
    K(x, r) \cap A \neq \emptyset
\end{equation*}
\begin{equation*}
    K(x, r) \cap X \setminus A \neq \emptyset
\end{equation*}
\begin{center}
    \emph{Dodać szczegółowy warunek dla punktu zewnętrznego z definicji p. brzeg i p. wewn}
\end{center}
\newpage
\subsubsection{Przykład 3: Kula jest zbiorem domkniętym}

Kula \(\overline{B}(x_0 , r)\) jest zbiorem domkniętym w przestrzeni metrycznej \((X, d)\).
Wystarczy pokazać, że zbiór \(G = X \setminus \overline{B}(x_0 , r)\)
jest otwarty. Niech \(x \in G\). Wtedy \(d(x, x_0) > r\). Przyjmijmy, że \(\delta = d(x, x_0) - r\). 
Założymy nie wprost, że istnieje przynajmniej jeden element \(e\)
w zbiorze \(B(x, \delta)\) który nie występuje w \(G\). Wtedy dla pewnego elementu \(e\) mamy:
\begin{equation*}
    e \in B(x , \delta)
\end{equation*}
\begin{equation*}
    e \in \overline{B}(x_0 , r)
\end{equation*}
Oznacza to, że pewien punkt \(e\) spełnia dwie nierówności:
\begin{equation*}
    d(e, x) < \delta
\end{equation*}
\begin{equation*}
    d(e, x_0) \leq r 
\end{equation*}
Dodając je stronami otrzymujemy:
\begin{equation*}
    d(e, x) + d(e, x_0) < \delta + r
\end{equation*}
Z nierówności trójkąta zauważamy że:
\begin{equation*}
    d(x, x_0) \leq d(e, x) + d(e, x_0) < \delta + r = d(x, x_0) - r + r = d(x, x_0)
\end{equation*}
\begin{equation*}
    d(x, x_0) < d(x, x_0)
\end{equation*}
Sprzeczność, a zatem, dla \(x \in G\) mamy \(B(x, \delta) \subset G\) więc zbiór \(G\) jest otwarty.

\subsubsection{Dowód twierdzenia 4.}

\begin{theorem}
    Wnętrze zbioru \(A\) jest największym zbiorem otwartym zawartym w \(A\), a domknięcie zbioru \(A\) jest najmniejszym
    zbiorem domkniętym zawierającym \(A\).
\end{theorem}
\begin{proof}[Dowód]
    Niech \(V\) będzie wnętrzem zbioru \(A\). Sprawdzamy, że \(V\) jest zbiorem otwartym. Żeby się o tym przekonać
    wystarczy pokazać, że \(K(x, \epsilon) \subset V\). Dla dowolnego punktu \(x \in V\)
    istnieje jego otoczenie \(K(x, \epsilon)\) zawarte w \(A\) (definicja punktu wewnętrznego). Wiemy, że kula jest zbiorem
    otwartym, więc dowolny punkt \(y \in K(x, \epsilon)\) jest punktem wewnętrzym zbioru \(K(x, \epsilon)\). Skoro \(y\)
    jest punktem wewnętrznym zbioru \(K(x, \epsilon)\) to istnieje otoczenie \(K(y, \delta) \subset K(x, \epsilon) \)
    (znowu, definicja punktu wewnętrznego). Mamy więc następującą sytuację:
    \begin{equation*}
        K(y, \delta) \subset K(x, \epsilon)
    \end{equation*}
    oraz
    \begin{equation*}
        K(x, \epsilon) \subset A
    \end{equation*}
    stąd mamy zatem:
    \begin{equation*}
        K(y, \delta) \subset A
    \end{equation*}
    więc \(y\) będący dowolnym punktem \(K(x, \epsilon)\) jest punktem wewnętrznym zbioru \(A\), inaczej, \(y \in V\).
    Pokazaliśmy, więc, że jeśli \(y \in K(x, \epsilon)\) to również \(y \in V\) a zatem \(V\) jest zbiorem otwartym.
    Teraz pokażemy, że \(V\) jest największym zbiorem otwartym zawartym w \(A\). Niech \(U\) będzie dowolnym zbiorem
    otwartym zawartym w \(A\). Wtedy dowolny punkt \(x \in U\) jest punktem wewnętrznym zbioru \(U\), ściślej mówiąc,
    dla dowolnego \(x\) w \(U\) istnieje otoczenie \(K(x, \epsilon)\) zawarte w \(U\). Mamy zatem:
    \begin{equation*}
        K(x, \epsilon) \subset U
    \end{equation*}
    \begin{equation*}
        U \subset A
    \end{equation*}
    tak więc:
    \begin{equation*}
        K(x, \epsilon) \subset A
    \end{equation*}
    a to oznacza, że \(x\) jest punktem wewnętrznym zbioru \(A\), czyli \(x \in V\). Ostatecznie zatem:
    \begin{equation*}
        x \in U \implies x \in V \iff U \subset V
    \end{equation*}
    i \(V\) jest największym zbiorem otwartym zawartym w \(A\).
    \newline
    Chcemy sprawdzić, że \(\overline{A}\) jest \emph{najmniejszym} zbiorem domkniętym zawierającym zbiór \(A\).
    Z definicji \(\overline{A} = W \cup B\), tzn. domknięcie zbioru jest sumą zbioru punktów wewnętrznych i zbioru
    punktów brzegowych. Biorąc w takim razie dopełnienienie domknięcia \(A\) otrzymujemy:
    \begin{equation*}
        \overline{A}^{c} = W^c \cap B^c = X \setminus \overline{A}
    \end{equation*}
    Zbiór \(X \setminus \overline{A}\) posiada wszystkie punkty zewnętrzne zbioru \(A\). Możemy na to popatrzeć również
    w inny sposób. Zbiór \(X \setminus \overline{A}\) jest \emph{wnętrzem} zbioru \(X \setminus A\) (uwaga na poziome kreski).
    Wnętrze zbioru to największy zbiór otwarty zawarty w tym zbiorze, zatem \(X \setminus \overline{A} \subset X \setminus A\)
    W takim razie  \(\overline{A}\) jest domknięty i \(A \subset \overline{A}\),
    gdzie użyliśmy faktu: jeśli \(A\) jest podzbiorem \(B\) to dopełnienie \(B\) jest podzbiorem dopełnienia \(A\).
    Pozostaje pytanie: Dlaczego \(\overline{A}\) jest w takim razie \emph{najmniejszym} zbiorem domkniętym zawierającym zbiór \(A\).

    \begin{lemma}
        Każdy punkt brzegowy zbioru domkniętego \(F\) należy do tego zbioru.
    \end{lemma}
    \begin{proof}[Dowód]
        Załóżmy, nie wprost, że istnieje punkt brzegowy \(x\) zbioru domkniętego \(F\), który nie należy do tego zbioru.
        Wtedy \(x \in X \setminus F\). Zbiór \(X \setminus F\) jest otwarty więc dla punktu \(x\) istnieje otoczenie
        takie, że \(K(x, r) \subset X \setminus F\). Wtedy dowolny element otoczenia \(K(x, r)\) zawiera się w
        \(X \setminus F\) i \emph{nie} zawiera się w \(F\), tzn. \(K(x, r) \cap F = \emptyset\) co przeczy
        definicji punktu brzegowego.
    \end{proof}
    \begin{lemma}
        Niech \((X, d)\) będzie pewną przestrzenią metryczną. Wtedy jeśli \(A\) i \(F\) są podzbiorami tej przestrzeni
        metrycznej oraz \(A \subset F\) to \(F\) zawiera w sobie domknięcie \(A\).
    \end{lemma}
    \begin{proof}
        Chcemy pokazać, że zbiór punktów wewnętrznych oraz zbiór punktów brzegowych zawiera się w zbiorze domkniętym \(F\).
        Zbiór punktów wewnętrznych to największy zbiór otwarty zawarty w \(A\) więc \(\interior{(A)} \subset A \subset F\).
        Teraz pokazujemy, że zbiór punktów brzegowych również zawiera się w \(F\). Załóżmy nie wprost, że istnieje
        punkt brzegowy \(x\) zbioru \(A\), który nie jest elementem zbioru \(F\). Wtedy \(x\) należy do zbioru otwartego
        \(X \setminus F\). To znaczy, że \(x\) jest punktem zewnętrznym zbioru \(F\) czyli istnieje otoczenie \(K(x, r)\)
        takie, że \(K(x,r) \cap F = \emptyset\). Wtedy dowolny punkt należący do tego otoczenia \emph{nie} należy do \(F\).
        W takim razie dowolny punkt tego otoczenia nie należy do \(A\) co daje nam sprzeczność z założeniem: \(K(x, r) \cap A \neq \emptyset\). 
    \end{proof}
\end{proof}

\subsubsection{Jeśli zbiór jest domknięty i ograniczony w \(\mathbb{R}\) to kresy górne i dolne są w nim zawarte}
\begin{theorem}
    Niech \(A\) będzie domkniętym i ograniczonym podzbiorem \(\mathbb{R}\). Jeżeli \(M = \sup(A)\), to 
    \(M \in A\).
\end{theorem}
\begin{proof}[Dowód]
    Liczba \(M\) jest kresem górnym \(A\), więc dla dowolnego \(n\) naturalnego istnieje taki
    punkt \(x_n\) ze zbioru \(A\), że (dopisek, skoro dla dowolnego \(\epsilon\) to działa to można podstawić \(\frac{1}{n}\)):
    \begin{equation*}
        M - \frac{1}{n} \leq x_n \leq M.
    \end{equation*}
    Stąd \(M = \lim\limits_{n \to \infty} x_n\) i ponieważ zbiór \(A\) jest domknięty, więc \(M \in A\).
    % Analogicznie postępujemy dla infimum. Dla celów poznawczych autor rozpisze się znacznie bardziej niż jest tego potrzeba.
    % Liczba \(m\) jest kresem dolnym \(A\), więc dla dowolnego \(n\) naturalnego
    % istnieje taki punkt \(x_n\) ze zbioru \(A\), że:
    % \begin{equation*}
    %     5
    % \end{equation*}
\end{proof}
\begin{center}
    PONIŻSZE NIE SPRAWDZONE JESZCZE
\end{center}
\begin{theorem}
    Niech \(A\) będzie domkniętym i ograniczonym podzbiorem \(\mathbb{R}\). Jeżeli \(m = \inf(A)\), to 
    \(m \in A\).
\end{theorem}
\begin{proof}[Dowód]
    Liczba \(m\) jest ograniczeniem dolnym zbioru \(A\), więc dowolny element tego zbioru jest większy bądź równy niż
    \(m\). Niech \(epsilon = \frac{1}{n}\) dla dowolnego \(n \in \mathbb{N}\). 
    Z definicji kresu dolnego istnieje wtedy taki element \(x_n\) że:
    \begin{equation*}
        m + \epsilon > x_n
    \end{equation*}
    Łącząc powyższe nierówności otrzymujemy:
    \begin{equation*}
        m + \epsilon > x_n \geq m
    \end{equation*}
    Z twierdzenia o trzech ciągach wynika, że \(m = \lim\limits_{n \to \infty} x_n\). Z założenia, że zbiór \(A\) jest
    domknięty wynika, że granica \(m\) ciągu \(x_n\) jest elementem zbioru \(A\).
\end{proof}

\begin{center}
    PONIŻSZE NIE SPRAWDZONE JESZCZE
\end{center}
\subsubsection{Równoważność metryk i zwartość zbioru}
\begin{theorem}
    Jeżeli w przestrzeni metrycznej mamy dwie metryki równoważne, to zbiór zwarty w jednej metryce jest zbiorem zwartym w drugiej
    metryce.
\end{theorem}
\begin{proof}
    Niech metryki \(d_{1} \ \mbox{oraz} \ d_{2}\) będą równoważne w przestrzeni metrycznej \(X\).
    Dla dowolnego \(r > 0\) i punktu \(x\) w \(A\) istnieje \(\delta > 0\) taka że \(K(x, \delta)_{d_1} \subset K(x, r)_{d_2}\).
    Niech \(A\) będzie zwartym podzbiorem tej przestrzeni w metryce \(d_1\). Dowolny ciąg
    \((x_n)\) o wyrazach w \(A\) zawiera podciąg \((x_{p_n})\) zbieżny do \(x \in A\). Zatem prawie wszystkie wyrazy
    ciągu \((x_{p_n})\) należą do \(K(x, \delta)_{d_1}\). Stąd, prawie wszystkie wyrazy ciągu \((x_{p_n})\)
    należą do \(K(x, r)_{d_2}\). Zatem zbiór \(A\) jest też zwarty w metryce \(d_2\).
\end{proof}

\subsubsection{Zbiór liczb rzeczywistych nie jest zbiorem zwartym}
Zbiór liczb rzeczywistych nie jest zbiorem zwartym. Zauważmy, że ciąg \((x_n)\) o wyrazach \(x_n = n\) jest rozbieżny do
nieskończoności. Każdy podciąg ciągu rozbieżnego do nieskończoności jest również rozbieżny do nieskończoności więc \((x_n)\)
nie zawiera podciągu zbieżnego do punktu w \(\mathbb{R}\).

\subsection{Zbiór domknięty i ograniczony w \(\mathbb{R}^m\) a zwartość}
\begin{center}
    Poniższe jeszcze do przemyślenia
\end{center}
\begin{theorem}
    Zbiór domknięty i ograniczony w \(\mathbb{R}^m\) jest zbiorem zwartym.
\end{theorem}
\begin{proof}
    Niech \(A\) będzie domkniętym i ograniczonym podzbiorem \(\mathbb{R}^m\). 
    Rozważmy dowolny ciąg \((x_n)\) o wyrazach w \(A\). Wiadome, jest, że
    \emph{każdy ciąg ograniczony o wyrazach w przestrzeni} \(\mathbb{R}^m\) \emph{ma podciąg zbieżny} (2.2.5 Twierdzenie 11).
    W takim razie ciąg \((x_{p_n})\) - który jest podciągiem ciągu \((x_n)\) - ma pewną granicę \(x \in \mathbb{R}^m\).
    Granica ta musi być elementem zbioru \(A\) bo z założenia, \(A\) jest zbiorem domkniętym. 
\end{proof}

\subsubsection{Każdy zbiór zwarty jest domknięty}
\begin{theorem}
    Każdy zbiór zwarty jest domknięty.
\end{theorem}
\begin{proof}
    Niech \(A\) będzie zbiorem zwartym, a \((x_n)\) ciągiem zbieżnym punktów ze zbioru \(A\).
    Niech \(x = \lim\limits_{n \to \infty} x_n\). Dowolny podciąg ciągu \((x_n)\) ma również granicę \(x\) (bo \(x_n\) zbieżny).
    Ze zwartości zbioru \(A\) wynika, że \emph{istnieje} (dlaczego istnieje? Dlaczego nie każdy?) podciąg ciągu \((x_n)\)
    zbieżny do punktu ze zbioru \(A\). Wszystkie podciągi \(x_n\) mają taką samą granicę \(x\), a skoro istnieje podciąg
    którego granica jest punktem w \(A\) to znaczy, że wszystkie te podciągi mają granicę w \(A\) (bo jest jedna, unikatowa).
    Zatem \(x \in A\), a więc zbiór \(A\) jest domknięty bo granica dowolnego ciągu punktów z tego zbioru należy do tego zbioru. 
\end{proof}
\subsubsection{Domknięty podzbiór zbioru zwartego jest zwarty}
\begin{theorem}
    Domknięty podzbiór zbioru zwartego jest zwarty.
\end{theorem}
\begin{proof} 
    Niech \(B\) będzie domkniętym zbiorem zawartym w \(A\). Niech \(A\) również będzie zwarty. Weźmy dowolny ciąg (nie
    czynimy założenia o zbieżności) \((y_n)\) o wyrazach w zbiorze \(B\). Zauważmy, że dowolny wyraz ciągu w 
    \(B\) jest też wyrazem w \(A\) (zawarcie zbioru B w A). W takim razie ze zwartości \(A\), dowolny podciąg \((y_{p_n})\) jest 
    zbieżny do punktu \(y \in A\). Zbiór \(B\) jest zbiorem domkniętym, zatem granica podciągu \((y_{p_n})\)
    musi być elementem \(B\). Pokazaliśmy zatem, że dowolny ciąg \((y_n)\) wyrazów w \(B\) ma podciąg zbieżny
    do punktu w \(B\). Zatem \(B\) jest zbiorem zwartym.
\end{proof}
\begin{theorem}
    
\end{theorem}

\subsection{Przestrzeń zupełna}
\subsubsection{Warunek Cauchy'ego}

\begin{theorem}
    Każdy ciąg zbieżny spełnia warunek Cauchy'ego.
\end{theorem}
\begin{proof}
    Dowód polega na umiejętnym skorzystaniu z definicji (plus warunek trójkąta) oraz pamiętaniu o tym, że możemy używać
    różnych literach do oznaczania indeksu i jeśli jeden indeks będzie większy niż drugi, dopóki indeksy będą
    większe niż pewna liczba naturalna dająca nam nierówność z epsilonem, dopóty będzie można dodać do siebie nierówności.
    Niech zatem \((x_n)\) będzie zbieżnym ciągiem w pewnej przestrzeni metrycznej \((X, d)\). Niech \(x = \lim\limits_{n \to \infty} x_n\).
    Ustalmy \(\epsilon > 0\). Z definicji granicy mamy wtedy dla \(n \geq n_0\) oraz \(m \geq n_0\):
    \begin{equation*}
        d(x_n, x) < \frac{\epsilon}{2}
    \end{equation*}
    \begin{equation*}
        d(x_m, x) < \frac{\epsilon}{2}
    \end{equation*}
    dodając stronami nierówność wychodzimy na:
    \begin{equation*}
        d(x_n, x) + d(x_m, x) < \frac{\epsilon}{2} + \frac{\epsilon}{2} = \epsilon
    \end{equation*}
    i nierówność trójkąta:
    \begin{equation*}
        d(x_n, x_m) \leq d(x_n, x) + d(x_m, x) <  \epsilon
    \end{equation*}
    czyli ciąg zbieżny spełnia warunek Cauchy'ego.
\end{proof}

\begin{theorem}
    Każdy ciąg spełniający warunek Cauchy'ego jest ograniczony.
\end{theorem}
\begin{proof}
    Zakładamy, że ciąg \(x_n\) pewnej przestrzeni metrycznej spełnia warunek Cauchy'ego. Wtedy dla
    \(n_0\) mamy \(d(x_n , x_{n_{0}}) < \epsilon\) (obsadzamy wyraz z indeksem od którego nierówność w granicy jest spełniona).
    Teraz szukamy promienia \(r\), takiego który jest największy spośród: \(\epsilon\)-a i odległości między wyrazami ciągu od 
    początku ciągu (\(n = 1\)) aż do wyrazu tuż przed wyrazem z indeksem \(n_0\) i wyrazem o indeksie \(n_0\).
    \begin{equation*}
        r = \max\{\epsilon, d(x_1, x_{n_0}), d(x_2, x_{n_0}), \dots, d(x_{n_{0} - 1}, x_{n_0})\}
    \end{equation*}
    tzn. zbieramy razem epsilona oraz wszystkie wyrazy, które \emph{nie} spełniają nierówności z warunku Cauchy'ego i szukamy
    pośród tych wyrazów wyrazu maksymalnego. Jeżeli w ten sposób go dobierzemy, mamy gwarancję, że \emph{każda} odległość pomiędzy
    dowolnym wyrazem, a wyrazem o indeksie \(n_0\) będzie mniejsza lub większa niż nasze \(r\) (nierówność ostra implikuje nieostrą).
    Jest tak dlatego, że nasze \(r\) będzie albo równe epsilonowi albo większe (ściśle) od niego zatem \(r\) będzie większe od każdej
    odległości od wyrazu o indeksie \(n_0\) (w szczególności, większe niż odległości z warunku Cauchy'ego).
    Mamy wtedy \emph{dla dowolnego wyrazu}:
    \begin{equation*}
        d(x_n, x_{n_{0}}) \leq r
    \end{equation*}
    Stąd, znaleźliśmy warunek na kulę domkniętą obejmującą każdy wyraz ciągu \( x_n \in \overline{K}(x_{n_{0}}, r)\),
    zatem ciąg jest ograniczony.
\end{proof}

\subsubsection{Warunek Cauchy'ego a metryki jednostajnie równoważne}
\subsubsection{Zbieżność ciągów w metrykach a równoważność}
\subsubsection{Równoważność zupełności dla metryk}
\subsubsection{Przykład 6 - przestrzeń zupełna a liczby niewymierne}
\subsubsection{Każda przestrzeń zwarta jest zupełna}
Zaczniemy od lematu, który pomoże nam w końcowej fazie dowodu:
\begin{lemma}
    Niech \((x_n)\) będzie dowolnym ciągiem w przestrzeni metrycznej \(X\) a \((x_{p_{n}})\) będzie jego podciągiem.
    Wtedy:
    \begin{equation*}
        p_n \geq n
    \end{equation*}
    dla dowolnego \(n \in \mathbb{N}\).
\end{lemma}
\begin{proof}
    Dla \(n = 1\) nierówność \(p_1 \geq n = 1\) jest oczywista bo \(p_n \in \{1, 2, 3, 4, \dots\}\).
    Załóżmy, że  \(p_n \geq n\) dla \(n \in \mathbb{N}\). Pokażemy, że \(p_{n + 1} \geq n + 1\).
    DALSZA CZĘŚĆ POJAWI SIĘ TUTAJ
\end{proof}
\begin{theorem}
    Każda przestrzeń zwarta jest zupełna.
\end{theorem}
\begin{proof}
    Niech \((x_n)\) będzie ciągiem spełniającym waruenk Cauchy'ego, a \(\epsilon\) liczbą dodatnią. Przestrzeń \(X\) jest
    zwarta, zatem ciąg \((x_n)\) zawiera podciąg zbieżny \((x_{p_n})_{n \in \mathbb{N}}\). 
    Niech \( x = \lim\limits_{n \to \infty} x_{p_n}\). Z definicji granicy istnieje takie \(n_0 \in \mathbb{N}\), że:
    \begin{equation} \label{kazda_przestrzen_zwarta_jest_zupelna: 3}
        d(x_{p_n}, x) < \frac{\epsilon}{2} \ \ \mbox{dla} \ \ n \geq n_0.
    \end{equation}
    Z warunku Cauchy'ego istnieje taka liczba \(n_1 \in \mathbb{N}\), że:
    \begin{equation} \label{kazda_przestrzen_zwarta_jest_zupelna: 4}
        d(x_m , x_n) < \frac{\epsilon}{2} \ \ \mbox{dla} \ \ m \geq n_1 \ \ \mbox{i} \ \ n \geq n_1.
    \end{equation}
    Obierzmy \(n_2 = \max\{n_0 , n_1\}\). Zauważmy, że:
    \begin{equation*} 
        p_n \geq n \ \ \mbox{CZEMU??}
    \end{equation*}
    Łącząc zatem warunek trójkąta, oraz \eqref{kazda_przestrzen_zwarta_jest_zupelna: 3}
    i \eqref{kazda_przestrzen_zwarta_jest_zupelna: 4} otrzymujemy (CZEMU??????):
    \begin{equation*}
        d(x, x_m) \leq d(x, x_{p_n}) + d(x_{p_n}, x_m) < \frac{\epsilon}{2} + \frac{\epsilon}{2} = \epsilon \ \ \mbox{CZEMU??}
    \end{equation*}
    dla \(n \geq n_2\) i \(m \geq n_2\). Zatem:
    \begin{equation*}
        d(x, x_{m}) < \epsilon
    \end{equation*}
    dla \(m \geq n_2\), więc ciąg \((x_n)\) jest zbieżny.
\end{proof}

\section{Granica i ciągłość funkcji}
\section{Własności funkcji ciągłych}




\end{document}
