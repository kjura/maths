\documentclass{article}
\usepackage{amsmath}
\usepackage{amssymb}
\usepackage{amsthm}
\usepackage[T1]{fontenc}


\newtheorem*{conjecture}{Przypuszczenie}
\newtheorem*{theorem}{Twierdzenie}
\newtheorem{lemma}{Lemat}


\makeatletter
\newcommand*{\rom}[1]{\expandafter\@slowromancap\romannumeral #1@}
\makeatother


\begin{document}

\section{Wstęp do matematyki}
\subsection{Elementy logiki i teorii zbiorów}

\subsection{Funkcje i relacje}

\subsection{Zbiory liczbowe}
\subsubsection{Liczby wymierne i rzeczywiste}
\subsubsection{Sformułowanie aksjomatu ciągłości przy użyciu przekroju Dedekinda}
Sformułowanie aksjomatu ciągłości w inny sposób niż ten podany w książce. Można użyć do tego
przekroju Dedekinda (Rudnicki nie podaje tego explicite).

\begin{theorem}
    Jeżeli \(A\) i \(B\) są niepustymi podzbiorami \(\mathbb{R}\) takimi, że:
    \begin{equation}
        \mathbb{R} = A \cup B, \ \ A \cap B = \varnothing,
    \end{equation}
    \begin{equation}
        (x \in A \ \ \land \ \ y \in B) \implies x < y,
    \end{equation}
    to albo zbiór \(A\) ma element największy, albo zbiór \(B\) ma element najmniejszy.
\end{theorem}
\begin{proof}[Dowód]
    Najpierw warto napisać dla jakich sytuacji mamy podział dwóch zbiorów na cały przekrój liczb rzeczywistych.
    Suma tych dwóch zbiorów zadziała dla tychże dwóch sytuacji (domkniętość przedziałów):
    \begin{equation*}
        A=(-\infty, t), \ B=[t,\infty) \ \ \mbox{lub} \ \ A=(-\infty, t], \ B=(t,\infty)
    \end{equation*}
    Zbiór \(B\) jest ograniczony z dołu (elementy zbioru \(A\) są mniejsze, więc znajdziemy ograniczenie dla \(B\) z dołu), 
    więc ma kres dolny (aksjomat ciągłości). Niech \(b = \inf(B)\).
    Ponieważ dowolny element \(a \in A\) ogranicza zbiór \(B\) z dołu, więc \(a \leq b\). Jeżeli \(b \in A\), to z ostatniej
    nierówności wynika, że \(b\) jest elementem największym zbioru \(A\). Jeżeli \(b \in B\), to z definicji kresu dolnego
    \(b\) jest elementem najmniejszym zbioru \(B\). 
\end{proof}


\end{document}
