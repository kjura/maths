\documentclass{article}
\usepackage{amsmath}


\begin{document}


Can I get a hint (just a hint for now please, no solution) how should I approach this exercise?


Let $A, B, C$ be pairwise independent i.e 

\begin{equation*}
    P(A \cap B) = P(A)P(B)
\end{equation*}
\begin{equation*}
    P(A \cap C) = P(A)P(C)
\end{equation*}
\begin{equation*}
    P(B\cap C) = P(B)P(C)
\end{equation*}

The question is to check if the below is possible (if these events are independent):

\begin{equation*}
    P\Big( (A \cap B) \cap C\Big) \stackrel{?}{=} P(A)P(B)P(C) 
\end{equation*}


Can I get a hint how to proceed? I was thinking about using chain rule (https://en.wikipedia.org/wiki/Chain_rule_(probability))
But I left it as I did not feel it's a good direction to take in this one. Also I would need to be sure that 'a posteriori' events
are not equal to zero and I did not assume that (or maybe this is the way?).


%${\displaystyle \mathrm {P} \left(A_{n}\cap \ldots \cap A_{1}\right)=\mathrm {P} \left(A_{n}|A_{n-1}\cap \ldots \cap A_{1}\right)\cdot \mathrm {P} \left(A_{n-1}\cap \ldots \cap A_{1}\right)}$

\end{document}