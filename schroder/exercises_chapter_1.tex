\documentclass{article}
\usepackage{amsmath}
\usepackage{amssymb}
\usepackage{amsthm}


\newtheorem*{conjecture}{Conjecture}
\newtheorem*{theorem}{Theorem}
\newtheorem*{lemma}{Lemma}


\makeatletter
\newcommand*{\rom}[1]{\expandafter\@slowromancap\romannumeral #1@}
\makeatother


\begin{document}

\section*{1-1}

We will use field axioms in \(\mathbb{R}\)  to show that \((-1)(-1) = 1\). First let us observe that:

\begin{equation*}
    0 = (-1) \cdot 0  \ \ \ \mbox{\emph{A neutral element 1 for multiplication}}
\end{equation*}
\begin{equation*}
    = (-1) \cdot (1 + (-1)) \ \ \ \mbox{\emph{Additive inverse}}
\end{equation*}
\begin{equation*}
    = (-1) \cdot 1 + (-1) \cdot (-1) \ \ \ \mbox{\emph{Left distributivity of addition}}
\end{equation*}
\begin{equation*}
    = 1 \cdot (-1) + (-1) \cdot (-1) \ \ \ \mbox{\emph{Commutativity of multiplication}}
\end{equation*}
\begin{equation*}
    = (-1) + (-1) \cdot (-1) \ \ \ \mbox{\emph{A neutral element 1 for multiplication}}
\end{equation*}

We end up with the following equation:
\begin{equation*}
    (-1) + (-1) \cdot (-1) = 0
\end{equation*}
Adding \(1\) on both sides and using \emph{Additive inverse} and \emph{Associativity of addition}:
\begin{equation*}
    (1 + (-1)) + (-1)(-1) = 0 + (-1)(-1) = (-1)(-1) = 1
\end{equation*}

\section*{1-2}

Let \(x, y, z \in \mathbb{R}\). We need to prove that:
\begin{equation*}
    (x + y)z = xz = yz
\end{equation*}
knowing from our field axioms that operation \((\cdot)\) is \emph{left} distributive. Let us make an observation
that by definition of field, \( (x + y) \in \mathbb{R}\). This means that \((x+y)z = z(x+y)\) because of \emph{Commutativity of multiplication}.
And then using \emph{Left distributivity of addition} we arrive to:
\begin{equation*}
    (x+y)z = z(x+y) = zx + zy
\end{equation*}

\section*{1-3}

Let \(x \in \mathbb{R}\) and \(x', \  \overline{x}\) be multiplicative inverses of \(x\). 
We will show that multiplicative inverses are unique:

\begin{equation*}
    x' = 1 \cdot x' \ \ \ \mbox{\emph{A neutral element 1 for multiplication}}
\end{equation*}
\begin{equation*}
    = (x \overline{x}) \cdot x' \ \ \ \mbox{\emph{Multiplicative inverse}}
\end{equation*}
\begin{equation*}
    = (\overline{x} x) \cdot x' \ \ \ \mbox{\emph{Commutativity of multiplication}}
\end{equation*}
\begin{equation*}
    = \overline{x} \cdot (xx') \ \ \ \mbox{\emph{Associativity of multiplication}}
\end{equation*}
\begin{equation*}
    = \overline{x} \cdot 1 \ \ \ \mbox{\emph{Multiplicative inverse}}
\end{equation*}
\begin{equation*}
    = \overline{x} \ \ \ \mbox{\emph{A neutral element 1 for multiplication}}
\end{equation*}

We have shown that \(\overline{x} = x'\) which means that if multiplicative inverse exists, it is unique.

\section*{1-4}

If \(0\) had a multiplicative inverse then we would have \(0 \cdot 0^{-1} = 1\). But that 
would mean that we could multiply an element of reals field \(0^{-1}\) by \(0\). According to
the field axioms we should have then had \(0^{-1} \cdot 0 = 0\) but this would contradict our
assumption that \(0 \cdot 0^{-1} = 1\).

\section*{1-5}

We know that \((xy)\) is invertible because we start with the assumption that \(x \neq 0\) and \(y \neq 0\).
Which means that \(x \cdot y \neq 0\). Hence, \(xy\) is some real number, and from axioms we know that
every real number other than zero has an inverse.  

Let us first use \emph{Existence of multiplicative inverses} to settle the ground for our proof:
\begin{equation*}
	(xy)^{-1}(xy) = 1
\end{equation*}
Now we will use the chain of properties of real numbers:
\begin{equation*}
	(xy)^{-1}(xy) = 1 \cdot 1 \ \ \ \emph{Existence of the multiplicative identity}
\end{equation*}
\begin{equation*}
	(xy)^{-1}(xy) = (x^{-1}x) \cdot (yy^{-1}) \ \ \ \emph{Existence of multiplicative inverses}
\end{equation*}
\begin{equation*}
	(xy)^{-1}(xy) = (x^{-1}y^{-1}) \cdot (xy) \ \ \ \emph{Associative and commutative laws for multiplication (lots of)} 
\end{equation*}
\begin{equation*}
	(xy)^{-1} = (x^{-1}y^{-1}) \ \ \ \emph{Existence of multiplicative inverses and associative law for multiplication} 
\end{equation*}
and our proof is done (see also Spivak Chapter 1 3.(iii)).


\section*{1-6}

\subsection*{(a)}
We will prove the following formula:
\begin{equation*}
    (a + b)^2 = a^2 + 2ab + b^2
\end{equation*}
Let us justify each step using field axioms of \(\mathbb{R}\):
\begin{equation*}
    (a + b)^2 = (a + b)(a + b) \ \ \ \mbox{\emph{Definition of exponentiation}}
\end{equation*}
\begin{equation*}
    = \Big(a \cdot (a + b)\Big) + \Big(b \cdot (a + b)\Big) \ \ \ \mbox{\emph{Distributivity over addition}}
\end{equation*}
\begin{equation*}
    = \Big( (a \cdot a) + (a \cdot b)\Big) + \Big( (b \cdot a) + (b \cdot b)\Big) \ \ \ \mbox{\emph{Distributivity over addition}} 
\end{equation*}
\begin{equation*}
    = \Big( a^2 + (a \cdot b)\Big) + \Big( (b \cdot a) + b^{2}\Big) \ \ \ \mbox{\emph{Definition of exponentiation}} 
\end{equation*}
\begin{equation*}
    = \Big( a^2 + (a \cdot b)\Big) + \Big( (a \cdot b) + b^{2}\Big) \ \ \ \mbox{\emph{Commutativity of multiplication}} 
\end{equation*}
\begin{equation*}
    = \Bigg(\Big(a^2 + (a \cdot b)\Big) + (a \cdot b)\Bigg)  + \Big(b^{2}\Big) \ \ \ \mbox{\emph{Associativity of addition}} 
\end{equation*}
\begin{equation*}
    = \Bigg(\Big(a^2 + 2a b)\Big)\Bigg)  + \Big(b^{2}\Big) \ \ \ \mbox{\emph{Associativity of addition}} 
\end{equation*}
\begin{equation*}
    = a^2 + 2ab + b^2
\end{equation*}

\subsection*{(b)}

We will prove the following formula:
\begin{equation*}
    (a - b)^2 = a^2 - 2ab + b^2
\end{equation*}
Let us justify each step using field axioms of \(\mathbb{R}\):
\begin{equation*}
    \Big(a + (-b)\Big)^2 = \Big(a + (-b)\Big) \Big(a + (-b)\Big) \ \ \ \mbox{\emph{Definition of exponentiation}}
\end{equation*}
\begin{equation*}
    = \Big(a \cdot a + a \cdot (-b)\Big) + \Big((-b) \cdot a + (-b) \cdot (-b)\Big) \ \ \ \mbox{\emph{Distributivity over addition}}
\end{equation*}
\begin{equation*}
    = \Bigg(a^2 + \Big(a \cdot (-b)\Big)\Bigg) + \Bigg(\Big( (-b) \cdot a \Big) + b^2 \Bigg) \ \ \ \mbox{\emph{Definition of exponentiation}}
\end{equation*}
\begin{equation*}
    =   \Bigg(a^2 + \Big(a \cdot (-b)\Big) + \Big( (-b) \cdot a \Big) \Bigg) + \Bigg(b^2 \Bigg) \ \ \ \mbox{\emph{Associativity of addition}}
\end{equation*}
\begin{equation*}
    = \Bigg(a^2 + 2 \Big(a \cdot (-b)\Big) \Bigg) + \Bigg(b^2 \Bigg) = a^2 -2ab + b^2
\end{equation*}
where in the last step we just present our result in a more convienient notation. With that the proof is done.


\subsection*{(c)}

We will prove the following formula:
\begin{equation*}
    (a + b)(a - b) = a^2 - b^2
\end{equation*}
\newline
\begin{equation*}
    (a + b)(a - b) = (a + b)(a +  (-b))
\end{equation*}
\begin{equation*}
    = \Big( a \cdot a + \big(a \cdot (-b)\big)\Big) + \Big( b \cdot a + \big(b \cdot (-b)\big)\Big) \ \ \ \mbox{\emph{Distributivity over addition}}
\end{equation*}
\begin{equation*}
    = \Big( a^2 + \big(a \cdot (-b)\big)\Big) + \Big( b \cdot a + (-b^2) \Big) \ \ \ \mbox{\emph{Definition of exponentiation}}
\end{equation*}
\begin{equation*}
    = \Big( a^2 + \big(a \cdot (-b)\big) + b \cdot a \Big) + (-b^2)  \ \ \ \mbox{\emph{Associativity of addition}}
\end{equation*}
\begin{equation*}
    = \Big( a^2 + 0 \Big) + (-b^2)  \ \ \ \mbox{\emph{Existence of additive inverses}}
\end{equation*}
\begin{equation*}
    = a^2  + (-b^2) = a^2 - b^2  \ \ \ \mbox{\emph{Neutral element for addition}}
\end{equation*}
As shown by field axioms it follows that \((a + b)(a - b) = a^2 - b^2 \)

\section*{1-10}
\subsection*{(a)}
\subsubsection*{\(x\) is positive}

Let \(x\) be positive. Then \(x \in \mathbb{R^{+}}\). We want to show that:

\begin{equation*}
    x > 0 \iff 0 < x \iff (x - 0) \in \mathbb{R^{+}}
\end{equation*}

Observe that by the field axioms of reals:

\begin{equation*}
    x - 0 = x + (-0) = x + ((-1) \cdot 0) = x + 0 = x 
\end{equation*}

From our initial assumption we know that \(x\) is positive, hence \(x - 0\) is also positive and
that concludes the proof in one direction. For the other direction, assume now that \(x > 0\). It means then that
\((x - 0) \in \mathbb{R^{+}}\). Again, from the observation above, this is just an equivalent way of saying that
\(x \in \mathbb{R^{+}}\), so \(x\) is positive.

\subsubsection*{\(x\) is negative}

Let \(x\) be negative. Then \(-x \in \mathbb{R^{+}}\). Using the same observation from the positive case above:
\begin{equation*}
    (0 - x) = 0 + (-x) = -x \in \mathbb{R^{+}}
\end{equation*}
And so, if \((0 - x) = -x\) and \(-x \in \mathbb{R^{+}}\) it follows that \(x < 0\). For the opposite direction, assume that
\(x < 0\). Then \((0 - x) \in \mathbb{R^{+}}\). From here, one can easily see that \((-x) \in \mathbb{R^{+}}\)

\subsection*{(b)}

Let \(x, y, z \in \mathbb{R^{+}}\) and let \(x \leq y\). For the first case, assume that \(x = y\). Then from the definition it is clear
to see that \(x \leq y\). Assume now that \(x < y\). Then \((y - x) \) is a positive real number.  We need to show that \(x + z \leq y + z\). In other words, we need to show that
\((y + z) - (x + z)\) is a positive real number. From the field axioms of reals we see that:
\begin{equation*}
    (y + z) - (x + z) = (y + z) + (-(x + z)) = (y + z) + (-x - z) = y + 0 - x = y - x
\end{equation*} 
But we know that this is a positive real number from \(x < y\). Thus \( \Big((y + z) - (x + z)\Big) \in \mathbb{R^{+}}\) and \(x + z \leq y + z\).

\section*{1-11}
\subsection*{(a)}

Let \(y < 0\). First, assume that \(x = 0\). Then we have the following:
\begin{equation*}
    |x||y| = x \cdot |y| = 0 \cdot |x| = 0 = |0| = |0 \cdot y| = |xy|
\end{equation*}
Nowm assume that \(x > 0\). Observe that \(xy\) is a negative number. So we have that \(|xy| = -(xy)\). We show the following:
\begin{equation*}
    |x||y| = x \cdot |y| = x \cdot (-y) = -(xy) = |xy|
\end{equation*}

\subsection*{(b)}

Let \(x \in \mathbb{R}\). Assume first that \(x \geq 0\). Then we have:
\begin{equation*}
    |x| = x \geq 0 \geq -x
\end{equation*}
Assume now that \(x < 0\). Then clearly:
\begin{equation*}
    |x| = -x  \geq -x 
\end{equation*}

\subsection*{(c)}
\paragraph*{TO DOOOOOOOOOOOOOOOOOOOOOOOOOOOOOOOOOOOOOOOOOOOOOOOOOOOOOOOOoo}

\section*{1-12}
Let \(a, b \in \ I \cap J\) and \(a < b\). Let \(x \in \mathbb{R}\). Assume \(a < x < b\), then:
\begin{equation*}
    a < x < b \iff x \in (a, b) \iff \{x \in \mathbb{R} : a < x < b\} \iff \mbox{????} \iff x \in \{x \in \mathbb{R} : x \in I \ \ \mbox{and} \ \ x \in J\}
\end{equation*}

\section*{1-13}
Let \(a < b\) and let \(x, y \in [a, b]\). Observe that \(x, y\) obey both these relationships:
\begin{equation*}
    a \leq x \leq b \ \ \mbox{and} \ \ a \leq y \leq b
\end{equation*}

So from \(a \leq x\) and \(y \leq b\) we can conclude that \(a -x + y - b \leq 0\) and this implies \(y - x \leq b - a\).
Similiarly, from  \(x \leq b\) and \(a \leq y\) we have \(x - y \leq b - a\). Assume now without loss of generality that
\(x \geq y\). Then using the inequalities we showed before we have:
\begin{equation*}
    |x - y| = x - y \leq b - a
\end{equation*}
For the second case, assume \(x < y\):
\begin{equation*}
    |x - y| = -(x - y) = y - x \leq b - a
\end{equation*}

\section*{1-15}
Let \(A \subseteq \mathbb{R}\) be bounded below and let \(s, t, \in \mathbb{R}\) both be infima of \(A\). Then for all lower bounds \(l\)
of \(A\) we have that \(l \leq s\). Specifically, if \(t\) is \emph{an} infimum, then the former observation applies and \(t \leq s\) (because t - 
being \emph{an} infimum - is also a lower bound). A similiar argument can be made for \(t\), that is, \(t\) is greater than or equal to
every lower bound, hence \(s \leq t\). Applying antisymmetric property of an order relation \(\leq\) we conclude that
\(s = t\), so infima are unique. 

\end{document}
