\documentclass{article}
\usepackage{amsmath}
\usepackage{amssymb}
\usepackage{amsthm}


\newtheorem*{conjecture}{Conjecture}
\newtheorem*{theorem}{Theorem}


\makeatletter
\newcommand*{\rom}[1]{\expandafter\@slowromancap\romannumeral #1@}
\makeatother


\begin{document}

\section*{1-1}

We will use field axioms in \(\mathbb{R}\)  to show that \((-1)(-1) = 1\). First let us observe that:

\begin{equation*}
    0 = (-1) \cdot 0  \ \ \ \mbox{\emph{A neutral element 1 for multiplication}}
\end{equation*}
\begin{equation*}
    = (-1) \cdot (1 + (-1)) \ \ \ \mbox{\emph{Additive inverse}}
\end{equation*}
\begin{equation*}
    = (-1) \cdot 1 + (-1) \cdot (-1) \ \ \ \mbox{\emph{Left distributivity of addition}}
\end{equation*}
\begin{equation*}
    = 1 \cdot (-1) + (-1) \cdot (-1) \ \ \ \mbox{\emph{Commutativity of multiplication}}
\end{equation*}
\begin{equation*}
    = (-1) + (-1) \cdot (-1) \ \ \ \mbox{\emph{A neutral element 1 for multiplication}}
\end{equation*}

We end up with the following equation:
\begin{equation*}
    (-1) + (-1) \cdot (-1) = 0
\end{equation*}
Adding \(1\) on both sides and using \emph{Additive inverse} and \emph{Associativity of addition}:
\begin{equation*}
    (1 + (-1)) + (-1)(-1) = 0 + (-1)(-1) = (-1)(-1) = 1
\end{equation*}

\section*{1-2}

Let \(x, y, z \in \mathbb{R}\). We need to prove that:
\begin{equation*}
    (x + y)z = xz = yz
\end{equation*}
knowing from our field axioms that operation \((\cdot)\) is \emph{left} distributive. Let us make an observation
that by definition of field, \( (x + y) \in \mathbb{R}\). This means that \((x+y)z = z(x+y)\) because of \emph{Commutativity of multiplication}.
And then using \emph{Left distributivity of addition} we arrive to:
\begin{equation*}
    (x+y)z = z(x+y) = zx + zy
\end{equation*}

\section*{1-3}

Let \(x \in \mathbb{R}\) and \(x', \  \overline{x}\) be multiplicative inverses of \(x\). 
We will show that multiplicative inverses are unique:

\begin{equation*}
    x' = 1 \cdot x' \ \ \ \mbox{\emph{A neutral element 1 for multiplication}}
\end{equation*}
\begin{equation*}
    = (x \overline{x}) \cdot x' \ \ \ \mbox{\emph{Multiplicative inverse}}
\end{equation*}
\begin{equation*}
    = (\overline{x} x) \cdot x' \ \ \ \mbox{\emph{Commutativity of multiplication}}
\end{equation*}
\begin{equation*}
    = \overline{x} \cdot (xx') \ \ \ \mbox{\emph{Associativity of multiplication}}
\end{equation*}
\begin{equation*}
    = \overline{x} \cdot 1 \ \ \ \mbox{\emph{Multiplicative inverse}}
\end{equation*}
\begin{equation*}
    = \overline{x} \ \ \ \mbox{\emph{A neutral element 1 for multiplication}}
\end{equation*}

We have shown that \(\overline{x} = x'\) which means that if multiplicative inverse exists, it is unique.

\section*{1-4}

If \(0\) had a multiplicative inverse then we would have \(0 \cdot 0^{-1} = 1\). But that 
would mean that we could multiply an element of reals field \(0^{-1}\) by \(0\). According to
the field axioms we should have then had \(0^{-1} \cdot 0 = 0\) but this would contradict our
assumption that \(0 \cdot 0^{-1} = 1\).

\section*{1-5}

We know that \((xy)\) is invertible because we start with the assumption that \(x \neq 0\) and \(y \neq 0\).
Which means that \(x \cdot y \neq 0\). Hence, \(xy\) is some real number, and from axioms we know that
every real number other than zero has an inverse.  

Let us first use \emph{Existence of multiplicative inverses} to settle the ground for our proof:
\begin{equation*}
	(xy)^{-1}(xy) = 1
\end{equation*}
Now we will use the chain of properties of real numbers:
\begin{equation*}
	(xy)^{-1}(xy) = 1 \cdot 1 \ \ \ \emph{Existence of the multiplicative identity}
\end{equation*}
\begin{equation*}
	(xy)^{-1}(xy) = (x^{-1}x) \cdot (yy^{-1}) \ \ \ \emph{Existence of multiplicative inverses}
\end{equation*}
\begin{equation*}
	(xy)^{-1}(xy) = (x^{-1}y^{-1}) \cdot (xy) \ \ \ \emph{Associative and commutative laws for multiplication (lots of)} 
\end{equation*}
\begin{equation*}
	(xy)^{-1} = (x^{-1}y^{-1}) \ \ \ \emph{Existence of multiplicative inverses and associative law for multiplication} 
\end{equation*}
and our proof is done.

\end{document}
