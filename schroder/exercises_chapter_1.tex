\documentclass{article}
\usepackage{amsmath}
\usepackage{amssymb}
\usepackage{amsthm}


\newtheorem*{conjecture}{Conjecture}
\newtheorem*{theorem}{Theorem}
\newtheorem{lemma}{Lemma}
\newtheorem*{lemma*}{Lemma}
\newtheorem*{proposition}{Proposition}


\makeatletter
\newcommand*{\rom}[1]{\expandafter\@slowromancap\romannumeral #1@}
\makeatother


\begin{document}

\section*{1-1}

We will use field axioms in \(\mathbb{R}\)  to show that \((-1)(-1) = 1\). First let us observe that:

\begin{equation*}
    0 = (-1) \cdot 0  \ \ \ \mbox{\emph{A neutral element 1 for multiplication}}
\end{equation*}
\begin{equation*}
    = (-1) \cdot (1 + (-1)) \ \ \ \mbox{\emph{Additive inverse}}
\end{equation*}
\begin{equation*}
    = (-1) \cdot 1 + (-1) \cdot (-1) \ \ \ \mbox{\emph{Left distributivity of addition}}
\end{equation*}
\begin{equation*}
    = 1 \cdot (-1) + (-1) \cdot (-1) \ \ \ \mbox{\emph{Commutativity of multiplication}}
\end{equation*}
\begin{equation*}
    = (-1) + (-1) \cdot (-1) \ \ \ \mbox{\emph{A neutral element 1 for multiplication}}
\end{equation*}

We end up with the following equation:
\begin{equation*}
    (-1) + (-1) \cdot (-1) = 0
\end{equation*}
Adding \(1\) on both sides and using \emph{Additive inverse} and \emph{Associativity of addition}:
\begin{equation*}
    (1 + (-1)) + (-1)(-1) = 0 + (-1)(-1) = (-1)(-1) = 1
\end{equation*}

\section*{1-2}

Let \(x, y, z \in \mathbb{R}\). We need to prove that:
\begin{equation*}
    (x + y)z = xz = yz
\end{equation*}
knowing from our field axioms that operation \((\cdot)\) is \emph{left} distributive. Let us make an observation
that by definition of field, \( (x + y) \in \mathbb{R}\). This means that \((x+y)z = z(x+y)\) because of \emph{Commutativity of multiplication}.
And then using \emph{Left distributivity of addition} we arrive to:
\begin{equation*}
    (x+y)z = z(x+y) = zx + zy
\end{equation*}

\section*{1-3}

Let \(x \in \mathbb{R}\) and \(x', \  \overline{x}\) be multiplicative inverses of \(x\). 
We will show that multiplicative inverses are unique:

\begin{equation*}
    x' = 1 \cdot x' \ \ \ \mbox{\emph{A neutral element 1 for multiplication}}
\end{equation*}
\begin{equation*}
    = (x \overline{x}) \cdot x' \ \ \ \mbox{\emph{Multiplicative inverse}}
\end{equation*}
\begin{equation*}
    = (\overline{x} x) \cdot x' \ \ \ \mbox{\emph{Commutativity of multiplication}}
\end{equation*}
\begin{equation*}
    = \overline{x} \cdot (xx') \ \ \ \mbox{\emph{Associativity of multiplication}}
\end{equation*}
\begin{equation*}
    = \overline{x} \cdot 1 \ \ \ \mbox{\emph{Multiplicative inverse}}
\end{equation*}
\begin{equation*}
    = \overline{x} \ \ \ \mbox{\emph{A neutral element 1 for multiplication}}
\end{equation*}

We have shown that \(\overline{x} = x'\) which means that if multiplicative inverse exists, it is unique.

\section*{1-4}

If \(0\) had a multiplicative inverse then we would have \(0 \cdot 0^{-1} = 1\). But that 
would mean that we could multiply an element of reals field \(0^{-1}\) by \(0\). According to
the field axioms we should have then had \(0^{-1} \cdot 0 = 0\) but this would contradict our
assumption that \(0 \cdot 0^{-1} = 1\).

\section*{1-5}

We know that \((xy)\) is invertible because we start with the assumption that \(x \neq 0\) and \(y \neq 0\).
Which means that \(x \cdot y \neq 0\). Hence, \(xy\) is some real number, and from axioms we know that
every real number other than zero has an inverse.  

Let us first use \emph{Existence of multiplicative inverses} to settle the ground for our proof:
\begin{equation*}
	(xy)^{-1}(xy) = 1
\end{equation*}
Now we will use the chain of properties of real numbers:
\begin{equation*}
	(xy)^{-1}(xy) = 1 \cdot 1 \ \ \ \emph{Existence of the multiplicative identity}
\end{equation*}
\begin{equation*}
	(xy)^{-1}(xy) = (x^{-1}x) \cdot (yy^{-1}) \ \ \ \emph{Existence of multiplicative inverses}
\end{equation*}
\begin{equation*}
	(xy)^{-1}(xy) = (x^{-1}y^{-1}) \cdot (xy) \ \ \ \emph{Associative and commutative laws for multiplication (lots of)} 
\end{equation*}
\begin{equation*}
	(xy)^{-1} = (x^{-1}y^{-1}) \ \ \ \emph{Existence of multiplicative inverses and associative law for multiplication} 
\end{equation*}
and our proof is done (see also Spivak Chapter 1 3.(iii)).


\section*{1-6}

\subsection*{(a)}
We will prove the following formula:
\begin{equation*}
    (a + b)^2 = a^2 + 2ab + b^2
\end{equation*}
Let us justify each step using field axioms of \(\mathbb{R}\):
\begin{equation*}
    (a + b)^2 = (a + b)(a + b) \ \ \ \mbox{\emph{Definition of exponentiation}}
\end{equation*}
\begin{equation*}
    = \Big(a \cdot (a + b)\Big) + \Big(b \cdot (a + b)\Big) \ \ \ \mbox{\emph{Distributivity over addition}}
\end{equation*}
\begin{equation*}
    = \Big( (a \cdot a) + (a \cdot b)\Big) + \Big( (b \cdot a) + (b \cdot b)\Big) \ \ \ \mbox{\emph{Distributivity over addition}} 
\end{equation*}
\begin{equation*}
    = \Big( a^2 + (a \cdot b)\Big) + \Big( (b \cdot a) + b^{2}\Big) \ \ \ \mbox{\emph{Definition of exponentiation}} 
\end{equation*}
\begin{equation*}
    = \Big( a^2 + (a \cdot b)\Big) + \Big( (a \cdot b) + b^{2}\Big) \ \ \ \mbox{\emph{Commutativity of multiplication}} 
\end{equation*}
\begin{equation*}
    = \Bigg(\Big(a^2 + (a \cdot b)\Big) + (a \cdot b)\Bigg)  + \Big(b^{2}\Big) \ \ \ \mbox{\emph{Associativity of addition}} 
\end{equation*}
\begin{equation*}
    = \Bigg(\Big(a^2 + 2a b)\Big)\Bigg)  + \Big(b^{2}\Big) \ \ \ \mbox{\emph{Associativity of addition}} 
\end{equation*}
\begin{equation*}
    = a^2 + 2ab + b^2
\end{equation*}

\subsection*{(b)}

We will prove the following formula:
\begin{equation*}
    (a - b)^2 = a^2 - 2ab + b^2
\end{equation*}
Let us justify each step using field axioms of \(\mathbb{R}\):
\begin{equation*}
    \Big(a + (-b)\Big)^2 = \Big(a + (-b)\Big) \Big(a + (-b)\Big) \ \ \ \mbox{\emph{Definition of exponentiation}}
\end{equation*}
\begin{equation*}
    = \Big(a \cdot a + a \cdot (-b)\Big) + \Big((-b) \cdot a + (-b) \cdot (-b)\Big) \ \ \ \mbox{\emph{Distributivity over addition}}
\end{equation*}
\begin{equation*}
    = \Bigg(a^2 + \Big(a \cdot (-b)\Big)\Bigg) + \Bigg(\Big( (-b) \cdot a \Big) + b^2 \Bigg) \ \ \ \mbox{\emph{Definition of exponentiation}}
\end{equation*}
\begin{equation*}
    =   \Bigg(a^2 + \Big(a \cdot (-b)\Big) + \Big( (-b) \cdot a \Big) \Bigg) + \Bigg(b^2 \Bigg) \ \ \ \mbox{\emph{Associativity of addition}}
\end{equation*}
\begin{equation*}
    = \Bigg(a^2 + 2 \Big(a \cdot (-b)\Big) \Bigg) + \Bigg(b^2 \Bigg) = a^2 -2ab + b^2
\end{equation*}
where in the last step we just present our result in a more convienient notation. With that the proof is done.


\subsection*{(c)}

We will prove the following formula:
\begin{equation*}
    (a + b)(a - b) = a^2 - b^2
\end{equation*}
\newline
\begin{equation*}
    (a + b)(a - b) = (a + b)(a +  (-b))
\end{equation*}
\begin{equation*}
    = \Big( a \cdot a + \big(a \cdot (-b)\big)\Big) + \Big( b \cdot a + \big(b \cdot (-b)\big)\Big) \ \ \ \mbox{\emph{Distributivity over addition}}
\end{equation*}
\begin{equation*}
    = \Big( a^2 + \big(a \cdot (-b)\big)\Big) + \Big( b \cdot a + (-b^2) \Big) \ \ \ \mbox{\emph{Definition of exponentiation}}
\end{equation*}
\begin{equation*}
    = \Big( a^2 + \big(a \cdot (-b)\big) + b \cdot a \Big) + (-b^2)  \ \ \ \mbox{\emph{Associativity of addition}}
\end{equation*}
\begin{equation*}
    = \Big( a^2 + 0 \Big) + (-b^2)  \ \ \ \mbox{\emph{Existence of additive inverses}}
\end{equation*}
\begin{equation*}
    = a^2  + (-b^2) = a^2 - b^2  \ \ \ \mbox{\emph{Neutral element for addition}}
\end{equation*}
As shown by field axioms it follows that \((a + b)(a - b) = a^2 - b^2 \)

\section*{1-10}
\subsection*{(a)}
\subsubsection*{\(x\) is positive}

Let \(x\) be positive. Then \(x \in \mathbb{R^{+}}\). We want to show that:

\begin{equation*}
    x > 0 \iff 0 < x \iff (x - 0) \in \mathbb{R^{+}}
\end{equation*}

Observe that by the field axioms of reals:

\begin{equation*}
    x - 0 = x + (-0) = x + ((-1) \cdot 0) = x + 0 = x 
\end{equation*}

From our initial assumption we know that \(x\) is positive, hence \(x - 0\) is also positive and
that concludes the proof in one direction. For the other direction, assume now that \(x > 0\). It means then that
\((x - 0) \in \mathbb{R^{+}}\). Again, from the observation above, this is just an equivalent way of saying that
\(x \in \mathbb{R^{+}}\), so \(x\) is positive.

\subsubsection*{\(x\) is negative}

Let \(x\) be negative. Then \(-x \in \mathbb{R^{+}}\). Using the same observation from the positive case above:
\begin{equation*}
    (0 - x) = 0 + (-x) = -x \in \mathbb{R^{+}}
\end{equation*}
And so, if \((0 - x) = -x\) and \(-x \in \mathbb{R^{+}}\) it follows that \(x < 0\). For the opposite direction, assume that
\(x < 0\). Then \((0 - x) \in \mathbb{R^{+}}\). From here, one can easily see that \((-x) \in \mathbb{R^{+}}\)

\subsection*{(b)}

Let \(x, y, z \in \mathbb{R^{+}}\) and let \(x \leq y\). For the first case, assume that \(x = y\). Then from the definition it is clear
to see that \(x \leq y\). Assume now that \(x < y\). Then \((y - x) \) is a positive real number.  We need to show that \(x + z \leq y + z\). In other words, we need to show that
\((y + z) - (x + z)\) is a positive real number. From the field axioms of reals we see that:
\begin{equation*}
    (y + z) - (x + z) = (y + z) + (-(x + z)) = (y + z) + (-x - z) = y + 0 - x = y - x
\end{equation*} 
But we know that this is a positive real number from \(x < y\). Thus \( \Big((y + z) - (x + z)\Big) \in \mathbb{R^{+}}\) and \(x + z \leq y + z\).

\section*{1-11}
\subsection*{(a)}

Let \(y < 0\). First, assume that \(x = 0\). Then we have the following:
\begin{equation*}
    |x||y| = x \cdot |y| = 0 \cdot |x| = 0 = |0| = |0 \cdot y| = |xy|
\end{equation*}
Nowm assume that \(x > 0\). Observe that \(xy\) is a negative number. So we have that \(|xy| = -(xy)\). We show the following:
\begin{equation*}
    |x||y| = x \cdot |y| = x \cdot (-y) = -(xy) = |xy|
\end{equation*}

\subsection*{(b)}

Let \(x \in \mathbb{R}\). Assume first that \(x \geq 0\). Then we have:
\begin{equation*}
    |x| = x \geq 0 \geq -x
\end{equation*}
Assume now that \(x < 0\). Then clearly:
\begin{equation*}
    |x| = -x  \geq -x 
\end{equation*}

\subsection*{(c)}
\paragraph*{TO DOOOOOOOOOOOOOOOOOOOOOOOOOOOOOOOOOOOOOOOOOOOOOOOOOOOOOOOOoo}

\section*{1-12}
Let \(a, b \in \ I \cap J\) and \(a < b\). Let \(x \in \mathbb{R}\). Assume \(a < x < b\), then:
\begin{equation*}
    a < x < b \iff x \in (a, b) \iff \{x \in \mathbb{R} : a < x < b\} \iff \mbox{????} \iff x \in \{x \in \mathbb{R} : x \in I \ \ \mbox{and} \ \ x \in J\}
\end{equation*}

\section*{1-13}
Let \(a < b\) and let \(x, y \in [a, b]\). Observe that \(x, y\) obey both these relationships:
\begin{equation*}
    a \leq x \leq b \ \ \mbox{and} \ \ a \leq y \leq b
\end{equation*}

So from \(a \leq x\) and \(y \leq b\) we can conclude that \(a -x + y - b \leq 0\) and this implies \(y - x \leq b - a\).
Similiarly, from  \(x \leq b\) and \(a \leq y\) we have \(x - y \leq b - a\). Assume now without loss of generality that
\(x \geq y\). Then using the inequalities we showed before we have:
\begin{equation*}
    |x - y| = x - y \leq b - a
\end{equation*}
For the second case, assume \(x < y\):
\begin{equation*}
    |x - y| = -(x - y) = y - x \leq b - a
\end{equation*}

\section*{1-15}
Let \(A \subseteq \mathbb{R}\) be bounded below and let \(s, t, \in \mathbb{R}\) both be infima of \(A\). Then for all lower bounds \(l\)
of \(A\) we have that \(l \leq s\). Specifically, if \(t\) is \emph{an} infimum, then the former observation applies and \(t \leq s\) (because t - 
being \emph{an} infimum - is also a lower bound). A similiar argument can be made for \(t\), that is, \(t\) is greater than or equal to
every lower bound, hence \(s \leq t\). Applying antisymmetric property of an order relation \(\leq\) we conclude that
\(s = t\), so infima are unique. 

\section*{1-16}
First, we state the proposition for infima:
\begin{proposition}
    Let \(I \subset \mathbb{R}\) be a nonempty subset of \( \mathbb{R} \) that is bounded below and let \(i =  \inf(I)\).
    Then for every \(\epsilon > 0\) there is an \(x \in I\) so that:
    \begin{equation*}
        x - i < \epsilon
    \end{equation*}
\end{proposition}

Assume for contradiction that there is an \(\epsilon > 0\) for every \(x \in I\) so that \(x - i \geq \epsilon\). Then from the inequality
we have for every \(x \in I\), \(x \geq i + \epsilon\). Observe that from the former, we can conclude that \(i + \epsilon\) is a lower
bound. We know that \(i = \inf(I)\) so \(i \geq i + \epsilon\). But both \(i = i + \epsilon\) and \(i > i + \epsilon\) are false (because
\(\epsilon\) cannot be equal to zero nor it can also be less than zero by our assumption, we assumed that at least one such \(\epsilon\) exists.).
\newline
\paragraph*{Is the proof significantly different from that of Proposition 1.21?}
We also showed that the proposition is true by contradiction. This time we had to consider a set bounded below and try to conjecture
that no matter how small a number we fix, we are always guaranteed that there will be at least one element of our set that 
is strictly less than the infimum enlarged by our fixed tolerance. We could also see that \(i + \epsilon > i\) 
(because epsilon is greater than zero) and this contradicts the infimum definition (because every
lower bound must be less than or equal to the infimum.)
\pagebreak
\section*{1-17}

\begin{lemma}
    Let \(x, y \in \mathbb{R}\), \(\epsilon > 0\) and \(|x - y| < \epsilon\). Then \(y - \epsilon < x < y + \epsilon\)
\end{lemma}
\begin{proof}
    For the first case asume that \((x - y) \geq 0\). From the definition of absolute value we have:
    \begin{equation*}
        |x - y| = x - y < \epsilon \implies x < \epsilon + y
    \end{equation*}
    We know that \(\epsilon > 0\). This means that \(- \epsilon < 0\). From our assumption we see that \(x - y \geq 0\).
    Using transitivity we conclude that \(x - y > \epsilon\) so \(- \epsilon + y < x\) and:
    \begin{equation*}
        y - \epsilon < x < y + \epsilon
    \end{equation*}
    For the second case, assume that \((x - y) < 0\). Again, from the definition of the absolute value we have:
    \begin{equation*}
        |x - y| = -(x - y) = y - x < \epsilon \implies y - \epsilon < x
    \end{equation*}
    By a similiar argument used in the first case, \(\epsilon > 0\) so using transitivity we have \(y - x > - \epsilon\)
    and finally \(y + \epsilon > x\) so:
    \begin{equation*}
        y - \epsilon < x < y + \epsilon
    \end{equation*}
\end{proof}
\begin{lemma}
    Let \(a, b, c \in \mathbb{R}\). Let \(a \geq b\) and \(b > c\). It follows then that \(a > c\).
    \begin{proof}
        If \(a = b\) we trivially have that \(a = b > c\). Assume \(b > c\) and \(a > b\). We need to prove that \(a - c \in \mathbb{R^+}\).
        From our assumptions we know that \(b - c\) and \(a - b\) are positive real numbers. Adding up both yields:
        \begin{equation*}
            (b - c) + (a - b) = -c + a = a - c
        \end{equation*}
        By axioms of the positive real numbers we are guaranteed that \(a - c \in \mathbb{R+}\) because adding two positive
        real numbers is also a positive real number. 
    \end{proof}
\end{lemma}
Let \(S \subseteq R\) be bounded above. Assume first that \(s \in \mathbb{R}\) is the supremum of S. Then \(s\) is an
upper bound of \(S\) by definition of the supremum. Fix \(\epsilon > 0\). We need to show that there is an \(x \in S\)
so that \(|s - x| < \epsilon\). By Proposition 1.21 we know that for every \(\epsilon > 0\) we are guaranteed that
there is an element \(x \in S\) so that \(s - x < \epsilon\) and what follows, \(s < x + \epsilon\). Now what remains is
to show that \(x - \epsilon < x\). Observe that \(x > x - \epsilon\) is true because of \(\epsilon > 0\). By the definition
of the supremum, we have \(s \geq x \) for every \(x \in S\). Applying \textbf{Lemma 2} and \textbf{Lemma 1} we conclude that:
\begin{equation*}
    x + \epsilon > s > x - \epsilon
\end{equation*}
For the other side, assume that \(s\) is an upper bound. Assume for contradiction that \(s\) is not the supremum.
Let every upper bound \(u\) of \(S\) obey \(u < s\). Observe that \(s - y > 0\). Fix \(\epsilon = \frac{(s - u)}{2}\). Then there exists
an element \(x \in S\) so that we have \(|s-x| < \epsilon\). Using \textbf{Lemma 1} we obtain \(s < x + \epsilon = x + \frac{(s - u)}{2}\).
We make the following transformations to our inequality:
\begin{equation*}
    s < x + \frac{(s - u)}{2} \ \ \mbox{for some \(x \in S\)}
\end{equation*}
\begin{equation*}
    2s - 2x < s - u \ \ \mbox{for some \(x \in S\)}
\end{equation*}
\begin{equation*}
    s - 2x < -u \ \ \mbox{for some \(x \in S\)}
\end{equation*}
\begin{equation*}
    s < 2x - u \ \ \mbox{for some \(x \in S\)}
\end{equation*}
We assumed that \(u < s\). From transitivity of \(<\) it follows that \(u < 2x - u\). But that would just mean:
\begin{equation*}
    u < x \ \ \mbox{for some \(x \in S\)}
\end{equation*} 
But \(u\) is an upper bound of \(S\), so we should have \(u \geq x \) for \emph{every} \(x \in S\), hence a contradiction.

\section*{1-18}
\subsection*{(a)}
Let \(A, B \subseteq \mathbb{R}\) be bounded from above. Then from Completness Axiom we know that \(A, B\) have suprema. 
We would like to show that \(\sup(A) \leq \sup(B)\).
We have that \(\sup(B) \geq x \ \ \mbox{for every \(x\) in \(B\)}\). Every element of \(A\) is also an
element of \(B\) so \(\sup(B) \geq x \ \ \mbox{for every \(x\) in \(A\)}\). The supremum is less than
or equal to any upper bound of a set so \(\sup(B) \geq \sup(A)\) because \(\sup(B)\) is also an upper bound of A and we are done.
\subsection*{(b)}
Let \(A, B \subseteq \mathbb{R}\) be bounded from below. From Proposition 1.20 every set that is nonempty and bounded below has
a greatest lower bound. We wish to show that \(\inf(A) \geq \inf(B)\). Any \(x \in A\) is greater than or equal to \(\inf(B)\) 
because any \(x\) from \(A\) is also contained in \(B\) and \(\inf(B)\) is the greatest lower bound of \(B\). This means
that \(\inf(B)\) is a lower bound of \(A\) and it must be \(\inf(B) \leq \inf(A)\) because \(\inf(A)\) is the 
greatest lower bound of \(A\).

\newpage
\section*{1-19}
Let \(A \subseteq \mathbb{R}\) be bounded above and let \(-A = \{x \in \mathbb{R}: -x \in A\}\).
From Completness Axiom, \(A\) has the lowest upper bound. Let \(u\)
be an upper bound of \(A\). For every \(x \in A\) we have:
\begin{equation*}
    u \geq \sup(A) \geq x
\end{equation*}
Multiplying by \(-1\) both sides we get:
\begin{equation*}
    -u \leq - \sup(A) \leq -x
\end{equation*}
We notice that \(-x \in -A\) (because \(-(-x)\) is in \(A\)). Let \(l = -u\). From the inequality above
every \(l\) is less than or equal to \(-x\). This means that \(-A\) is bounded below and from that we conclude
that \(-A\) also has the greatest lower bound. \(- \sup(A)\) is a lower bound of \(-A\) and is also greater
than or equal to any lower bound of \(-A\) and that means that \(- \sup(A) = \inf(-A)\).

\section*{1-20}
Let \(m \in \mathbb{N}\) be arbitrary and \(S_m = \{n \in \mathbb{N} : mn \in \mathbb{N}\}\). We will use the Principle of 
Induction to show that for any \(m, n \in \mathbb{N}\) we also have \(m \cdot n \in \mathbb{N}\). First we will show that
\(1\) is in \(S_m\). Observe that for \(1 \in S_m\) we must have that \(1 \cdot m \in \mathbb{N}\):
\begin{equation*}
    m \cdot n = m \cdot 1 = m \in \mathbb{N} \ \ \mbox{By our initial assumption}
\end{equation*}
Let us now prove that for each \(n \in S_m\) we have \(n + 1\) also in \(S_m\). If \(n \in S_m\) then \(mn \in \mathbb{N}\).
Now, we will check if \(n + 1\) is also in \(S_m\):
\begin{equation*}
    (n + 1) \cdot m = nm + 1 \cdot m = nm + m = mn + m \in \mathbb{N}
\end{equation*} 
where we used the fact that natural numbers are closed under addition while also using the properties of real numbers.
By the Principle of Induction, we conclude that \(S_m = \mathbb{N}\). Because \(m \in \mathbb{N}\) was arbitrary, this means
that for any \(m, n \in \mathbb{N}\) we have \(m \cdot n \in \mathbb{N}\).

\section*{1-21}
Let \(S = \{n \in \mathbb{N} : n \geq 1\}\). We will show, using the Principle of Induction, that \(S = \mathbb{N}\).
First, observe that \(1 \in S\) because \(1 \geq 1\). Now, assume that \(n \in S\). Let us show that \(n + 1\) is also in \(S\).
Indeed, applying some algebra we receive:
\begin{equation*}
    n \geq 1 \ \ (+1)
\end{equation*}
\begin{equation*}
    n + 1 \geq 1 + 1 \geq 1
\end{equation*}
\begin{equation*}
    n + 1 \geq 1 
\end{equation*}
and by the Principle of Induction \(S = \mathbb{N}\). Because \(n\) was arbitrary, for any natural number \(n\) we have that \(n \geq 1\).

\section*{1-22}
Let \(S\) be a subset of \(\mathbb{N}\) such that two conditions are met:
\begin{equation*}
    1 \in S
\end{equation*}
\begin{equation*}
    \forall n \in S \implies n + 1 \in S
\end{equation*}
Let us assume for contradiction that \(\mathbb{N } \setminus S \) is not empty. Then we can claim - using the Well ordering principle -
that \(\mathbb{N} \setminus S \) has a smallest element. Call this element \(k\). We know that \(k\) cannot be equal to \(1\) 
because \(1 \in S\). \(k - 1\) then is less than a smallest element of \(\mathbb{N} \setminus S\) so it cannot belong to
\(\mathbb{N} \setminus S\) (otherwise \(k\) would not be a smallest element but \(k-1\) would). So \(k - 1\) must be in \(S\).
But that would imply the following relation:
\begin{equation*}
    k - 1 \in S \implies k \in S
\end{equation*}
taken from the charateristic of \(S\). And thus we arrive to contradiction: 
a smallest element of \(\mathbb{N} \setminus S\) cannot be in \(S\).

\section*{1-23}
\subsection*{(a)}
This can be done by adding "without loss of generality, assume ..." because it's the symetrical case of \(n \in \mathbb{N}\)
and \(m \in \mathbb{N}\). You can also deduce it by using the commutative property of the reals.
\subsection*{(b)}
\begin{center}
    THIS IS POTENTIALLY WRONG!!! CONSULT BOLDED TEXT!!! 
\end{center}
We will show that integers are closed under subtraction by utilizing already estabilished fact that integers
are closed under addition. Assume that \(m \in \mathbb{Z}\) and \(n \in \mathbb{Z}\). We want to show that
\(m - n \in \mathbb{Z}\). By applying algebra we get:
\begin{equation*}
    m - n = m + (-n)
\end{equation*}
Now the key question here is whether, given that \(n \in \mathbb{Z}\) we also have the same for \(-n \in \mathbb{Z}\).
Fortunately that is the case. \textbf{Observe that \(-n = (-1) \cdot n\). Since \(-(-1) \in \mathbb{N}\) it follows that \(-1\) is also in
\(\mathbb{Z}\). This means that \(-n \in \mathbb{Z}\) (closure under addition of integers).} And from this we easily conclude that:
\begin{equation*}
    m - n = m + (-n) \in \mathbb{Z}
\end{equation*}
\subsection*{(c)}
We will show that integers are closed under multiplication. Let \(m, n \in \mathbb{Z}\). If either of one them is \(0\) (or both)
then there is nothing to prove. First, assume that \(m > 0\) and \(n > 0\). Then by the closure under multiplication of the
natural numbers we can say that \(mn \in \mathbb{Z}\). Now assume that \(m < 0\) and \(n < 0\). Observe that:
\begin{equation*}
    mn = (-m) \cdot (-n) \in \mathbb{N}
\end{equation*}

and since \((-m) \in \mathbb{N}\) and \((-n) \in \mathbb{N}\) it follows from the closure under multiplication of the natural numbers
that \(mn \in \mathbb{Z}\). For the last part, assume without loss of generality that \(m > 0\) and \(n < 0\). If we
manage to show that \(mn < 0\) then we will be sure that \(mn \in \mathbb{Z}\) (because then \(-(mn) \in \mathbb{N}\)).
Setting up our chain of inequalities we have:
\begin{equation*}
    m \cdot (-n) = -(mn) > 0 > - (m \cdot (-n)) = - (-(mn)) = mn
\end{equation*}
and from that we can conclude that \(mn \in \mathbb{Z}\).

\subsection*{(d)}
Let \(m, n \in \mathbb{Z}\) with \(m > n\). Assume for a contradiction that \(m - n < 1\). Then:
\begin{equation*}
    n < m < n + 1
\end{equation*}
By the closure under subtraction of the integers we should have exactly one of the following:
\begin{equation*}
    m - n = 0
\end{equation*}
\begin{equation*}
    m - n \in \mathbb{N}
\end{equation*}
\begin{equation*}
    - (m - n) \in \mathbb{N}
\end{equation*}
But \(m - n\) cannot be \(0\) because \(m > n\), similiarly \(m - n\) is not a natural number because it is less than \(1\)
and every natural number is greater than or equal to \(1\) (Proposition 1.26). What remains is the last case when
\(- (m - n) \in \mathbb{N}\) but again this implies that \(-(m - n) = n - m \in \mathbb{N}\) which cannot occur because 
we would have had \(n - m < 0\) but we assumed \(m > n\). Thus \(m - n \geq 1\) with \(m > n\).

\paragraph*{Remark} Similiarly to natural numbers, you cannot insert an integer between two integers with a difference of \(1\).

\newpage
\subsection*{(e)}
Let \(A \in \mathbb{Z}\) be a nonempty set that is bounded above. Observe that
this also means that \(A \in \mathbb{R}\) and since it is bounded above it has a supremum \(s\). 
By proposition 1.21 for suprema, there exists an element \(a \in A\) such that:
\begin{equation*}
    s - a < 1
\end{equation*} 
Now we know that \(a \in (s - 1, s]\). Let us check if \(a\) is unique. Let \(m, n\) be some integers such that \(m > n\). Then:
\begin{equation*}
    s - 1 < m \leq s
\end{equation*}
\begin{equation*}
    s - 1 < n \leq s
\end{equation*} 
next we check:
\begin{equation*}
    m - n \leq s - s = 0 < 1 \implies m - n < 1
\end{equation*}
which invalidates our results from Exercise 1.23 (d). Thus \(a\) is the only integer in \((s - 1, s]\). It cannot be greater
than a supremum, what is more, it cannot be less than other members of \(A\) not in \((s - 1, s]\).
\(a\) must be then a maximum of \(A\).

\section*{1-24}
\subsection*{(a)}
Let \(a, c \in \mathbb{Z}\) and \(b, d \in \mathbb{N}\), and let \(\frac{a}{b} \in \mathbb{Q}\), \(\frac{c}{d} \in \mathbb{Q}\).
Then \(\mathbb{Q}\) is closed under subtraction because:
\begin{equation*}
    \frac{a}{b} - \frac{c}{d} = ab^{-1} - cd^{-1} = add^{-1}b^{-1} - cbb^{-1}d^{-1} = (b^{-1}d^{-1})(ad - cb) = 
\end{equation*}
\begin{equation*}
    = (ad - cb)(b^{-1}d^{-1}) = (ad - cb)(bd)^{-1} = \frac{ad - cb}{bd}.
\end{equation*}
\subsection*{(b)}
\begin{lemma*}
    Let \(n \in \mathbb{Z} \setminus \{0\}\) and \(d \in \mathbb{N}\). 
    Then:
    \begin{equation*}
        \Big(\frac{n}{d}\Big)^{-1} = \frac{d}{n}
    \end{equation*}
    \begin{proof}
    Using little algebra:
    \begin{equation*}
        \Big(\frac{n}{d}\Big)^{-1} = \frac{1}{\frac{n}{d}} = \frac{1}{nd^{-1}}
    \end{equation*}
    \begin{equation*}
        \frac{1}{nd^{-1}}nd^{-1} = 1 = 1 \cdot 1 = dn^{-1}d^{-1}n
    \end{equation*}
    \begin{equation*}
        \frac{1}{nd^{-1}}nd^{-1} = dn^{-1}d^{-1}n = (dn^{-1})(nd^{-1}) \implies \frac{1}{nd^{-1}} = dn^{-1} = \frac{d}{n}
    \end{equation*}
    \end{proof}
\end{lemma*}
Now we proceed to prove that if \(q, r \in \mathbb{Q}\) then \(\frac{q}{r}\) in \(\mathbb{Q}\) (of course \(r \neq 0\)).
Let \(a, c \in \mathbb{Z} \setminus \{0\}\), \(b, d \in \mathbb{N}\) and \(q = \frac{a}{b}\), \(r = \frac{c}{d}\).
We start with the following:
\begin{equation*}
    \frac{q}{r} = \frac{\frac{a}{b}}{\frac{c}{d}} = \frac{a}{b}\Big(\frac{c}{d}\Big)^{-1} = 
\end{equation*}
Using the lemma above and closure under multiplication of the rationals:
\begin{equation*}
    = \frac{a}{b}\frac{d}{c} = \frac{ad}{bc}
\end{equation*}
which was to be demonstrated.

\paragraph*{Remark} How does multiplication of the denominators reflect the definition of \(\mathbb{Q}\) given by Schroder?
We multiply a natural number with an integer, we will not necessarily end up in Naturals.
\section*{1-25}
Let \(a, b \in \mathbb{R}\) such that for all \(\epsilon > 0\) we have \(a \leq b + \epsilon\). Assume for contradiction
that \(a > b\). Observe that \(a - b > 0\). Fix \(\epsilon = \frac{a - b}{2}\). We can then show that:
\begin{equation*}
    a \leq b + \epsilon = b + \frac{a - b}{2} = \frac{2b + a - b}{2} = \frac{b + a}{2}
\end{equation*}
which in turn gives
\begin{equation*}
    2a \leq b + a
\end{equation*}
\begin{equation*}
    a \leq b 
\end{equation*}
but this cannot occur because we explicitly assumed \(a > b\).
\section*{1-27}
Let \(x \in \mathbb{R}\) and \(\epsilon > 0\). Then by Theorem 1.32 we can consider a real number
\(\frac{x}{\epsilon}\) (personal note: we don't need to worry about division by zero) 
for which there is an \(n \in \mathbb{N}\) so that:
\begin{equation*}
    n > \frac{x}{\epsilon}
\end{equation*} 
simplyfing the term we end up with:
\begin{equation*}
    \epsilon > \frac{x}{n}
\end{equation*}
which was to be demonstrated.
\section*{1-28}
We will use axioms and properties of real numbers to achieve the desired result:
\begin{equation*}
    \frac{1}{3} + \frac{1}{3} + \frac{1}{3} = 
\end{equation*}
\begin{equation*}
    = \Big(\frac{1}{3} + \frac{1}{3}\Big) + \frac{1}{3} \ \ \ \mbox{\emph{Associativity of the real numbers}}
\end{equation*}
\begin{equation*}
    = \Big(\frac{2}{3}\Big) + \frac{1}{3} \ \ \ \mbox{\emph{Closure under addition of the rational numbers}}
\end{equation*}
\begin{equation*}
    = \frac{2}{3} + \frac{1}{3} = \frac{3}{3} = \ \ \ \mbox{\emph{Closure under addition of the rational numbers}}
\end{equation*}
\begin{equation*}
    =  3 \cdot \frac{1}{3} = 1 \ \ \ \mbox{\emph{Existence of multiplicative inverse}}
\end{equation*}
\section*{1-30}
\subsection*{(a)}
Let \(x\) be a real number and let \(A\) be a set with the following definition:
\begin{equation*}
    \{r \in \mathbb{Q} : r \leq x\}
\end{equation*}
We want to prove that \(x\) is a supremum of \(A\). For that to happen, we need to show that \(x\) is an upper bound
and that for every \(\epsilon > 0\) there is some element in \(A\), call it \(q\), so that \(|x - q| < \epsilon\).
We can easily see that \(x\) is an upper bound because every element of \(A\) is equal to it or lies below.
To show the second condition, let us recall that by Theorem 1.36 we are guaranteed that there exists a rational
number \(q\) such that \(x - \epsilon < q < x\). Observe that \(q\) is in \(A\). From the inequality we also see that:
\begin{equation*}
    x - \epsilon < q < x < x + \epsilon
\end{equation*}
and this is just \(|x - q| < \epsilon\) underneath. Thus it concludes our check and we can say that \(\sup(A) = x\).
\subsection*{(b) I}
Let \(n, m \in \mathbb{N}\). We will first show that the function \(f\) is additive, that is:
\begin{equation*}
    f(n + m) = f(n) \tilde{+} f(m)
\end{equation*}
First, let us reorganize the terms to use cleverly \(f\)'s definition:
\begin{equation*}
    f(n + m) = f(n + m + 1 - 1) = f\Big((n + m - 1) + 1\Big) = f(n + m - 1) \ \tilde{+} \ \tilde{1}
\end{equation*}
Now we will use induction on \(n\) to show that \(f(n + m - 1) = f(m) \ \tilde{+} \ f(n - 1)\). First, for \(n = 1\) we have:
\begin{equation*}
    f(n + m - 1) = f(1 + m - 1) \ \tilde{+} \ \tilde{0} = f(m) \ \tilde{+} \ \tilde{0} = f(m) \ \tilde{+} \ f(1 - 1)
\end{equation*}
For the induction step, let us assume that \(f(n + m - 1) = f(m) \ \tilde{+} \ f(n - 1)\). We must show that
\begin{equation*}
    f(m + (n + 1) - 1) = f(m) \ \tilde{+} \  f(n)
\end{equation*}
We show:
\begin{equation*}
    f\Big((m + (n + 1) - 1)\Big) = f\Big((m + n - 1) + 1\Big) = f(m + n - 1) \ \tilde{+} \ \tilde{1} = 
\end{equation*}
Now we use the induction step:
\begin{equation*}
    = f(m) \ \tilde{+} \ f(n - 1) \ \tilde{+} \ \tilde{1} = f(m) \ \tilde{+} \ f(n)
\end{equation*}
We have just proved that \(f(n + m - 1) = f(m) \ \tilde{+} \ f(n - 1)\). We can use this fact finally show that:
\begin{equation*}
    f(n + m) = f(n + m + 1 - 1) = f(n + m - 1) \ \tilde{+} \ \tilde{1} = f(m) \ \tilde{+} \ f(n - 1) \ \tilde{+} \ \tilde{1} =  
\end{equation*}
\begin{equation*}
    = f(m) \ \tilde{+} \ f\Big((n - 1) + 1\Big) = f(m) \ \tilde{+} \ f(n)
\end{equation*}
which was to be demonstrated.
Next, want to show that for any \(n, m \in \mathbb{N}\) we have:
\begin{equation*}
    f(mn) = f(m)f(n)
\end{equation*}
We will again use induction on \(n\) to prove the above. For \(n = 1\) we see:
\begin{equation*}
    f(mn) = f(m \cdot 1) = f(m) \ \tilde{\cdot} \ \tilde{1} = f(m)f(1)
\end{equation*}
Our induction step is \(f(mn) = f(m)f(n)\). We need to show \(f\Big(m \cdot (n + 1)\Big) = f(m)f(n+1)\). We use our
previously proven result for additivity and easily see:
\begin{equation*}
    f\Big(m \cdot (n + 1)\Big) = f(mn + m) = f(mn) \ \tilde{+} \ f(m) = f(m)f(n) \ \tilde{+} \ f(m) = 
\end{equation*}
\begin{equation*}
    = f(m) \ \tilde{\cdot} \ \Big(f(n) \ \tilde{+} \ \tilde{1}\Big) = f(m)f(n+1)
\end{equation*}
because of \(f\)'s definition. Now we are ready to show that the function assings exactly one value to each rational number.
Let \(m, m' \in \mathbb{Z}\) and \(n, n' \in \mathbb{N}\). We want to prove that:
\begin{equation*}
    f\Big(\frac{m}{n}\Big) = f\Big(\frac{m'}{n'}\Big)
\end{equation*}
To show this we can prove that for \(mn' = m'n\) we have \(f(m)f(n') = f(m')f(n)\). One can show:
\begin{equation*}
    f(m)f(n') = f(mn') = f(m'n) = f(m')f(n)
\end{equation*}
for we can use this result to see that:
\begin{equation*}
    \frac{f(m)}{f(n)} = \frac{f(m')}{f(n')}
\end{equation*}
Now we want to show that \(f\) preservers order. Let us first start with proving that \(f(n) \in \tilde{\mathbb{N}}\) for
any \(n\) in \(\mathbb{N}\). First, for \(n = 1\) by definition of \(f\) we see that \(f(1) = \tilde{1} \in \tilde{\mathbb{N}}\),
assume for the induction step that \(f(n) \in \tilde{\mathbb{N}}\) for
any \(n\) in \(\mathbb{N}\). We get:
\begin{equation*}
    f(n + 1) = f(n) \ \tilde{+} \ \tilde{1} \in \tilde{\mathbb{N}}
\end{equation*}
by the closure under addition of the natural numbers. Next, we show that for any \(x \in \mathbb{Z}\) we have
\(f(x) \geq 0\). When \(x = 0\) then \(f(x) = f(0) = f(-0) = \ \tilde{-} \ \tilde{0} = \tilde{0}\). If \(x > 0\), by the 
definition of the integers, \(x \in \mathbb{N}\) and from the previous result, \(f(x) \in \tilde{\mathbb{N}}\)
so \(f(x) > 1 > 0\) for every natural number is greater than or equal to one. To conclude our preparation for the main result,
we show that such relation is obeyed when for any \(q > 0 \in \mathbb{Q}\)
we also have \(f(q) > 0\). Let \(z \in \mathbb{Z}\) and \(n \in \mathbb{N}\) such that \(q = \frac{z}{n}\). If \(q = 0\)
then \(z = 0\) so we have:
\begin{equation*}
    f(q) = f(\frac{0}{n}) = f(0) = \ \tilde{0}
\end{equation*}
For the second case, we assume that \(q > 0\) which means that \(z \in \mathbb{N}\). This means that both \(f(z)\) and
\(f(n)\) are greater than zero. Then:
\begin{equation*}
    f(q) = f(\frac{z}{n}) > 0
\end{equation*}
Now we can finally prove that \(f\) preservers order. Let \(z, x\) be rational numbers such that \(x < z\).
Let \(p, r \in \mathbb{Z}\) and \(q, s \in \mathbb{N}\) such that 
\(z = \frac{p}{q}\) and \(x = \frac{r}{s}\). Observe by the previous relation for the positive
rational numbers that \(f(z - x) > 0\). Then the following relation holds:
\begin{equation*}
    f(z - x) = f(\frac{p}{q} - \frac{r}{s}) = f(\frac{ps - rq}{qs}) = 
\end{equation*}
\begin{equation*}
    = \frac{\tilde{p}\tilde{s}}{\tilde{q}\tilde{s}} \ \tilde{-} \ \frac{\tilde{r}\tilde{q}}{\tilde{q}\tilde{s}} = 
\end{equation*}
\begin{equation*}
    = f(\frac{ps}{qs}) - f(\frac{rq}{qs}) = f(\frac{p}{q}) - f(\frac{r}{s})  = f(z) - f(x) > 0
\end{equation*}

\subsection*{(b) II}

Let \(A = \{f(r) : r \in \mathbb{Q}, r \leq x\}\). To prove that \(f\) has been defined in a valid way, we show that
\(A\) is nonempty and has an upper bound. Clearly one can find a rational number that is less than or equal to some
real number so there exists \(f(r)\) contained in \(A\) thus making it nonempty. To show that \(A\) has an upper bound,
take any \(q \in \mathbb{Q}\) such that \(q > x\), knowing that \(f\) preservers order it follows that \(f(q) > f(x)\).
One can eventually see that:
\begin{equation*}
    f(q) > f(x) \geq f(r) \ \ \mbox{given \(r \leq x\)}
\end{equation*}
which shows that \(f(q)\) is an upper bound of \(A\) so \(\sup(A)\) exists and \(f(x)\) is well defined.
 
\subsection*{(b) III}
We want to prove that the function \(f\) is injective. Let us assume without loss of generality that \(x < y\) where
\(x, y\) are in reals. We are guaranteed that there exists \(r \in \mathbb{Q}\) such that:
\begin{equation*}
    x < r < y
\end{equation*}
Let us first show for more clarity both of the tails of the above inequality:
\begin{equation*}
    f(x) = \sup \{f(a) : a \in \mathbb{Q} \ \mbox{and} \ a \leq x\}
\end{equation*}
\begin{equation*}
    f(y) = \sup \{f(b) : b \in \mathbb{Q} \ \mbox{and} \ b \leq y\}
\end{equation*}
We can take \(a \in \mathbb{Q}\) obeying \(a \leq x\). Then \(a \leq x < r < y\) and since \(f\) preserves the order
for the rationals we will end up with \(f(a) < f(r)\). Similarly, we can take \(b \in \mathbb{Q}\) obeying \(r < b \leq y\).
Then \(r < b\) and since \(f\) preserves the order for the rationals we will end up with \(f(r) < f(b)\). Combining the inequalities:
\begin{equation*}
    f(a) < f(r) < f(b)
\end{equation*}
we see that \(f(a) \neq f(b)\) and since \(a \leq x\) and \(b \leq y\) it follows that \(f\) is injective.


\end{document}
