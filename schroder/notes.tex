\documentclass{article}
\usepackage{amsmath}
\usepackage{amssymb}
\usepackage{amsthm}


\newtheorem*{conjecture}{Conjecture}
\newtheorem*{theorem}{Theorem}
\newtheorem*{lemma}{Lemma}


\makeatletter
\newcommand*{\rom}[1]{\expandafter\@slowromancap\romannumeral #1@}
\makeatother


\begin{document}

\section{Chapter 1 - Notes}
\subsection{Field axioms}
\subsection{Order axioms}
\subsubsection{Triangular inequality}
\begin{theorem}
    For all, \(x, y \in \mathbb{R}\), we have \(|x + y| \leq |x| + |y|\)
\end{theorem}

He uses in this one two references to part 2 of Theorem 1.10. Let, \(x, y \in \mathbb{R}\). If the
inequality \(x + y \geq 0\) holds, then by part 2 of Theorem 1.10 at least one of \(x, y\) is greater than or equal to \(0\).

\subsection{Lowest Upper and Greatest Lower Bounds}
\subsubsection*{Example: A set that does not obey Completness Axiom}
Take the set of even numbers. It is not empty but it is also not bounded. If you take any real number, it won't be enough
to cover all of the evens, since you can always find an even number greater than your pick.
\subsubsection*{Notes for Proposition 1.21}
Just remember that you should aim to show (for a contradiction) that \(s - \epsilon < s\) basically represents the fact that for instance:
\begin{equation*}
    5 - 0.01 = 4.99 < 5 
\end{equation*}

Also, this little transformation with \(x\) and \(\epsilon\) is just to show something is greater than or equal to every element 
of your considered set, so upper bound defintion lurking in behind. Remember about "flipping" quantifiers when you contradict.

\paragraph*{Non strict inequality}

One thing to have in mind, quoting myself: "why didnt he use non-strict inequality (definition of GLB) though?
Did he silently assume that a reader can fill in the gaps for s = s - epsilon?"
\newline
My answer: Yes, but also notice that \(\epsilon > 0\) so you automatically have that \(s > s - \epsilon\),
because underneath you have \(0 > - \epsilon\) and this is true because you are negating a positive number.
In exercise 1-16 we used a different reasoning, relying more on GUB and infimum definition albeit still
mathematically valid. 

\subsection{Natural Numbers, Integers, and Rational Numbers}
\subsection{Recursion, Induction, Summations, and Products}
\end{document}
