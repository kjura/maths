\documentclass{article}
\usepackage{amsmath}
\usepackage{amssymb}
\usepackage{amsthm}


\newtheorem*{conjecture}{Conjecture}
\newtheorem*{theorem}{Theorem}
\newtheorem*{lemma}{Lemma}


\makeatletter
\newcommand*{\rom}[1]{\expandafter\@slowromancap\romannumeral #1@}
\makeatother


\begin{document}

\section{Chapter 1}
\subsection{Field axioms}
\subsection{Order axioms}
\subsubsection{Triangular inequality}
\begin{theorem}
    For all, \(x, y \in \mathbb{R}\), we have \(|x + y| \leq |x| + |y|\)
\end{theorem}

He uses in this one two references to part 2 of Theorem 1.10. Let, \(x, y \in \mathbb{R}\). If the
inequality \(x + y \geq 0\) holds, then by part 2 of Theorem 1.10 at least one of \(x, y\) is greater than or equal to \(0\).

\subsection{Lowest Upper and Greatest Lower Bounds}
\subsubsection*{Example: A set that does not obey Completness Axiom}
Take the set of even numbers. It is not empty but it is also not bounded. If you take any real number, it won't be enough
to cover all of the evens, since you can always find an even number greater than your pick.
\subsection{Natural Numbers, Integers, and Rational Numbers}
\subsection{Recursion, Induction, Summations, and Products}
\end{document}
