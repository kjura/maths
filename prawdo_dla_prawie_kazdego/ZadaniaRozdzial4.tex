\documentclass{article}
\usepackage{amsmath}
\usepackage{amssymb}
\usepackage{amsthm}
% \usepackage{bbm}
\usepackage[T1]{fontenc}


\newtheorem*{conjecture}{Conjecture}
\newtheorem*{theorem}{Theorem}
\newtheorem*{definition}{Definicja}


\makeatletter
\newcommand*{\rom}[1]{\expandafter\@slowromancap\romannumeral #1@}
\makeatother


\begin{document}

\section*{4.1.1}
\begin{center}
    Rzucamy kostką. Zdarzeniu "wypadło \(k\) oczek" przypisujemy liczbę \(3^k\). Wyznaczyć rozkład powstałej w ten sposób
    zmiennej losowej
\end{center}

Przywołajmy definicję rozkładu prawdopodobieństwa:
\begin{definition}
Rozkładem prawdopodobieństwa zmiennej losowej \(X\) nazywamy prawdopodobieństwo \(\mu_{X}\) zależnością:
\begin{equation*}
        \mu_{X}(B) = P(X^{-1}(B)), \ \ B \in \mathcal{B}(\mathbb{R})
\end{equation*}
\end{definition}
\noindent
Przystępujemy do wyznaczenia rozkładu \(X\). Niech: 
\begin{equation*}
    \Omega = \{1, 2, 3, 4, 5, 6\}.
\end{equation*}
Każdemu rzutowi o liczbie oczek \(k \in \Omega\) przypisujemy odpowiednio liczbę \(3^k\). Wobec tego dla
zdarzenia \(A \in \mathcal{B}(\mathbb{R})\) mamy:
\begin{equation*}
    \mu_{X}(A) = \mu_{X}(A \cap \{3^1, \dots, 3^6\}) = \mu_{X}\Big((A \cap \{3^1\}) \cup \dots \cup (A \cap \{3^6\})\Big) =
\end{equation*}
\begin{equation*}
    \mu_{X}\Biggr(\bigcup_{k=1}^{6}(A \cap \{3^k\})\Biggr) = \sum_{k=1}^{6} \mu_{X}(A \cap \{3^k\}) = 
\end{equation*}
\begin{equation*}
    = \sum_{k=1}^{6} \mathbf{1}_{A}(3^k) \mu_{X}(\{3^k\}) = \sum_{k=1}^{6} \mathbf{1}_{A}(3^k) P(\{k\}) = 
\end{equation*}
\begin{equation*}
    = \sum_{k=1}^{6} \mathbf{1}_{A}(3^k) \frac{1}{6}
\end{equation*}

\section*{4.1.3}

\begin{center}
    \emph{Z talii 52 kart ciągniemy 6. Każdemu losowaniu przypisujemy liczbę pików. Znaleźć rozkład określonej w ten sposób
zmiennej losowej.}
\end{center}

Niech \(X\) będzie zmienną losową która zlicza liczbę wylosowanych pików w ciągach sześcioelementowych.
\par Zbiorem wartości zmiennej losowej \(X\) są \(\{0, 1, 2, 3, 4, 5, 6\}\). Zakładamy, że kolejność losów nie ma znaczenia.
Wtedy \(\# \Omega = {\binom{52}{6}}\)
Prawdopodobieństwo wylosowania \(k \in \{0, 1, 2, 3, 4, 5, 6\}\) pików z \(52\) kart możemy wtedy zapisać jako:

\begin{equation*}
    P(X = k) = \frac{\binom{13}{k} \cdot \binom{39}{6 - k}}{\binom{52}{6}}
\end{equation*}
gdzie interpretujemy zliczanie w następujący sposób:
\begin{equation*}
    \underbrace{\binom{13}{k}}_\text{Wybór \(k\) pików} \underbrace{\binom{39}{6 - k}}_\text{Wybór pozostałych kolorów}
\end{equation*}

Obliczamy prawdopodobieństwa dla wartości realizacji zmiennej losowej:

\begin{equation*}
    P(X = 0) = \frac{\binom{13}{0} \cdot \binom{39}{6}}{\binom{52}{6}} = \frac{2109}{13160} \approx 0.160258
\end{equation*}
\begin{equation*}
    P(X = 1) = \frac{\binom{13}{1} \cdot \binom{39}{5}}{\binom{52}{6}} = \frac{82251}{223720} \approx 0.367652
\end{equation*}
\begin{equation*}
    P(X = 2) = \frac{\binom{13}{2} \cdot \binom{39}{4}}{\binom{52}{6}} = \frac{246753}{783020} \approx 0.315130
\end{equation*}
\begin{equation*}
    P(X = 3) = \frac{\binom{13}{3} \cdot \binom{39}{3}}{\binom{52}{6}} = \frac{100529}{783020} \approx 0.128386
\end{equation*}
\begin{equation*}
    P(X = 4) = \frac{\binom{13}{4} \cdot \binom{39}{2}}{\binom{52}{6}} = \frac{8151}{313208} \approx 0.026024
\end{equation*}
\begin{equation*}
    P(X = 5) = \frac{\binom{13}{5} \cdot \binom{39}{1}}{\binom{52}{6}} = \frac{3861}{1566040} \approx 0.002465
\end{equation*}
\begin{equation*}
    P(X = 6) = \frac{\binom{13}{6} \cdot \binom{39}{0}}{\binom{52}{6}} = \frac{33}{391510} \approx 0.000084
\end{equation*}

Zatem rozkład zmiennej losowej \(X\) na zbiorze \(S\) wynosi:
\begin{align*} 
    S = \Big\{(0, \frac{2109}{13160}), (1, \frac{82251}{223720}), (2, \frac{246753}{783020}), (3, \frac{100529}{783020}), \\
             (4, \frac{8151}{313208}), (5, \frac{3861}{1566040}), (6, \frac{33}{391510})\Big\}
\end{align*}

przy czym:
\begin{equation*}
    \mu_{X}(S) = 1
\end{equation*}

\section*{4.1.4}

\begin{center}
    Dla jakich \(a\) funkcja \(f(x) = ax^{2}\mathbf{1}_{[0,2]}(x)\) jest gęstością?
\end{center}

Sprawdzamy dwa warunki gęstości tj.:
\begin{equation*}
    f \geq 0
\end{equation*}
oraz
\begin{equation*}
    \int_{-\infty}^{\infty} f(x)dx = 1
\end{equation*}
Łatwo zauważamy, że zbiór wartości funkcji \(ax^2\) na przedziale \([0,2]\) to \([0, 4a]\).
Wobec tego warunek pierwszy będziemy spełniony jeśli \(a \geq 0\). Sprawdzamy teraz warunek drugi:
\begin{equation*}
    1 = \int_{-\infty}^{\infty} f(x)dx = \int_{-\infty}^{0} f(x)dx + \int_{0}^{2} f(x)dx + \int_{2}^{\infty} f(x)dx
\end{equation*}
Pierwsza i trzecia całka zeruje się ze względu na wartości funkcji charakterystycznej zbioru:
\begin{equation*}
    1 = \int_{-\infty}^{\infty} f(x)dx = \int_{0}^{2} ax^{2} \cdot 1 dx = a \cdot \int_{0}^{2} x^{2} dx = 
\end{equation*}
\begin{equation*}
    = a \cdot \frac{1}{3}x^{3} \Bigg\rvert_{0}^{2} = a \cdot \frac{1}{3}\Bigg(2^{3} - 0^{3}\Bigg) = a \cdot \frac{1}{3} \cdot 8 =
\end{equation*}
\begin{equation*}
    = \frac{8a}{3}
\end{equation*}

Układając końcowe równanie:
\begin{equation*}
    1 = \frac{8a}{3}
\end{equation*}
udaje nam się policzyć \(a\) jako:
\begin{equation*}
    a = \frac{3}{8}
\end{equation*}
i jest to jedyne \(a\) dla którego funkcja \(f\) jest gęstością.

\section*{4.1.5}
\begin{center}
    \textbf{Dwustronny rozkład wykładniczy (Laplace'a)}. Niech \(X\) ma gęstość \(f(x) = Ce^{-|x|}\).
    Wyznaczyć \(C\) i \(P(X > -1)\).
\end{center}
Wyznaczmy najpierw stałą \(C\). Z warunku na gęstość musimy mieć \(f \geq 0\), więc \(C \geq 0\). Z warunku unormowania do jedności mamy:
\begin{equation*}
    1 = \int_{-\infty}^{\infty} f(x)dx = \int_{-\infty}^{\infty} Ce^{-|x|} dx = 
\end{equation*}
\begin{equation*}
    = C \Big(\int_{-\infty}^{0} e^{x} dx + \int_{0}^{\infty} e^{-x} dx\Big)
\end{equation*}
po policzeniu całek otrzymujemy:
\begin{equation*}
    = C \Big(e^0 - 0 - (0 - e^{-0})\Big) = C \cdot (1 - 0 - 0 + 1) = 2C
\end{equation*}
i wyznaczamy \(C\):
\begin{equation*}
    1 = 2C \implies C = \frac{1}{2}
\end{equation*}
Gęstość \(f\) wynosi zatem:
\begin{equation*}
    f(x) = \frac{1}{2}e^{-|x|}
\end{equation*}
\par Obliczamy teraz \(P(X > -1)\)
\begin{equation*}
    P(X > -1) = 1 - P(X \leq -1) = 1 - \int_{-\infty}^{-1} \frac{1}{2}e^{-|x|} dx = 1 - \int_{-\infty}^{-1} \frac{1}{2}e^{x} dx = 
\end{equation*}
\begin{equation*}
    = 1 - \frac{1}{2} \int_{-\infty}^{-1} e^{x} dx = 1 - \frac{1}{2e} = \frac{2e - 1}{2e} \approx 0.8160602
\end{equation*}

\end{document}
