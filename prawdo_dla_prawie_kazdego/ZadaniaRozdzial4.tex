\documentclass{article}
\usepackage{amsmath}
\usepackage{amssymb}
\usepackage{amsthm}
\usepackage[T1]{fontenc}


\newtheorem*{conjecture}{Conjecture}
\newtheorem*{theorem}{Theorem}


\makeatletter
\newcommand*{\rom}[1]{\expandafter\@slowromancap\romannumeral #1@}
\makeatother


\begin{document}


\section*{4.1.3}

\begin{center}
    \emph{Z talii 52 kart ciągniemy 6. Każdemu losowaniu przypisujemy liczbę pików. Znaleźć rozkład określonej w ten sposób
zmiennej losowej.}
\end{center}


Szukamy rozkładu dyskretnego (prawdopodobieństwo dla każdej karty jest rozłożone równo).\par 
Niech \(X\) będzie zmienną losową która zlicza liczbę wylosowanych pików w ciągach sześcioelementowych.
\par Zbiorem wartości zmiennej losowej \(X\) są \(\{0, 1, 2, 3, 4, 5, 6\}\). Zakładamy, że kolejność losów nie ma znaczenia.
Wtedy \(\# \Omega = {\binom{52}{6}}\)
Prawdopodobieństwo wylosowania \(k \in \{0, 1, 2, 3, 4, 5, 6\}\) pików z \(52\) kart możemy wtedy zapisać jako:

\begin{equation*}
    P(X = k) = \frac{\binom{13}{k} \cdot \binom{39}{6 - k}}{\binom{52}{6}}
\end{equation*}
gdzie interpretujemy zliczanie w następujący sposób:
\begin{equation*}
    \underbrace{\binom{13}{k}}_\text{Wybór \(k\) pików} \underbrace{\binom{39}{6 - k}}_\text{Wybór pozostałych kolorów}
\end{equation*}

Obliczamy prawdopodobieństwa dla wartości realizacji zmiennej losowej:

\begin{equation*}
    P(X = 0) = \frac{\binom{13}{0} \cdot \binom{39}{6}}{\binom{52}{6}} = \frac{2109}{13160} \approx 0.160258
\end{equation*}
\begin{equation*}
    P(X = 1) = \frac{\binom{13}{1} \cdot \binom{39}{5}}{\binom{52}{6}} = \frac{82251}{223720} \approx 0.367652
\end{equation*}
\begin{equation*}
    P(X = 2) = \frac{\binom{13}{2} \cdot \binom{39}{4}}{\binom{52}{6}} = \frac{246753}{783020} \approx 0.315130
\end{equation*}
\begin{equation*}
    P(X = 3) = \frac{\binom{13}{3} \cdot \binom{39}{3}}{\binom{52}{6}} = \frac{100529}{783020} \approx 0.128386
\end{equation*}
\begin{equation*}
    P(X = 4) = \frac{\binom{13}{4} \cdot \binom{39}{2}}{\binom{52}{6}} = \frac{8151}{313208} \approx 0.026024
\end{equation*}
\begin{equation*}
    P(X = 5) = \frac{\binom{13}{5} \cdot \binom{39}{1}}{\binom{52}{6}} = \frac{3861}{1566040} \approx 0.002465
\end{equation*}
\begin{equation*}
    P(X = 6) = \frac{\binom{13}{6} \cdot \binom{39}{0}}{\binom{52}{6}} = \frac{33}{391510} \approx 0.000084
\end{equation*}

Zatem rozkład zmiennej losowej \(X\) na zbiorze \(S\) wynosi:
\begin{align*} 
    S = \Big\{(0, \frac{2109}{13160}), (1, \frac{82251}{223720}), (2, \frac{246753}{783020}), (3, \frac{100529}{783020}), \\
             (4, \frac{8151}{313208}), (5, \frac{3861}{1566040}), (6, \frac{33}{391510})\Big\}
\end{align*}

przy czym:
\begin{equation*}
    \mu_{X}(S) = 1
\end{equation*}


\end{document}
