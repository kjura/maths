\documentclass{article}
\usepackage{amsmath}
\usepackage{amssymb}
\usepackage{amsthm}
\usepackage[T1]{fontenc}


\newtheorem*{conjecture}{Conjecture}
\newtheorem*{theorem}{Theorem}


\makeatletter
\newcommand*{\rom}[1]{\expandafter\@slowromancap\romannumeral #1@}
\makeatother


\begin{document}

\section{Zmienne losowe jednowymiarowe}

\subsection{Rozkłady funkcji zmiennych losowych}

Jeśli zmienna losowa \(X : \Omega \rightarrow \mathbb{R}\) ma rozkład dyskretny: \(X \sim \{(x_{i}, p_{1})_{i \in I}\}\),
a \(\phi: \mathbb{R} \rightarrow \mathbb{R}\) jest dowolną funkcją, to \(Y = \phi(X)\)
 jest zmienną losową i ma także rozkład dyskretny:

\begin{equation*}
\begin{split}
        P(Y=y) & = P(\phi(X) = y) = P(\{\omega: \phi(X(\omega))=y\}) = \\
        & = P \Bigg(\bigcup_{\{n: \phi(x_n) = y\}} \{\omega:X(\omega) = x_{n}\}\Bigg) = \\
        & = \sum_{\{n: \phi(x_n) = y\}} P(\{\omega: X(\omega) = x_{n}\}) = \sum_{\{n: \phi(x_n) = y\}} p_n
\end{split}
\end{equation*}

Przejście z równości 4 na 5 wynika z rozłączności zbiorów uwzględnianych przy sumacji.
Jako przykład weźmy pewna prostą transformację zmiennej losowej. Niech:

\begin{equation*}
    P(X=2) = P(X = -2) = \frac{1}{6} \ \ \mbox{oraz} \ \ P(X=0) = \frac{2}{3} 
\end{equation*}

Będziemy chcieli wyznaczyć rozkład zmiennej losowej \(Y = \phi(X) = X^2\)



\end{document}
