\documentclass{article}
\usepackage{amsmath}
\usepackage{amssymb}
\usepackage{amsthm}
% \usepackage{bbm}
\usepackage[T1]{fontenc}


\newtheorem*{conjecture}{Conjecture}
\newtheorem*{theorem}{Theorem}
\newtheorem*{definition}{Definicja}


\makeatletter
\newcommand*{\rom}[1]{\expandafter\@slowromancap\romannumeral #1@}
\makeatother


\begin{document}

\section*{4.2.1}

\begin{center}
    Niech \(X\) będzie wynikiem pojedynczego rzutu kostką. Wyznaczyć dystrybuantę \(X\) oraz
    \(P(X \in (3, 5))\), \(P(X \in (3, 5])\), \(P(X \in [3, 5])\), \(P(X \in [3, 5))\)
\end{center}

\(X\) jest zmienną losową dyskretną bowiem zasięg tej zmiennej zawiera się w skończonym 
zbiorze \(\{1, 2, 3, 4, 5, 6\}\) (tj. \(X(\omega) \in \{1, 2, 3, 4, 5, 6\}\)). Przystępujemy
do wyznaczania dystrybuanty:

\begin{equation*}
    F(t) = 
        \begin{cases}
            0,& t \in (- \infty, 1)\\
            \frac{1}{6},& t \in [1, 2)\\
            \frac{2}{6},& t \in [2, 3)\\
            \frac{3}{6},& t \in [3, 4)\\
            \frac{4}{6},& t \in [4, 5)\\
            \frac{5}{6},& t \in [5, 6)\\
            1,& t \in [6, \infty)\\ 
        \end{cases}
\end{equation*}

Mając dystrybuantę, możemy teraz policzyć prawdopodobieństwa:

\begin{equation*}
    P(X \in (3, 5)) = F(5^{-}) - F(3) = \Big(\lim_{s \to 5^{-}} F(s)\Big) - F(3) = \Big(\lim_{s \to 5^{-}} \frac{4}{6}\Big) - F(3) =
\end{equation*}
\begin{equation*}
    = \frac{2}{3} - \frac{3}{6} = \frac{1}{6}
\end{equation*}
 
\begin{equation*}
    P(X \in (3, 5]) = F(5) - F(3) = \frac{5}{6} - \frac{3}{6} = \frac{2}{6} = \frac{1}{3}
\end{equation*}

\begin{equation*}
    P(X \in [3, 5]) = F(5) - F(3^{-}) = \frac{5}{6} - \Big(\lim_{s \to 3^{-}} F(s)\Big) = \frac{5}{6} - \Big(\lim_{s \to 3^{-}} \frac{2}{6}\Big) =
\end{equation*}

\begin{equation*}
    = \frac{5}{6} - \frac{2}{6} = \frac{3}{6} = \frac{1}{2}
\end{equation*}

\begin{equation*}
    P(X \in [3, 5)) = F(5^{-}) - F(3^{-}) = \Big(\lim_{s \to 5^{-}} F(s)\Big) - \Big(\lim_{s \to 3^{-}} F(s)\Big) =
\end{equation*}
\begin{equation*}
    = \Big(\lim_{s \to 5^{-}} \frac{4}{6}\Big) - \Big(\lim_{s \to 3^{-}} \frac{2}{6}\Big) = 
\end{equation*}
\begin{equation*}
    = \frac{4}{6} - \frac{2}{6} = \frac{2}{6} = \frac{1}{3}
\end{equation*}

\end{document}
