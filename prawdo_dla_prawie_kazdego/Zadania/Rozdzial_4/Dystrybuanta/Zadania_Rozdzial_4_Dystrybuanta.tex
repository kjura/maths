\documentclass{article}
\usepackage{amsmath}
\usepackage{amssymb}
\usepackage{amsthm}
% \usepackage{bbm}
\usepackage[T1]{fontenc}


\newtheorem*{conjecture}{Conjecture}
\newtheorem*{theorem}{Theorem}
\newtheorem*{definition}{Definicja}


\makeatletter
\newcommand*{\rom}[1]{\expandafter\@slowromancap\romannumeral #1@}
\makeatother


\begin{document}

\section*{4.2.1}

\begin{center}
    Niech \(X\) będzie wynikiem pojedynczego rzutu kostką. Wyznaczyć dystrybuantę \(X\) oraz
    \(P(X \in (3, 5))\), \(P(X \in (3, 5])\), \(P(X \in [3, 5])\), \(P(X \in [3, 5))\)
\end{center}

\(X\) jest zmienną losową dyskretną bowiem zasięg tej zmiennej zawiera się w skończonym 
zbiorze \(\{1, 2, 3, 4, 5, 6\}\) (tj. \(X(\omega) \in \{1, 2, 3, 4, 5, 6\}\)). Przystępujemy
do wyznaczania dystrybuanty:

\begin{equation*}
    F(t) = 
        \begin{cases}
            0,& t \in (- \infty, 1)\\
            \frac{1}{6},& t \in [1, 2)\\
            \frac{2}{6},& t \in [2, 3)\\
            \frac{3}{6},& t \in [3, 4)\\
            \frac{4}{6},& t \in [4, 5)\\
            \frac{5}{6},& t \in [5, 6)\\
            1,& t \in [6, \infty)\\ 
        \end{cases}
\end{equation*}

Mając dystrybuantę, możemy teraz policzyć prawdopodobieństwa:

\begin{equation*}
    P(X \in (3, 5)) = F(5^{-}) - F(3) = \Big(\lim_{s \to 5^{-}} F(s)\Big) - F(3) = \Big(\lim_{s \to 5^{-}} \frac{4}{6}\Big) - F(3) =
\end{equation*}
\begin{equation*}
    = \frac{2}{3} - \frac{3}{6} = \frac{1}{6}
\end{equation*}
 
\begin{equation*}
    P(X \in (3, 5]) = F(5) - F(3) = \frac{5}{6} - \frac{3}{6} = \frac{2}{6} = \frac{1}{3}
\end{equation*}

\begin{equation*}
    P(X \in [3, 5]) = F(5) - F(3^{-}) = \frac{5}{6} - \Big(\lim_{s \to 3^{-}} F(s)\Big) = \frac{5}{6} - \Big(\lim_{s \to 3^{-}} \frac{2}{6}\Big) =
\end{equation*}

\begin{equation*}
    = \frac{5}{6} - \frac{2}{6} = \frac{3}{6} = \frac{1}{2}
\end{equation*}

\begin{equation*}
    P(X \in [3, 5)) = F(5^{-}) - F(3^{-}) = \Big(\lim_{s \to 5^{-}} F(s)\Big) - \Big(\lim_{s \to 3^{-}} F(s)\Big) =
\end{equation*}
\begin{equation*}
    = \Big(\lim_{s \to 5^{-}} \frac{4}{6}\Big) - \Big(\lim_{s \to 3^{-}} \frac{2}{6}\Big) = 
\end{equation*}
\begin{equation*}
    = \frac{4}{6} - \frac{2}{6} = \frac{2}{6} = \frac{1}{3}
\end{equation*}

\section*{4.2.2}

\begin{center}
    Wiemy, że \(X\) ma rozkład wykładniczy z parametrem \(\lambda > 0\) i \(P(X < 2) = \frac{3}{4}\).
    Obliczyć \(\lambda\).
\end{center}

\begin{equation*}
    \frac{3}{4} = P(X < 2) = P(X \leq 2) = F_{X}(2) = \int_{- \infty}^{2} f(z) dz = 
\end{equation*}
\begin{equation*}
    = \int_{- \infty}^{0} f(z)dz + \int_{0}^{2} f(z)dz =
\end{equation*}
\begin{equation*}
    = 0 + \int_{0}^{2} \lambda e^{- \lambda z} dz = \lambda \int_{0}^{2} - \lambda \cdot \frac{1}{- \lambda} e^{- \lambda z} =
\end{equation*}
\begin{equation*}
    \lambda \cdot \frac{1}{- \lambda}  \int_{0}^{2} - \lambda \cdot e^{- \lambda z} = - \int_{0}^{2}  e^{t} dt = -e^{- \lambda z} \biggr\rvert_{0}^{2} =
\end{equation*}
\begin{equation*}
    = -e^{- \lambda \cdot 2} - (-e^{- \lambda \cdot 0}) = -e^{- \lambda \cdot 2} - (-e^{0}) = 
\end{equation*}
\begin{equation*}
    = -e^{- \lambda \cdot 2} + 1 = 1 - e^{- \lambda \cdot 2}
\end{equation*}

Wykorzystujemy własności funkcji logarytmicznej aby znaleźć \(\lambda\):

\begin{equation*}
    1 - e^{- \lambda \cdot 2} = \frac{3}{4}
\end{equation*}
\begin{equation*}
    \frac{1}{4} = e^{- \lambda \cdot 2}
\end{equation*}
\begin{equation*}
    4 = e^{\lambda \cdot 2}
\end{equation*}
Aplikując logarytm naturalny po obu stronach równania mamy:
\begin{equation*}
    \ln(4) = \ln(e^{\lambda \cdot 2}) = \lambda \cdot 2
\end{equation*}
\begin{equation*}
    \frac{\ln(4)}{2} = \frac{\ln(2^2)}{2} = \frac{2 \ln(2)}{2} = \ln(2) \approx 1.38629436 \approx \lambda
\end{equation*}

\section*{4.2.3}

\begin{center}
    Dystrybuanta zmiennej losowej \(X\) dana jest wzorem:
    \begin{equation*}
        F(t) = 
            \begin{cases}
                0, & \text{dla} \ \  t < 0 \\
                0,1 + t, & \text{dla} \ \   0 \leq t < 0,5\\
                0,4 + t, & \text{dla} \ \  0,5 \leq t < 0,55\\
                1, & \text{dla} \ \  t \geq 0,55\\
            \end{cases}
    \end{equation*}
    Wyznaczyć \(P(X = \frac{1}{2})\), \(P(X \in [0, \frac{1}{2}])\), \(P(X \in (0; 0,55))\)
\end{center}

Dystrybuanta nie jest ciągła, więc nie jest to rozkład ciągły, użyjemy własności dystrybuanty
by policzyć szukane prawdopodobieństwa:

\begin{equation*}
    P(X = \frac{1}{2}) = F_{X}(\frac{1}{2}) - F_{X}\Big((\frac{1}{2})^{-}\Big) = (0,4 + 0,5) - \lim_{t \to \frac{1}{2}^{-}} 0,1 + t =
\end{equation*}
\begin{equation*}
    = 0,9 - (0,1 + \frac{1}{2}) = 0,9 - 0,6 = 0,3
\end{equation*}

\begin{equation*}
    P(X \in [0, \frac{1}{2}]) = F_{X}(\frac{1}{2}) - F_{X}(0^{-}) = 0,9 - \lim_{t \to 0^{-}} 0 = 0,9 - 0 = 0,9
\end{equation*}

\begin{equation*}
    P(X \in (0; 0,55)) = F_{X}(0,55^{-}) - F_{X}(0) = (\lim_{t \to 0,55^{-}} 0,4 + t) - (0,1 + 0) = 
\end{equation*}
\begin{equation*}
    = (0,4 + 0,55) - (0,1 + 0) = 0,95 - 0,1 = 0,85
\end{equation*}

\section*{4.2.4}

\begin{center}
    Czy dystrybuanta:
    \begin{equation*}
        F_{X}(t) = 
            \begin{cases}
                0, & \text{dla} \ \  t < 0 \\
                t^2, & \text{dla} \ \   0 \leq t < 1,\\
                1, & \text{dla} \ \  1 \leq t\\
            \end{cases}
    \end{equation*}
    jest dystrybuantą zmiennej losowej o rozkładzie ciągłym?
\end{center}

Zbadajmy najpierw ciągłość dystrybuanty zmiennej losowej \(X\). Jeśli w funkcji pojawi się skok, rozkład nie będzie
ciągły. Badamy zatem ciągłość w punktach \(0\) oraz \(1\):

\begin{equation*}
    \lim_{t \to 0^{-}} F_{X}(t) = \lim_{t \to 0^{-}} 0 = 0
\end{equation*}
\begin{equation*}
    \lim_{t \to 0^{+}} F_{X}(t) = F(0) = 0^2 = 0 
\end{equation*}

\begin{equation*}
    \lim_{t \to 1^{-}} F_{X}(t) = \lim_{t \to 1^{-}} t^2 = 1^2 = 1
\end{equation*}
\begin{equation*}
    \lim_{t \to 1^{+}} F_{X}(t) = F(1) = 1
\end{equation*}

Dystrybuanta jest zatem ciągła. Będziemy badać istnienie pochodnej dystrybuanty. Mamy zatem:

% \lim_{h \to 0} \frac{F(a + h) - F(a)}{h}
\begin{equation*}
    F'(t) = \lim_{h \to 0} \frac{F(a + h) - F(a)}{h}  = \lim_{h \to 0} \frac{0 - 0}{h} = 0 \ \ \text{dla} \ \ a \in (- \infty, 0)
\end{equation*}
\begin{equation*}
    F'(t) = \lim_{h \to 0} \frac{F(a + h) - F(a)}{h}  = \lim_{h \to 0} \frac{0 - 0}{h} = \lim_{h \to 0} \frac{1 - 1}{h} = 0 \ \ \text{dla} \ \ a \in (1, \infty)
\end{equation*}
\begin{equation*}
    F'(t) = \lim_{h \to 0} \frac{F(a + h) - F(a)}{h} = \lim_{h \to 0} \frac{(a + h)^{2} - a^2}{h} = \lim_{h \to 0} \frac{2ah + h^2}{h} = 
\end{equation*}
\begin{equation*}
    = \lim_{h \to 0} 2a + h = 2a + 0 = 2a \ \ \text{dla} \ \ a \in (0, 1)
\end{equation*}

Pokażemy teraz, że pochodna istnieje w \(a = 0\):
\begin{equation*}
    F'_{+}(t) = \lim_{h \to 0^{+}} \frac{F(0 + h) - F(0)}{h} = 
\end{equation*}
\begin{equation*}
    = \lim_{h \to 0^{+}} \frac{(0 + h)^{2} - 0^2}{h} = \lim_{h \to 0^{+}} \frac{h^2}{h} = \lim_{h \to 0^{+}} h = 0
\end{equation*}
\begin{equation*}
    F_{-}'(t) = \lim_{h \to 0^{-}} \frac{F(0 + h) - F(0)}{h} =  \lim_{h \to 0^{+}} \frac{0 - 0^2}{h} = 0
\end{equation*}
ale nie istnieje w \(a = 1\) bo \(F'_{-} \neq F'_{+}\):

\begin{equation*}
    F'_{+}(t) = \lim_{h \to 0^{+}} \frac{F(1 + h) - F(1)}{h} = \lim_{h \to 0^{+}} \frac{1 - 1}{h} = 0
\end{equation*}
\begin{equation*}
    F'_{-}(t) = \lim_{h \to 0^{-}} \frac{F(1 + h) - F(1)}{h} = \lim_{h \to 0^{-}} \frac{(1 + h)^{2} - F(1)}{h} = 
\end{equation*}
\begin{equation*}
    = \lim_{h \to 0^{-}} \frac{1 + 2h + h^2 - 1}{h} = \lim_{h \to 0^{-}} 2 + h = 2
\end{equation*}

Funkcja \(F(t)\) jest zatem dystrybuantą zmiennej losowej ciągłej. Jej gęstość to:

\begin{equation*}
    g(t) = 
        \begin{cases}
            2t, & \text{dla} \ \   0 < t < 1,\\
            0, & \text{w pozostałych przypadkach} \\
        \end{cases}
\end{equation*}

\end{document}
