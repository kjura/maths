\documentclass{article}
\usepackage{amsmath}
\usepackage{amssymb}
\usepackage{amsthm}


\newtheorem*{conjecture}{Conjecture}
\newtheorem*{theorem}{Theorem}


\makeatletter
\newcommand*{\rom}[1]{\expandafter\@slowromancap\romannumeral #1@}
\makeatother


\begin{document}

\section*{1}

\subsection*{(\rom{1})}

We will use the properties of numbers stated in chapter 1:

\begin{equation*}
	ax = a
\end{equation*}
\begin{equation*}
	a^{-1}(ax) = a^{-1}a \ \ \ \mbox{\emph{Multiplying by the inverse}}	
\end{equation*}
\begin{equation*}
	(a^{-1}a)x = 1  \ \ \ \mbox{\emph{Associative law for multiplication}}
\end{equation*}
\begin{equation*}
	1 \cdot x = 1  \ \ \ \mbox{\emph{Existence of multiplicative inverses}}
\end{equation*}
\begin{equation*}
	x = 1  \ \ \ \mbox{\emph{Existence of a multiplicative identity}}
\end{equation*}

and the proof is done.


\subsection*{(\rom{2})}

\begin{equation*}
	(x-y)(x+y) =
\end{equation*}
\begin{equation*}
	=(x-y) \cdot x + (x-y) \cdot y \ \ \ \mbox{\emph{Distributive law}}
\end{equation*}
\begin{equation*}
	=x \cdot x + (-y \cdot x) + x \cdot y + (-y \cdot y) \ \ \
	\mbox{\emph{Distributive law}}
\end{equation*}
\begin{equation*}
	=x \cdot x + (-y \cdot x) + y \cdot x + (-y \cdot y)  \ \ \
	\mbox{\emph{Commutative law for multiplication}}
\end{equation*}
\begin{equation*}
	=x \cdot x +  (-y \cdot y) \ \ \
	\mbox{\emph{Existence of additive inverses}}
\end{equation*}
\begin{equation*}
	= x^2 - y^2
\end{equation*}


\subsection*{(\rom{3})}
We start by subtracting \(y^2\) from both sides of our equation: 
\begin{equation*}
	x^2 = y^2  \ \ \ / +(-y^2) 
\end{equation*}
We will use the property of  existence of additive inverse
\begin{equation*}
	x^2 - y^2 = y^2 - y^2 = 0
\end{equation*}
Now, let us use the property we proved in exercise 1.(II):
\begin{equation*}
	x^2 - y^2 = (x-y)(x+y) = 0
\end{equation*}
From the above, either \((x-y)=0\) or \((x+y) = 0\) so using simple algebra we
end up with
\begin{equation*}
	x = y
\end{equation*}
\begin{equation*}
	x = -y
\end{equation*}
and our proof is done.



\subsection*{(\rom{4})}
Let us use the power of algebra and let use expand the expression on the right
side of equation:
\begin{equation*}
	(x-y)(x^2 + xy + y^2) = x^{3} + x^{2}y + xy^{2} - yx^{2} - xy^{2} -
	y^{3}
\end{equation*}
Now we will apply the properties of real numbers:
\begin{equation*}
	x^{3} + x^{2}y + xy^{2} - x^{2}y - xy^{2} -
	y^{3} \ \ \  \mbox{\emph{Commutative law for multiplication}}
\end{equation*}
\begin{equation*}	
	=x^{3} + x^{2}y - x^{2}y + xy^{2} - xy^{2} -
	y^{3} \ \ \  \mbox{\emph{Commutative law for addition}}
\end{equation*}
\begin{equation*}	
	=x^{3} + 0 + 0 -
	y^{3} \ \ \  \mbox{\emph{Existence of additive inverses}}
\end{equation*}
\begin{equation*}	
	=x^{3} - y^{3} \ \ \  \mbox{\emph{Existence of an additive identity}}
\end{equation*}
and the last step concludes our proof.

\subsection*{(\rom{5})}
We will proceed with induction to prove that:
\begin{equation*}
	x^{n} - y^{n} = (x-y)(x^{n-1} + x^{n-2}y + \cdots + xy^{n-2} + y^{n-1}) = (x-y)\sum_{k=0}^{n}x^{n-k}y^{k}
\end{equation*}

For the base case, let \(n=1\). Then \(x^{1} - y^{1} = (x-y)\sum_{k=0}^{1}x^{1-k}y^{k} = (x-y)\Big(x^{1}y^{0} + x^{0}y^{1}\Big)\)

\subsection*{(\rom{6})}

Let us observe that:
\begin{equation*}
	x^{3} - y^{3} = (x - y)(x^{2} + xy + y^{2})
\end{equation*}

We will show that:
\begin{equation*}
	x^{3} + y^{3} = (x+y)(x^{2} - xy + y^{2})
\end{equation*}

\end{document}
