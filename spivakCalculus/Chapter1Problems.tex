\documentclass{article}
\usepackage{amsmath}
\usepackage{amssymb}
\usepackage{amsthm}


\newtheorem*{conjecture}{Conjecture}
\newtheorem*{theorem}{Theorem}


\makeatletter
\newcommand*{\rom}[1]{\expandafter\@slowromancap\romannumeral #1@}
\makeatother


\begin{document}

\section*{1}

\subsection*{(\rom{1})}

We will use the properties of numbers stated in chapter 1:

\begin{equation*}
	ax = a
\end{equation*}
\begin{equation*}
	a^{-1}(ax) = a^{-1}a \ \ \ \mbox{\emph{Multiplying by the inverse}}	
\end{equation*}
\begin{equation*}
	(a^{-1}a)x = 1  \ \ \ \mbox{\emph{Associative law for multiplication}}
\end{equation*}
\begin{equation*}
	1 \cdot x = 1  \ \ \ \mbox{\emph{Existence of multiplicative inverses}}
\end{equation*}
\begin{equation*}
	x = 1  \ \ \ \mbox{\emph{Existence of a multiplicative identity}}
\end{equation*}

and the proof is done.


\subsection*{(\rom{2})}

\begin{equation*}
	(x-y)(x+y) =
\end{equation*}
\begin{equation*}
	=(x-y) \cdot x + (x-y) \cdot y \ \ \ \mbox{\emph{Distributive law}}
\end{equation*}
\begin{equation*}
	=x \cdot x + (-y \cdot x) + x \cdot y + (-y \cdot y) \ \ \
	\mbox{\emph{Distributive law}}
\end{equation*}
\begin{equation*}
	=x \cdot x + (-y \cdot x) + y \cdot x + (-y \cdot y)  \ \ \
	\mbox{\emph{Commutative law for multiplication}}
\end{equation*}
\begin{equation*}
	=x \cdot x +  (-y \cdot y) \ \ \
	\mbox{\emph{Existence of additive inverses}}
\end{equation*}
\begin{equation*}
	= x^2 - y^2
\end{equation*}


\subsection*{(\rom{3})}
We start by subtracting \(y^2\) from both sides of our equation: 
\begin{equation*}
	x^2 = y^2  \ \ \ / +(-y^2) 
\end{equation*}
We will use the property of  existence of additive inverse
\begin{equation*}
	x^2 - y^2 = y^2 - y^2 = 0
\end{equation*}
Now, let us use the property we proved in exercise 1.(II):
\begin{equation*}
	x^2 - y^2 = (x-y)(x+y) = 0
\end{equation*}
From the above, either \((x-y)=0\) or \((x+y) = 0\) so using simple algebra we
end up with
\begin{equation*}
	x = y
\end{equation*}
\begin{equation*}
	x = -y
\end{equation*}
and our proof is done.



\subsection*{(\rom{4})}
Let us use the power of algebra and let use expand the expression on the right
side of equation:
\begin{equation*}
	(x-y)(x^2 + xy + y^2) = x^{3} + x^{2}y + xy^{2} - yx^{2} - xy^{2} -
	y^{3}
\end{equation*}
Now we will apply the properties of real numbers:
\begin{equation*}
	x^{3} + x^{2}y + xy^{2} - x^{2}y - xy^{2} -
	y^{3} \ \ \  \mbox{\emph{Commutative law for multiplication}}
\end{equation*}
\begin{equation*}	
	=x^{3} + x^{2}y - x^{2}y + xy^{2} - xy^{2} -
	y^{3} \ \ \  \mbox{\emph{Commutative law for addition}}
\end{equation*}
\begin{equation*}	
	=x^{3} + 0 + 0 -
	y^{3} \ \ \  \mbox{\emph{Existence of additive inverses}}
\end{equation*}
\begin{equation*}	
	=x^{3} - y^{3} \ \ \  \mbox{\emph{Existence of an additive identity}}
\end{equation*}
and the last step concludes our proof.

\subsection*{(\rom{5})}
We will distribute the expression over (\(x-y)\):
\begin{equation*}
	(x-y) \cdot (x^{n-1} + x^{n-2}y + \cdots + xy^{n-2} + y^{n-1}) = 
\end{equation*}
\begin{multline*}
	= x \cdot (x^{n-1} + x^{n-2}y + \cdots + xy^{n-2} + y^{n-1})\\
	+ \Big(-y \cdot (x^{n-1} + x^{n-2}y + \cdots + xy^{n-2} + y^{n-1}) \Big) = 
\end{multline*}
\begin{multline*}
	= (x \cdot x^{n-1} + x \cdot x^{n-2}y + \cdots + x \cdot y^{n-2} + x \cdot y^{n-1})\\
	+ \Big(-y \cdot x^{n-1} -y \cdot x^{n-2}y - \cdots -y \cdot xy^{n-2} -y \cdot y^{n-1}\Big) = 
\end{multline*}
\begin{multline*}
	= (x^{n} + x^{n-1}y + \cdots + x^{2} \cdot y^{n-2} + x \cdot y^{n-1})\\
	+ \Big(-y \cdot x^{n-1} - x^{n-2}y^{2} - \cdots - xy^{n-1} - y^{n}\Big)
\end{multline*}
Let us use commutative laws (addition and multiplication) to rearrange the terms, and present the equation to find a pattern:
\begin{multline*}
	= (x^{n} + x^{n-1}y + x^{n-2}y^2 + x^{n-3}y^3 + \cdots + x^{2}  y^{n-2} + x  y^{n-1})\\
	+ \Big(-y  x^{n-1} - x^{n-2}y^{2} - x^{n-3}y^{3} - \cdots - x^{2}y^{n-2} - xy^{n-1} - y^{n}\Big) = 
\end{multline*}
\begin{multline*}
	= (x^{n} + x^{n-1}y + x^{n-2}y^2 + x^{n-3}y^3 + \cdots + x^{2}  y^{n-2} + x  y^{n-1})\\
	+ \Big(-y^{n} - x^{n-1}y - x^{n-2}y^{2} - x^{n-3}y^{3} - \cdots - x^{2}y^{n-2} - xy^{n-1}\Big) 
\end{multline*}
We can see that every term (other than \(x^{n}\) and \(y^{n}\) ) has a corresponding additive inverse, meaning that every such pair
will equal to zero. We are left then with:
\begin{multline*}
	= (x^{n} + x^{n-1}y + x^{n-2}y^2 + x^{n-3}y^3 + \cdots + x^{2}  y^{n-2} + x  y^{n-1})\\
	+ \Big(-y^{n} - x^{n-1}y - x^{n-2}y^{2} - x^{n-3}y^{3} - \cdots - x^{2}y^{n-2} - xy^{n-1}\Big) = x^{n} - y^{n}
\end{multline*}
and that is what had to be proved.

\subsection*{(\rom{6})}

Let us observe that:
\begin{equation*}
	y^{3} = y \cdot y \cdot y = -\Big(-y \cdot (-y \cdot -y)\Big) = -(-y)^{3}
\end{equation*}

We will use this fact and the result from 1.(iv) to prove that \(x^3 + y^3 = (x+y)(x^2 -xy + y^2)\). We can write our formula
in an equivalent way from observation above:
\begin{equation*}
	x^3 + y^3 = x^3 + (-(-y)^{3}) = x^3 - (-y)^{3}
\end{equation*}

Now, we proceed further using \((x-y)(x^2 + xy + y^2) = x^3 - y^3\):
\begin{equation*}
	x^3 + y^3 = x^3 + (-(-y)^{3}) = x^3 - (-y)^{3} = \Big(x-(-y)\Big)(x^2 + x \cdot (-y) + (-y)^2) = (x+y)(x^{2} - xy + y^{2})
\end{equation*}
and that concludes our end result. 


\section*{2}

One may ask why does the below seem to work:

\begin{align*}
	x^{2} &=  xy\\
	x^{2} - y^{2} &=  xy - y^{2}\\
	(x+y)(x-y) &=  y(x - y)\\
	x + y &=  y\\
	2y &=  y\\
	2 &=  1
\end{align*}

Our assumption is that \(x = y\). But this means that there is an illegal operation from equation three to equation four. Namely, 
we try to use \emph{existence of multiplicative inverses} for this operation:

\begin{equation*}
	(x-y) \cdot (x-y)^{-1} = 1
\end{equation*}

but this is not legal, because from our assumption we can see that:

\begin{align*}
	x &= y \\
	x + (-y) &= y + (-y) = 0
\end{align*}
where we used \emph{existence of additive inverses}. This shows that \((x-y)\) does not have the multiplicative inverse because there is
no number \(0^{-1}\) satisfying \(0 \cdot 0^{-1} = 1\)
 
\section*{3}

\subsection*{(i)}

\begin{align*}
	\frac{a}{b} &= ab^{-1} = ab^{-1} \cdot 1 = ab^{-1} \cdot c \cdot c^{-1} = (ac)(b^{-1}c^{-1}) = (ac)(bc)^{-1} = \frac{ac}{bc}
\end{align*}

where, going from the left, we used the definition of the fraction symbol, \emph{existence of multiplicative identity},
\emph{existence of multiplicative inverses}, \emph{associative and commutative laws for multiplication} and 
the result from exercise 1.(iii) (i.e \((ab)^{-1} = a^{-1}b^{-1}\))

\subsection*{(ii)}

We will start with the right side and arrive to the term on the left side of equality:

\begin{equation*}
	\frac{ad + bc}{bd} =
\end{equation*}
\begin{equation*}
	= (ad + bc)(bd)^{-1} \ \ \ \emph{Definition of fraction symbol}
\end{equation*}
\begin{equation*}
	= \Big((ad)(bd)^{-1}\Big) \cdot \Big((bc)(bd)^{-1}\Big) \ \ \ \mbox{\emph{Distributive law}}
\end{equation*}
\begin{equation*}
	= \Big((ad)(b^{-1}d^{-1})\Big) \cdot \Big((bc)(b^{-1}d^{-1})\Big) \ \ \ \mbox{\emph{Equivalent form of inverse}}
\end{equation*}
\begin{equation*}
	= \Big((ad)(d^{-1}b^{-1})\Big) \cdot \Big((cb)(b^{-1}d^{-1})\Big) \ \ \ \mbox{\emph{Commutative law for multiplication}}
\end{equation*}
\begin{equation*}
	= \Big(a((dd^{-1})b^{-1})\Big) \cdot \Big(c((bb^{-1})d^{-1})\Big) \ \ \ \mbox{\emph{Associative law for multiplication}}
\end{equation*}
\begin{equation*}
	= ab^{-1} + cd^{-1} \ \ \ \mbox{\emph{Existence of multiplicative inverses}}
\end{equation*}
\begin{equation*}
	= \frac{a}{b} + \frac{c}{d} \ \ \ \mbox{\emph{Definition of fraction symbol}}
\end{equation*}
Quod erat demonstrandum.


\subsection*{(iii)}

We know that \((ab)\) is invertible because we start with the assumption that \(a \neq 0\) and \(b \neq 0\).
Which means that \(a \cdot b \neq 0\). Hence, \(ab\) is some real number, and from axioms we know that
every real number other than zero has an inverse.  

Let us first use \emph{Existence of multiplicative inverses} to settle the ground for our proof:
\begin{equation*}
	(ab)^{-1}(ab) = 1
\end{equation*}
Now we will use the chain of properties of real numbers:
\begin{equation*}
	(ab)^{-1}(ab) = 1 \cdot 1 \ \ \ \emph{Existence of the multiplicative identity}
\end{equation*}
\begin{equation*}
	(ab)^{-1}(ab) = (a^{-1}a) \cdot (bb^{-1}) \ \ \ \emph{Existence of multiplicative inverses}
\end{equation*}
\begin{equation*}
	(ab)^{-1}(ab) = (a^{-1}b^{-1}) \cdot (ab) \ \ \ \emph{Associative and commutative laws for multiplication (lots of)} 
\end{equation*}
\begin{equation*}
	(ab)^{-1} = (a^{-1}b^{-1}) \ \ \ \emph{Existence of multiplicative inverses and associative law for multiplication} 
\end{equation*}
and our proof is done.

\end{document}
