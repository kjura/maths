\documentclass{article}
\usepackage{amsmath}
\usepackage{amssymb}
\usepackage{amsthm}
\usepackage{physics}
\usepackage{graphicx}

\graphicspath{ {img/} }


\newtheorem*{conjecture}{Conjecture}
\newtheorem*{theorem}{Theorem}
\newtheorem{lemma}{Lemma}
\newtheorem*{lemma*}{Lemma}
\newtheorem*{proposition}{Proposition}


\makeatletter
\newcommand*{\rom}[1]{\expandafter\@slowromancap\romannumeral #1@}
\makeatother


\begin{document}

\section*{1.1}

\subsection*{(i)}
% \textbf{}
\begin{equation*}
    3 \textbf{a} + 2 \textbf{b} - 4 \textbf{c} = 
\end{equation*}
\begin{equation*}
    = 3(2 \textbf{i} - \textbf{j} - 2 \textbf{k}) + 2(3 \textbf{i} - 4 \textbf{k}) - 4(\textbf{i} - 5 \textbf{j} + 3 \textbf{k})
\end{equation*}
\begin{equation*}
    = 6\textbf{i} - 3\textbf{j} - 6\textbf{k} + 6\textbf{i} - 8\textbf{k} - 4\textbf{i} + 20\textbf{j} - 12\textbf{k}
\end{equation*}
\begin{equation*}
    = 8\textbf{i} + 17\textbf{j} -26\textbf{k}
\end{equation*}

\begin{equation*}
    |\textbf{a} - \textbf{b}|^2 = (\textbf{a} - \textbf{b}) \cdot (\textbf{a} - \textbf{b}) = 
\end{equation*}
\begin{equation*}
    = (2\textbf{i} - \textbf{j} -2\textbf{k} - 3\textbf{i} + 4\textbf{k}) \cdot (2\textbf{i} - \textbf{j} -2\textbf{k} - 3\textbf{i} + 4\textbf{k})
\end{equation*}
\begin{equation*}
    = (- \textbf{i} - \textbf{j} +2 \textbf{k} ) \cdot (- \textbf{i} - \textbf{j} +2 \textbf{k} )
\end{equation*}
\begin{equation*}
    = (-1^2) + (-1^2) + 2^2 = 1 + 1 + 4 + 6
\end{equation*}

\subsection*{(ii)}

\begin{equation*}
    |\textbf{a}| = \sqrt{\textbf{a} \cdot \textbf{a}} = \sqrt{2^2 + (-1^2) + (-2^2)} = \sqrt{9} = 3
\end{equation*}
\begin{equation*}
    |\textbf{b}| = \sqrt{\textbf{b} \cdot \textbf{b}} = \sqrt{3^2 + 0^2 + 4^2} = \sqrt{25} = 5
\end{equation*}

\begin{equation*}
    \textbf{a} \cdot \textbf{b} = 6 + 0 + 8 = 14
\end{equation*}
\begin{equation*}
    |\textbf{a}||\textbf{b}| \cos(\theta) = \textbf{a} \cdot \textbf{b}
\end{equation*}
\begin{equation*}
    15 \cos(\theta) = 14 \implies \cos(\theta) = \frac{14}{15} \implies \theta = \cos^{-1}(\frac{14}{15})
\end{equation*}


\subsection*{(iii)}

\begin{equation*}
    |\textbf{a}| = \sqrt{\textbf{a} \cdot \textbf{a}} = \sqrt{2^2 + (-1^2) + (-2^2)} = \sqrt{9} = 3
\end{equation*}
\begin{equation*}
    |\textbf{b}| = \sqrt{\textbf{b} \cdot \textbf{b}} = \sqrt{3^2 + 0^2 + 4^2} = \sqrt{25} = 5
\end{equation*}
\begin{equation*}
    \hat{\textbf{a}} = \frac{\textbf{a}}{|\textbf{a}|} = \frac{\textbf{a}}{3} = \frac{2}{3}\textbf{i} - \frac{1}{3}\textbf{j} - \frac{2}{3}\textbf{k}
\end{equation*}
\begin{equation*}
    \hat{\textbf{b}} = \frac{\textbf{b}}{|\textbf{b}|} = \frac{\textbf{b}}{5} = \frac{3}{5}\textbf{i} + 0\textbf{j} - \frac{4}{5}\textbf{k} = \frac{3}{5}\textbf{i} - \frac{4}{5}\textbf{k}
\end{equation*}
\begin{equation*}
    \textbf{c} \cdot \hat{\textbf{a}} = (\textbf{i} - 5\textbf{j} + 3\textbf{k}) \cdot (\frac{2}{3}\textbf{i} - \frac{1}{3}\textbf{j} - \frac{2}{3}\textbf{k}) = \frac{2}{3} + \frac{5}{3} - \frac{6}{3} = \frac{7}{3} - \frac{6}{3} = \frac{1}{3}
\end{equation*}
\begin{equation*}
    \textbf{c} \cdot \hat{\textbf{b}} = (\textbf{i} - 5\textbf{j} + 3\textbf{k}) \cdot (\frac{3}{5}\textbf{i} + 0\textbf{j} - \frac{4}{5}\textbf{k}) = \frac{3}{5} + 0 - \frac{12}{5} = - \frac{9}{5}
\end{equation*}

\subsection*{(iv)}

\begin{equation*}
    \textbf{a} \times \textbf{b} =  \mdet{\textbf{i} & \textbf{j} & \textbf{k}\\ 2 & -1 & -2 \\ 3 & 0 & -4} =
\end{equation*}
\begin{eqnarray}
    = (4 - 0)\textbf{i} - (-8 - (-6))\textbf{j} + (0 - (-3))\textbf{k} = 4\textbf{i} + 2\textbf{j} + 3\textbf{k}
\end{eqnarray}

\begin{equation*}
    \textbf{b} \times \textbf{c} =  \mdet{\textbf{i} & \textbf{j} & \textbf{k}\\ 3 & 0 & -4 \\ 1 & -5 & 3} =
\end{equation*}
\begin{eqnarray}
    = (0 - (20))\textbf{i} - (9 - (-4))\textbf{j} + (-15 - 0)\textbf{k} = -20\textbf{i} -13\textbf{j}  -15\textbf{k}
\end{eqnarray}

\begin{equation*}
    (\textbf{a} \times \textbf{b}) \times (\textbf{b} \times \textbf{c}) =  \mdet{\textbf{i} & \textbf{j} & \textbf{k}\\ 4 & 2 & 3 \\ -20 & -13 & -15} =
\end{equation*}
\begin{equation*}
    = (-30 - (-39))\textbf{i} - (-60 - (-60))\textbf{j} + (-52 - (-40))\textbf{k} = 9\textbf{i} - 0\textbf{j}  -12\textbf{k} = 9\textbf{i} -12\textbf{k} 
\end{equation*}

\subsection*{(v)}
From the previous problem point we know that:
\begin{equation*}
    \textbf{a} \times \textbf{b} = 4\textbf{i} + 2\textbf{j} + 3\textbf{k}
\end{equation*}
\begin{equation*}
    \textbf{b} \times \textbf{c} = -20\textbf{i} -13\textbf{j}  -15\textbf{k}
\end{equation*}
We can proceed to calculate \(\textbf{a} \cdot (\textbf{b} \times \textbf{c}) \)
\begin{equation*}
    \textbf{a} \cdot (\textbf{b} \times \textbf{c}) = (2\textbf{i} - \textbf{j} - 2\textbf{k}) \cdot (-20\textbf{i} -13\textbf{j}  -15\textbf{k}) =
\end{equation*}
\begin{equation*}
    = -40  + 13  + 30 = 3
\end{equation*}
Similarly:
\begin{equation*}
    (\textbf{a} \times \textbf{b}) \cdot \textbf{c} = (4\textbf{i} + 2\textbf{j} + 3\textbf{k}) \cdot (\textbf{i} -5\textbf{j} + 3\textbf{k}) =
\end{equation*}
\begin{equation*}
    = 4 + (-10) + 9 = 3
\end{equation*}
We have verified that \(\textbf{a} \cdot (\textbf{b} \times \textbf{c})\) and \((\textbf{a} \times \textbf{b}) \cdot \textbf{c}\)
are equal. Moreover we see that \( [ \textbf{a}, \ \textbf{b}, \ \textbf{c}] > 0\) so the set \(\{\textbf{a}, \ \textbf{b}, \ \textbf{c}\}\)
is \emph{right-handed}.

\subsection*{(vi)}
First, let us calculate \(\textbf{a} \times (\textbf{b} \times \textbf{c})\). We already know that:
\begin{equation*}
    \textbf{b} \times \textbf{c} = -20\textbf{i} -13\textbf{j}  -15\textbf{k}
\end{equation*}
Then
\begin{equation*}
    \textbf{a} \times (\textbf{b} \times \textbf{c}) = \mdet{\textbf{i} & \textbf{j} & \textbf{k}\\ 2 & -1 & -2 \\ -20 & -13 & -15} =
\end{equation*}
\begin{equation*}
    = (15 - (26))\textbf{i} - (-30 - (40))\textbf{j} + (-26 - (20))\textbf{k} = -11 \textbf{i} + 70 \textbf{j} - 46\textbf{k}
\end{equation*}
Now calculating the right side:
\begin{equation*}
    \textbf{a} \cdot \textbf{c} = (2\textbf{i} -\textbf{j} - 2\textbf{k}) \cdot (\textbf{i} -5\textbf{j} + 3\textbf{k}) = 2 + 5 - 6 = 1
\end{equation*}
\begin{equation*}
    \textbf{a} \cdot \textbf{b} = (2\textbf{i} -\textbf{j} - 2\textbf{k}) \cdot (3\textbf{i} - 4\textbf{k}) = 6 + 0 + 9 = 14
\end{equation*}
so:
\begin{equation*}
    (\textbf{a} \cdot \textbf{c})\textbf{b} = 1(\textbf{b}) = \textbf{b} = 3\textbf{i} - 4\textbf{k}
\end{equation*}
\begin{equation*}
    (\textbf{a} \cdot \textbf{b})\textbf{c} = 14(\textbf{c}) = 14\textbf{i} - 70\textbf{j} + 42\textbf{k}
\end{equation*}
And finally:
\begin{equation*}
    (\textbf{a} \cdot \textbf{c})\textbf{b} - (\textbf{a} \cdot \textbf{b})\textbf{c} = 
\end{equation*}
\begin{equation*}
    = (3\textbf{i} -4\textbf{k}) - (14 \textbf{i} - 70\textbf{j} + 42\textbf{k}) = -11 \textbf{i} + (0 - (-70))\textbf{j} + (-4 -42)\textbf{k} =
\end{equation*}
\begin{equation*}
    = -11 \textbf{i} + 70 \textbf{j} - 46\textbf{k}
\end{equation*}
and we see now that:
\begin{equation*}
    \textbf{a} \times (\textbf{b} \times \textbf{c}) = (\textbf{a} \cdot \textbf{c})\textbf{b} - (\textbf{a} \cdot \textbf{b})\textbf{c}
\end{equation*}
is valid.


\section*{1.2}

Let us introduce a helpful drawing of a cube to introduce some notation:


\begin{figure}[ht]
    \caption{Exercise 1.2 - Cube}
    \centering
    \includegraphics[scale=0.3]{Gregory-1-2-cube.png}
\end{figure}

We would like to calculate the angle between any two diagonals. Let us choose the angle formed
when \(\overrightarrow{AG}\) and \(\overrightarrow{BH}\) intersect. We are working in the standard basis
so we can quickly deduce that:
\begin{equation*}
    \overrightarrow{AG} = [1, 1, 1]
\end{equation*}
\begin{equation*}
    \overrightarrow{BH} = [-1, 1, 1]
\end{equation*}
 
Now we need to calculate the dot product between the two:
\begin{equation*}
    |\overrightarrow{AG}| |\overrightarrow{BH}| \cos(\theta) = \overrightarrow{AG} \cdot \overrightarrow{BH}
\end{equation*}
\begin{equation*}
    \overrightarrow{AG} \cdot \overrightarrow{BH} = -1 + 1 + 1 = 1
\end{equation*}
\begin{equation*}
    |\overrightarrow{AG}| = \sqrt{1 + 1 + 1} = \sqrt{3}
\end{equation*}
\begin{equation*}
    |\overrightarrow{BH}| = \sqrt{1 + 1 + 1} = \sqrt{3}
\end{equation*}

We are interested in \(\theta\) so:
\begin{equation*}
    \cos(\theta) = \frac{1}{\sqrt{3}\sqrt{3}} = \frac{1}{3}
\end{equation*}
\begin{equation*}
    \theta = \cos^{-1}(\frac{1}{3}) \approx 1.23
\end{equation*}

\section*{1.7}


Let us denote for a fact that the centre of mass \(G\) of the particles is defined to be the point of space
with position vector:
\begin{equation*}
    \textbf{R} = \frac{m_{1}\textbf{r}_{1} + m_{2}\textbf{r}_{2} + m_{3}\textbf{r}_{3} + \dots}{m_1 + m_2 + m_3 + \dots}
\end{equation*}

If we chose a different origin, call it \(O'\) then we could always bound our initial vector to
the vector written in a different coordinate space (\(O'\) origin) such that for every
\(j = 1, 2, 3, \dots\) we would have:
\begin{equation*}
    \textbf{r}_j = \textbf{r}'_{j} + \textbf{T}
\end{equation*} 
where \(\textbf{T}\) would be the transformation vector and \(\textbf{r}'_{j}\) would be
a vector \(\textbf{r}_j\) written in a different space.
We use \(\textbf{R}'\) to indicate the centre of mass written in a different coordinate space
adjusted using the transformation vector:
\begin{equation*}
    \textbf{R}' = \frac{\sum_{j}m_{j}\textbf{r}'_{j} + m_{j}\textbf{T}}{\sum_{j}m_j} = (\sum_{j}m_{j}\textbf{r}'_{j}) + \textbf{T}
\end{equation*}

Having all the formal pieces of
the underlying machinery we proceed to the solution:
\begin{equation*}
    \textbf{R} = \frac{m_{1}\textbf{r}_{1} + m_{2}\textbf{r}_{2} + m_{3}\textbf{r}_{3} + \dots}{m_1 + m_2 + m_3 + \dots} =
\end{equation*} 
\begin{equation*}
    = \frac{m_{1}(\textbf{r}'_{1} + \textbf{T}) + m_{2}(\textbf{r}'_{2} + \textbf{T}) + m_{3}(\textbf{r}'_{3} + \textbf{T}) + \dots}{{m_1 + m_2 + m_3 + \dots}} =
\end{equation*}
\begin{equation*}
    = \frac{m_{1}\textbf{r}'_{1} + m_{1}\textbf{T} + m_{2}\textbf{r}'_{2} + m_{2}\textbf{T} + m_{3}\textbf{r}'_{3} + m_{3}\textbf{T} + \dots}{{m_1 + m_2 + m_3 + \dots}} =
\end{equation*}
\begin{equation*}
    = \frac{m_{1}\textbf{r}'_{1} + m_{2}\textbf{r}'_{2} + m_{3}\textbf{r}'_{3} + \dots + \textbf{T}(m_1 + m_2 + m_3 + \dots)}{m_1 + m_2 + m_3 + \dots}
\end{equation*}
\begin{equation*}
    = m_{1}\textbf{r}'_{1} + m_{2}\textbf{r}'_{2} + m_{3}\textbf{r}'_{3} + \dots + \textbf{T} = (\sum_{j}m_{j}\textbf{r}'_{j}) + \textbf{T} = \textbf{R}'
\end{equation*}
which shows that no matter the origin, we still get to keep \(G\) at the same point of space.

% Ok, so it will change by O where O
% in our case is a constant vector of a transformation between two different spaces right?

% For example

% A = [2, 2] Origin (0, 0)

% Now I want to shift my origin +7 through x-axis but keep physics "the same"

% So A' = [9, 2] in my new origin, Origin (7, 2), the starting point

% So to communicate betwee the two we can just:

% [2, 2] = [9, 2] + [-7, 0]

% So O = [-7, 0]

% And now A and A' obey the relation

% A = A' + O

\section*{1.8}

Our goal is to show that the three perpendiculars of a triangle are concurrent. \(\overrightarrow{OC}\) must be perpendicular
to \(\overrightarrow{AB}\). Using the dot product we find the direction of \(\overrightarrow{OC}\):
\begin{equation*}
    [b, -a] \cdot [x_1, y_1] = 0 = bx_{1} + (-ay_{1}) = ba + (-ab) = 0
\end{equation*}
so the slope is \([a, b]\) and \(\overrightarrow{OC}\) goes through \([c, 0]\). We construct the equation for the \(\overrightarrow{OC}\) line:
\begin{equation*}
    \overrightarrow{OC}(t) = [a, b]t + [c, 0]
\end{equation*}
which is also:

\begin{equation*}
    \overrightarrow{OC}(t) = 
            \begin{cases}
                x = at + c,\\
                y = bt
            \end{cases}
\end{equation*}

We want to find the point of intersection, since the equation for \(AO'\) is \(x = 0\), for \(y\) to be discovered, we need
to transform our equation to \(y\) only:
\begin{equation*}
    t = \frac{x - c}{a} \implies y = b(\frac{x - c}{a}) = \frac{b}{a}(x-c)
\end{equation*}
So the point of intersection is \((0, \frac{-bc}{a})\). Now we use the same reasoning to show that the other perpendicular
has also the same point. The \(BO''\) perpendicular must also have the dot product equal to zero with \(AC\). This means that
for  \(\overrightarrow{AC} = [c, 0] - [0, a] = [c, -a]\) we must find a vector perpendicular to it:
\begin{equation*}
    [c, -a] \cdot [x_1, y_1] = 0 = cx_{1} -ay_{1}
\end{equation*}
which is just \([a, c]\). Forming again the line equation, this time for \(BO''\) we have:
\begin{equation*}
    \overrightarrow{BO''}(t) = [a, c]t + [0, \frac{-bc}{a}]
\end{equation*}
which is also:
\begin{equation*}
    \overrightarrow{BO''}(t) = 
            \begin{cases}
                x = at + 0,\\
                y = ct - \frac{bc}{a}
            \end{cases}
\end{equation*}
Removing \(t\) and unifying equations:
\begin{equation*}
    y = \frac{cx - bc}{a}
\end{equation*}
We end up with a line that intersects \(O\) because:
\begin{equation*}
    y(b) = \frac{cb - bc}{a} = 0 \implies (b, 0)
\end{equation*}
\begin{equation*}
    y(0) = \frac{0 - bc}{a} = \frac{-bc}{a} \implies (0, \frac{-bc}{a})
\end{equation*}

\section*{1.9}

\begin{equation*}
    \textbf{\(a_2\)} \times \textbf{\(a_3\)} = (\mu_{2}\nu_{3} - \nu_{2}\mu_{3})\textbf{i} + (\nu_{2}\lambda_{3} - \lambda_{2}\nu_{3})\textbf{j} + (\lambda_{2}\mu_{3} - \mu_{2}\lambda_{3})\textbf{k}
\end{equation*}
\begin{equation*}
    \textbf{\(a_1\)} \cdot (\textbf{\(a_2\)} \times \textbf{\(a_3\)}) = (\lambda_1 \textbf{i} + \mu_1\textbf{j} +  \nu_1\textbf{k}) \cdot ((\mu_{2}\nu_{3} - \nu_{2}\mu_{3})\textbf{i} + (\nu_{2}\lambda_{3} - \lambda_{2}\nu_{3})\textbf{j} + (\lambda_{2}\mu_{3} - \mu_{2}\lambda_{3})\textbf{k})
\end{equation*}
\begin{equation*}
    = (\lambda_1(\mu_{2}\nu_{3} - \nu_{2}\mu_{3})) + (\mu_1(\nu_{2}\lambda_{3} - \lambda_{2}\nu_{3})) + (\nu_1(\lambda_{2}\mu_{3} - \mu_{2}\lambda_{3})) =
\end{equation*}
\begin{equation*}
    = \lambda_{1}\mu_{2}\nu_{3} - \lambda_{1}\nu_{2}\mu_{3} + \mu_{1}\nu_{2}\lambda_{3} - \mu_{1}\lambda_{2}\nu_{3} + \nu_{1}\lambda_{2}\mu_{3} - \nu_{1}\mu_{2}\lambda_{3}
\end{equation*}
Using the Sarrus rule:
\begin{equation*}
    \mdet{\lambda_1 & \mu_1 & \nu_1 \\ \lambda_2 & \mu_2 & \nu_2 \\ \lambda_3 & \mu_3 & \nu_3} =
\end{equation*}
\begin{equation*}
   = \lambda_{1}\mu_{2} \nu_{3} - \lambda_{1}\nu_{2} \mu_{3} - \mu_{1}\lambda_{2} \nu_{3} + \mu_{1}\nu_{2}\lambda_{3} + \nu_{1}\lambda_{2} \mu_{3} - \nu_{1}\mu_{2} \lambda_{3} =
\end{equation*}
\begin{equation*}
    = \lambda_{1}\mu_{2}\nu_{3} - \lambda_{1}\nu_{2}\mu_{3} + \mu_{1}\nu_{2}\lambda_{3} - \mu_{1}\lambda_{2}\nu_{3} + \nu_{1}\lambda_{2}\mu_{3} - \nu_{1}\mu_{2}\lambda_{3}
 \end{equation*}
which is the same as the triple scalar product.

% {{1, 2, 3}, {4, 5, 6}, {42, 69, 44}}

% {{4, 5, 6}, {42, 69, 44}, {1, 2, 3}}

% {{42, 69, 44}, {1, 2, 3}, {4, 5, 6}}

We will use the properties of determinants to deduce that cyclic rotation of the vectors in a triple scalar product leaves the value of the
product unchanged. Since the interchange of any two rows (or columns) of the determinant changes its sign we can put the following
chain of equalities for cyclic rotations, namely:

\begin{equation*}
    \mdet{\lambda_1 & \mu_1 & \nu_1 \\ \lambda_2 & \mu_2 & \nu_2 \\ \lambda_3 & \mu_3 & \nu_3} 
\end{equation*}
\begin{equation*}
    = - \mdet{\lambda_2 & \mu_2 & \nu_2 \\ \lambda_1 & \mu_1 & \nu_1 \\ \lambda_3 & \mu_3 & \nu_3} 
\end{equation*}
\begin{equation*}
    = \mdet{\lambda_2 & \mu_2 & \nu_2 \\ \lambda_3 & \mu_3 & \nu_3 \\ \lambda_1 & \mu_1 & \nu_1} 
\end{equation*}
\begin{equation*}
    = - \mdet{\lambda_3 & \mu_3 & \nu_3 \\ \lambda_2 & \mu_2 & \nu_2 \\ \lambda_1 & \mu_1 & \nu_1} 
\end{equation*}
\begin{equation*}
    = \mdet{\lambda_3 & \mu_3 & \nu_3 \\ \lambda_1 & \mu_1 & \nu_1 \\ \lambda_2 & \mu_2 & \nu_2}
\end{equation*}

Which finally gives us:
\begin{equation*}
    \mathbf{a_1} \cdot (\mathbf{a_2} \times \mathbf{a_3}) = \mdet{\lambda_1 & \mu_1 & \nu_1 \\ \lambda_2 & \mu_2 & \nu_2 \\ \lambda_3 & \mu_3 & \nu_3} = \mdet{\lambda_2 & \mu_2 & \nu_2 \\ \lambda_3 & \mu_3 & \nu_3 \\ \lambda_1 & \mu_1 & \nu_1} = \mathbf{a_2} \cdot (\mathbf{a_3} \times \mathbf{a_1}) = \mdet{\lambda_3 & \mu_3 & \nu_3 \\ \lambda_1 & \mu_1 & \nu_1 \\ \lambda_2 & \mu_2 & \nu_2} = 
\end{equation*}
\begin{equation*}
    = \mathbf{a_3} \cdot (\mathbf{a_1} \times \mathbf{a_2})
\end{equation*}
\end{document} 
