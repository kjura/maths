\documentclass{article}
\usepackage{amsmath}
\usepackage{amssymb}
\usepackage{amsthm}
\usepackage[T1]{fontenc}


\newtheorem*{conjecture}{Conjecture}
\newtheorem*{theorem}{Theorem}


\makeatletter
\newcommand*{\rom}[1]{\expandafter\@slowromancap\romannumeral #1@}
\makeatother


\begin{document}

\section*{2.54}

\subsection*{(a)}

\begin{equation*}
    U = 2X + 3
\end{equation*}



Rozkład \(X\) wygląda następująco: \(X \sim \{(-3, \frac{1}{10}), (-1, \frac{2}{10}), (3, \frac{5}{10}), (5, \frac{2}{10})\}\)
Szukamy rozkładu dla \(U\). Funkcją transformującą jest \(\phi(x) = 2x + 3\). Policzmy zatem punkty skokowe zmiennej \(U\):
\begin{equation*}
    u_1 = 2 \cdot -3 + 3 = -3
\end{equation*}
\begin{equation*}
    u_2 = 2 \cdot -1 + 3 = 1
\end{equation*}
\begin{equation*}
    u_3 = 2 \cdot -3 + 3 = 9
\end{equation*}
\begin{equation*}
    u_4 = 2 \cdot -5 + 3 = 13
\end{equation*}

Teraz liczymy prawdopodobieństwa zmiennej losowej \(U\) dla powyższych punktów skokowych:

\begin{equation*}
    P(U = -3) = P(\{\omega: 2X + 3 = -3\}) = P(X = -3) = 0.1
\end{equation*}
\begin{equation*}
    P(U = 1) = P(X = -1) = 0.2
\end{equation*}
\begin{equation*}
    P(U = 9) = P(X = 3) = 0.5
\end{equation*}
\begin{equation*}
    P(U = 13) = P(X = 5) = 0.2
\end{equation*}

Ostatecznie \(U \sim \{(-3, \frac{1}{10}), (1, \frac{2}{10}), (9, \frac{5}{10}), (13, \frac{2}{10})\}\)

\end{document}
