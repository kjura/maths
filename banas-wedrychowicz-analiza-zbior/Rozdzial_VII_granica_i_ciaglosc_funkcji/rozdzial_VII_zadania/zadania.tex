\documentclass{article}
\usepackage{amsmath}
\usepackage{amssymb}
\usepackage{amsthm}
\usepackage[T1]{fontenc}


\newtheorem*{conjecture}{Przypuszczenie}
\newtheorem*{theorem}{Twierdzenie}
\newtheorem{lemma}{Lemat}
\renewcommand*{\proofname}{Dowód}


\makeatletter
\newcommand*{\rom}[1]{\expandafter\@slowromancap\romannumeral #1@}
\makeatother


\DeclareMathOperator{\interior}{int}

\begin{document}

\section{Część A}

\subsection{Obliczyć granice}


\subsubsection*{1}

\begin{equation*}
    \lim_{x \to -1} \frac{x^4 + 3x^2 -4}{x + 1}
\end{equation*}

Używając schematu Hornera uzyskujemy \(x^4 + 3x^2 -4 = (x^2 + 4) (x - 1) (x + 1)\)

Zatem:
\begin{equation*}
    \lim_{x \to -1} \frac{x^4 + 3x^2 -4}{x + 1} = \lim_{x \to -1} \frac{(x^2 + 4) (x - 1) (x + 1)}{x + 1} = \lim_{x \to -1} (x^2 + 4)(x - 1) =
\end{equation*}
\begin{equation*}
    = (-1^2 + 4)(-1 - 1) = 5 \cdot -2 = -10  
\end{equation*}

\subsubsection*{2}

\begin{center}
    NOTATKA: jako że \(x \to +\infty\), to możemy założyć \(x>0\) i wtedy \(|x| = x\). Nie zmieni to wyniku granicy.
    Analogicznie dla \(x \to -\infty\) mamy \(|x| = -x\)
\end{center}
\begin{center}
    NOTATKA 2: Warto się przekonać formalnie o \(\lim_{x \to p} \sqrt{f(x)} = \sqrt{\lim_{x \to p} f(x)}\) (także \(f(x) >= 0\) bo pierwiastek)
\end{center}

\begin{equation*}
    \lim_{x \to +\infty} (\sqrt{x^2 + 1} - \sqrt{x^2 - 1}) = \lim_{x \to +\infty} (\sqrt{x^2 + 1} - \sqrt{x^2 - 1}) \frac{\sqrt{x^2 + 1} + \sqrt{x^2 - 1}}{\sqrt{x^2 + 1} + \sqrt{x^2 - 1}} =
\end{equation*}
\begin{equation*}
    = \lim_{x \to +\infty} \frac{x^2 + 1 -x^2 + 1}{\sqrt{x^2 + 1} + \sqrt{x^2 - 1}} = \lim_{x \to +\infty} \frac{2}{\sqrt{x^2 + 1} + \sqrt{x^2 - 1}} =
\end{equation*}
\begin{equation*}
    = \lim_{x \to +\infty} \frac{2}{\sqrt{x^2} \sqrt{1 + \frac{1}{x^2}} + \sqrt{x^2}\sqrt{1 - \frac{1}{x^2}}} = \lim_{x \to +\infty} \frac{2}{\sqrt{x^2}( \sqrt{1 + \frac{1}{x^2}} + \sqrt{1 - \frac{1}{x^2}})} = 
\end{equation*}
\begin{equation*}
    = \lim_{x \to +\infty} \frac{2}{|x|( \sqrt{1 + \frac{1}{x^2}} + \sqrt{1 - \frac{1}{x^2}})} = \lim_{x \to +\infty} \frac{2}{x( \sqrt{1 + \frac{1}{x^2}} + \sqrt{1 - \frac{1}{x^2}})} =
\end{equation*}
\begin{equation*}
    = \Bigg(\lim_{x \to +\infty} \frac{2}{x}\Bigg) \cdot  \Bigg(\lim_{x \to +\infty} \frac{1}{( \sqrt{1 + \frac{1}{x^2}} + \sqrt{1 - \frac{1}{x^2}})}\Bigg) = 0 \cdot \frac{1}{2} = 0
\end{equation*}

\subsubsection*{3}

\begin{center}
    NOTATKA: Podobno można szybciej, tip: wzór skróconego mnożenia na górze, wolfram po komendzie simplify
    potrafi zrobić z tego \( \sqrt{x} + 2 \cdot x^{1/4} + 4 \) a wtedy podstawienie \(12\) leci od razu.
\end{center}


\textbf{Sposób pierwszy - nieoptymalny}
\begin{equation*}
    \lim_{x \to 16} \frac{\sqrt{x \sqrt{x}} - 8}{\sqrt[4]{x} - 2} = \lim_{x \to 16} \frac{\sqrt[4]{x^3} - 8}{\sqrt[4]{x} - 2} =
\end{equation*}
\begin{equation*}
    = \lim_{x \to 16} \Bigg(\frac{\sqrt[4]{x^3} - 8}{\sqrt[4]{x} - 2}\Bigg) \Bigg(\frac{\sqrt[4]{x^3} + 2\sqrt[4]{x^2} + 2^{2}\sqrt[4]{x} + 2^3}{\sqrt[4]{x^3} + 2\sqrt[4]{x^2} + 2^{2}\sqrt[4]{x} + 2^3}\Bigg) = 
\end{equation*}
\begin{equation*}
    = \lim_{x \to 16} \frac{(\sqrt[4]{x^3} - 8)(\sqrt[4]{x^3} + 2\sqrt[4]{x^2} + 2^{2}\sqrt[4]{x} + 2^3)}{x - 16} = 
\end{equation*}
\begin{equation*}
    = \lim_{x \to 16} \Bigg(\frac{(\sqrt[4]{x^3} - 8)(\sqrt[4]{x^3} + 2\sqrt[4]{x^2} + 2^{2}\sqrt[4]{x} + 2^3)}{x - 16}\Bigg) \cdot 
\end{equation*}
\begin{equation*}
    \cdot \Bigg(\frac{\sqrt[4]{x^9} + 8\sqrt[4]{x^6} + 8^{2}\sqrt[4]{x^3} + 8^3}{\sqrt[4]{x^9} + 8\sqrt[4]{x^6} + 8^{2}\sqrt[4]{x^3} + 8^3}\Bigg) =
\end{equation*}
\begin{equation*}
    = \lim_{x \to 16} \frac{(x^3 - 8^4)}{(x - 16)}\frac{(\sqrt[4]{x^3} + 2\sqrt[4]{x^2} + 2^{2}\sqrt[4]{x} + 2^3)}{(\sqrt[4]{x^9} + 8\sqrt[4]{x^6} + 8^{2}\sqrt[4]{x^3} + 8^3)}
\end{equation*}

Dzieląc wielomian \(x^3 - 8^4 = x^3 - 16^3\) przez \(x-16\) otrzymujemy \(x^3 - 16^3 = (x-16)(x^2 + 16x + 256)\). Zatem:

\begin{equation*}
    = \lim_{x \to 16} \frac{(x-16)(x^2 + 16x + 256)}{(x - 16)}\frac{(\sqrt[4]{x^3} + 2\sqrt[4]{x^2} + 2^{2}\sqrt[4]{x} + 2^3)}{(\sqrt[4]{x^9} + 8\sqrt[4]{x^6} + 8^{2}\sqrt[4]{x^3} + 8^3)} =
\end{equation*}
\begin{equation*}
    = \lim_{x \to 16} \frac{(x^2 + 16x + 256)(\sqrt[4]{x^3} + 2\sqrt[4]{x^2} + 2^{2}\sqrt[4]{x} + 2^3)}{(\sqrt[4]{x^9} + 8\sqrt[4]{x^6} + 8^{2}\sqrt[4]{x^3} + 8^3)}
\end{equation*}
\begin{equation*}
    = \frac{(16^2 + 16^2 + 256)(\sqrt[4]{16^3} + 2\sqrt[4]{16^2} + 2^{2}\sqrt[4]{16} + 2^3)}{(\sqrt[4]{16^9} + 8\sqrt[4]{16^6} + 8^{2}\sqrt[4]{16^3} + 8^3)} =
\end{equation*}
\begin{equation*}
    = \frac{(16^2 + 16^2 + 16^2)(2^3 + 2^{1} \cdot 2^{2} + 2^{2} \cdot 2^{1} + 2^3)}{(2^9 + 2^{3} \cdot 2^{6} + 2^{6} \cdot 2^{3} + 2^{9})} = 
\end{equation*}
\begin{equation*}
    = \frac{(2^8 + 2^8 + 2^8)(2^3 + 2^3 + 2^3 + 2^3)}{(2^9 + 2^9 + 2^9 + 2^9)} = 
\end{equation*}
\begin{equation*}
    =  \frac{(3 \cdot 2^8)(4 \cdot 2^3)}{(4 \cdot 2^9)} = \frac{3 \cdot 2^8}{2^6} = 3 \cdot 2^2 = 12
\end{equation*}

\textbf{Sposób drugi: optymalny}

Podstawmy \(y = \sqrt[4]{x}\). Wtedy \(y^3 = \sqrt[4]{x^3}\). Wychodzimy na:

\begin{equation*}
    \frac{\sqrt{x \sqrt{x}} - 8}{\sqrt[4]{x} - 2} = \frac{\sqrt[4]{x^3} - 8}{\sqrt[4]{x} - 2} = \frac{y^3 - 8}{y - 2} = 
\end{equation*}
\begin{equation*}
    = \frac{(y - 2)(y^2 + 2y + 4)}{y -2} = (y^2 + 2y + 4) = \sqrt[4]{x^2} + 2\sqrt[4]{x} + 4 
\end{equation*}
%
Zatem:
%
\begin{equation*}
    \lim_{x \to 16} \frac{\sqrt{x \sqrt{x}} - 8}{\sqrt[4]{x} - 2} = \lim_{x \to 16} \sqrt[4]{x^2} + 2\sqrt[4]{x} + 4 = \sqrt[4]{16^2} + 2\sqrt[4]{16} + 4 = 4 + 2 \cdot 2 + 4 = 12
\end{equation*}

\textbf{Jedna notatka warta wspomnienia}. Jeśli \(x\to 16\) to \(\sqrt[4]{x} \to \sqrt[4]{16}=2\) gdy chcemy
dokonać zamiany zmiennej pod granicą.

\subsubsection*{4}

\begin{equation*}
    \lim_{x \to 1} \frac{1 - \sqrt[3]{x}}{1 - \sqrt[5]{x}} = \lim_{x \to 1} \frac{1 - \sqrt[3]{x}}{1 - \sqrt[5]{x}}\Bigg(\frac{(1 + \sqrt[3]{x} + \sqrt[3]{x}^2)}{(1 + \sqrt[3]{x} + \sqrt[3]{x}^2)}\Bigg) =
\end{equation*}
\begin{equation*}
    = \lim_{x \to 1} \frac{1 - \sqrt[3]{x}}{1 - \sqrt[5]{x}} \Bigg(\frac{(1 + \sqrt[3]{x} + \sqrt[3]{x}^2)}{(1 + \sqrt[3]{x} + \sqrt[3]{x}^2)}\Bigg) \Bigg(\frac{1 + \sqrt[5]{x} + \sqrt[5]{x}^2 + \sqrt[5]{x}^3 + \sqrt[5]{x}^4}{1 + \sqrt[5]{x} + \sqrt[5]{x}^2 + \sqrt[5]{x}^3 + \sqrt[5]{x}^4}\Bigg)
\end{equation*}
\begin{equation*}
    = \lim_{x \to 1} \frac{(1 - x)(1 + \sqrt[5]{x} + \sqrt[5]{x}^2 + \sqrt[5]{x}^3 + \sqrt[5]{x}^4)}{(1 - x)(1 + \sqrt[3]{x} + \sqrt[3]{x}^2)}
\end{equation*}
\begin{equation*}
    = \lim_{x \to 1} \frac{(1 + \sqrt[5]{x} + \sqrt[5]{x}^2 + \sqrt[5]{x}^3 + \sqrt[5]{x}^4)}{(1 + \sqrt[3]{x} + \sqrt[3]{x}^2)}
\end{equation*}
\begin{equation*}
    = \frac{1 + \sqrt[5]{1} + \sqrt[5]{1^2} + \sqrt[5]{1^3} + \sqrt[5]{1^4}}{1 + \sqrt[3]{1} + \sqrt[3]{1^2}} = \frac{5}{3}
\end{equation*}

\subsubsection*{5}

\begin{equation*}
    \lim_{x \to 0} \frac{\sqrt{x^2 + 1} - \sqrt{x + 1}}{1 - \sqrt{x + 1}} = \lim_{x \to 0} \Bigg(\frac{\sqrt{x^2 + 1} - \sqrt{x + 1}}{1 - \sqrt{x + 1}}\Bigg) \Bigg(\frac{1 + \sqrt{x + 1}}{1 + \sqrt{x + 1}}\Bigg)
\end{equation*}
\begin{equation*}
    = \lim_{x \to 0} \frac{\Big(\sqrt{x^2 + 1} - \sqrt{x + 1}\Big)(1 + \sqrt{x + 1})}{1 - x - 1}
\end{equation*}
\begin{equation*}
    = \lim_{x \to 0} \frac{\Big(\sqrt{x^2 + 1} - \sqrt{x + 1}\Big)(1 + \sqrt{x + 1})}{-x}
\end{equation*}
\begin{equation*}
    = \lim_{x \to 0} \frac{\Big(\sqrt{x^2 + 1} - \sqrt{x + 1}\Big)(1 + \sqrt{x + 1})}{-x} \frac{(\sqrt{x^2 + 1} + \sqrt{x + 1})}{(\sqrt{x^2 + 1} + \sqrt{x + 1})}
\end{equation*}
\begin{equation*}
    = \lim_{x \to 0} \frac{(x^2 + 1 -x - 1)(1 + \sqrt{x + 1})}{-x(\sqrt{x^2 + 1} + \sqrt{x + 1})}
\end{equation*}
\begin{equation*}
    = \lim_{x \to 0} \frac{-x(1 - x)(1 + \sqrt{x + 1})}{-x(\sqrt{x^2 + 1} + \sqrt{x + 1})}
\end{equation*}
\begin{equation*}
    = \lim_{x \to 0} \frac{(1 - x)(1 + \sqrt{x + 1})}{(\sqrt{x^2 + 1} + \sqrt{x + 1})}
\end{equation*}
\begin{equation*}
    = \frac{(1 - 0)(1 + \sqrt{0 + 1})}{(\sqrt{0^2 + 1} + \sqrt{0 + 1})} = \frac{(1 + 1)}{(1 + 1)} = \frac{2}{2} = 1
\end{equation*}


\subsubsection*{6}

\begin{center}
    NOTATKA: Warto przekonać się o istnieniu granicy (poza zerem) dla \(\frac{|x|}{x}\) i \(\frac{x}{|x|}\)
\end{center}

\begin{equation*}
    \lim_{x \to -\infty} (\sqrt{x^2 -8x + 3} - \sqrt{x^2 +11x}) 
\end{equation*}
\begin{equation*}
    = \lim_{x \to -\infty} (\sqrt{x^2 -8x + 3} - \sqrt{x^2 +11x}) \frac{(\sqrt{x^2 -8x + 3} + \sqrt{x^2 +11x})}{(\sqrt{x^2 -8x + 3} + \sqrt{x^2 +11x})} 
\end{equation*}
\begin{equation*}
    = \lim_{x \to -\infty} \frac{x^2 -8x + 3 -x^2 - 11x}{(\sqrt{x^2 -8x + 3} + \sqrt{x^2 +11x})}
\end{equation*}
\begin{equation*}
    = \lim_{x \to -\infty} \frac{-19x + 3}{(\sqrt{x^2 -8x + 3} + \sqrt{x^2 +11x})}
\end{equation*}
\begin{equation*}
    = \lim_{x \to -\infty} \frac{x(\frac{3}{x} - 19)}{\Big(\sqrt{x^2(1 - \frac{8}{x} + \frac{3}{x^2})} + \sqrt{x^2(1 + \frac{11}{x}})\Big)}
\end{equation*}
\begin{equation*}
    = \lim_{x \to -\infty} \frac{x(\frac{3}{x} - 19)}{\Big(\sqrt{x^2}\sqrt{(1 - \frac{8}{x} + \frac{3}{x^2})} + \sqrt{x^2}\sqrt{(1 + \frac{11}{x}})\Big)}
\end{equation*}
\begin{equation*}
    = \lim_{x \to -\infty} \frac{x(\frac{3}{x} - 19)}{\Big(|x|\sqrt{(1 - \frac{8}{x} + \frac{3}{x^2})} + |x|\sqrt{(1 + \frac{11}{x}})\Big)}
\end{equation*}
\begin{equation*}
    = \lim_{x \to -\infty} \frac{x(\frac{3}{x} - 19)}{|x| \Big(\sqrt{(1 - \frac{8}{x} + \frac{3}{x^2})} + \sqrt{(1 + \frac{11}{x}}\Big)}
\end{equation*}
\begin{equation*}
    = \Bigg(\lim_{x \to -\infty} \frac{x}{|x|}(\frac{3}{x} - 19)\Bigg) \cdot \Bigg(\lim_{x \to -\infty} \frac{1}{\Big(\sqrt{(1 - \frac{8}{x} + \frac{3}{x^2})} + \sqrt{(1 + \frac{11}{x}}\Big)}\Bigg)
\end{equation*}
\begin{equation*}
    = \Bigg(\lim_{x \to -\infty} \frac{x}{|x|}\Bigg) \cdot \Bigg(\lim_{x \to -\infty} \frac{3}{x} - 19 \Bigg) \cdot \frac{1}{\lim_{x \to -\infty} \Big(\sqrt{(1 - \frac{8}{x} + \frac{3}{x^2})} + \sqrt{(1 + \frac{11}{x}}\Big)}
\end{equation*}
\begin{equation*}
   = \Bigg(\lim_{x \to -\infty} \frac{x}{|x|}\Bigg) \cdot \Bigg(\lim_{x \to -\infty} \frac{3}{x} - 19 \Bigg) \cdot
\end{equation*}
\begin{equation*}
    \cdot \frac{1}{\Big(\lim_{x \to -\infty} \sqrt{(1 - \frac{8}{x} + \frac{3}{x^2})}\Big) + \Big(\lim_{x \to -\infty} \sqrt{(1 + \frac{11}{x}}\Big)}
\end{equation*}
\begin{equation*}
    = \Bigg(-1 \Bigg) \cdot \Bigg( 0 - 19 \Bigg) \cdot
\end{equation*}
\begin{equation*}
    \cdot \frac{1}{\Big( \sqrt{(1 - 0 + 0)}\Big) + \Big(\sqrt{(1 + 0}\Big)}
\end{equation*}
\begin{equation*}
    = \Bigg(-1 \Bigg) \cdot \Bigg( 0 - 19 \Bigg) \cdot \frac{1}{2} = \frac{19}{2}
\end{equation*}

\subsubsection*{7}

\begin{equation*}
    \lim_{x \to -\infty} \Bigg(\sqrt{2x^2 + 1} + \sqrt[3]{x^3 - x + 2}\Bigg) = 
\end{equation*}
\begin{equation*}
    = \lim_{x \to -\infty} \Bigg(\sqrt{x^{2}(2 + \frac{1}{x^2})} + \sqrt[3]{x^3(1 - \frac{1}{x^2} + \frac{2}{x^3})}\Bigg) 
\end{equation*}
\begin{equation*}
    = \lim_{x \to -\infty} \Bigg(\sqrt{x^{2}} \sqrt{2 + \frac{1}{x^2}} + \sqrt[3]{x^3} \sqrt[3]{1 - \frac{1}{x^2} + \frac{2}{x^3}}\Bigg) 
\end{equation*}
\begin{equation*}
    = \lim_{x \to -\infty} \Bigg(|x| \sqrt{2 + \frac{1}{x^2}} + x \sqrt[3]{1 - \frac{1}{x^2} + \frac{2}{x^3}}\Bigg) 
\end{equation*}
\begin{equation*}
    = \lim_{x \to -\infty} \Bigg(-x \sqrt{2 + \frac{1}{x^2}} + x \sqrt[3]{1 - \frac{1}{x^2} + \frac{2}{x^3}}\Bigg) 
\end{equation*}
\begin{equation*}
    = \lim_{x \to -\infty} (-x) \cdot \Bigg(\sqrt{2 + \frac{1}{x^2}} - \sqrt[3]{1 - \frac{1}{x^2} + \frac{2}{x^3}}\Bigg) 
\end{equation*}
\begin{equation*}
    = + \infty \cdot \Bigg(\sqrt{2 + 0} - \sqrt[3]{1 - 0 + 0}\Bigg) =  + \infty \cdot (\sqrt{2} - 1) = + \infty
\end{equation*}

\subsubsection*{8}

\begin{equation*}
    \lim_{x \to 0} \frac{\sqrt[3]{1+ x} - 1}{x} = \lim_{x \to 0} \frac{\sqrt[3]{1+ x} - 1}{x} \frac{\Bigg((\sqrt[3]{1+ x})^2 + \sqrt[3]{1+ x} + 1\Bigg)}{\Bigg((\sqrt[3]{1+ x})^2 + \sqrt[3]{1+ x} + 1\Bigg)} = 
\end{equation*}
\begin{equation*}
    = \lim_{x \to 0} \frac{1 + x - 1}{x \Big((\sqrt[3]{1+ x})^2 + \sqrt[3]{1+ x} + 1\Big)} = \lim_{x \to 0} \frac{x}{x \Big( (\sqrt[3]{1+ x})^2 + \sqrt[3]{1+ x} + 1)\Big)} =
\end{equation*}
\begin{equation*}
    = \lim_{x \to 0} \frac{1}{\Big( (\sqrt[3]{1+ x})^2 + \sqrt[3]{1+ x} + 1\Big)} = \frac{1}{\Big( (\sqrt[3]{1+ 0})^2 + \sqrt[3]{1+ 0} + 1\Big)} = \frac{1}{3}
\end{equation*}


\subsubsection*{9}

\begin{center}
    NOTATKA: W odpowiedzi błąd: Powinno być \(\frac{4}{3}\) a jest \(\frac{2}{3}\)
\end{center}

\begin{equation*}
    \lim_{x \to + \infty} \Big(\sqrt[3]{x(x+1)^2} - \sqrt[3]{x(x-1)^2}\Big) = 
\end{equation*}
\begin{equation*}
    = \lim_{x \to + \infty} \Big(\sqrt[3]{x(x+1)^2} - \sqrt[3]{x(x-1)^2}\Big) \times 
\end{equation*}
\begin{equation*}
    \times \frac{\Big(\sqrt[3]{x^{2}(x+1)^4} + \sqrt[3]{x(x+1)^{2}x(x-1)^{2}} +  \sqrt[3]{x^{2}(x-1)^4}\Big)}{\Big(\sqrt[3]{x^{2}(x+1)^4} + \sqrt[3]{x(x+1)^{2}x(x-1)^{2}} +  \sqrt[3]{x^{2}(x-1)^4}\Big)} =
\end{equation*}
\begin{equation*}
    = \lim_{x \to + \infty} \frac{x(x+1){^2} - x(x-1)^2}{\Big(\sqrt[3]{x^{2}(x+1)^4} + \sqrt[3]{x(x+1)^{2}x(x-1)^{2}} +  \sqrt[3]{x^{2}(x-1)^4}\Big)}
\end{equation*}
\begin{equation*}
    = \lim_{x \to + \infty} \frac{x \Big((x+1){^2} - (x-1)^2\Big)}{\Big(\sqrt[3]{x^{2}(x+1)^4} + \sqrt[3]{x(x+1)^{2}x(x-1)^{2}} +  \sqrt[3]{x^{2}(x-1)^4}\Big)}
\end{equation*}
\begin{equation*}
    = \lim_{x \to + \infty} \frac{x \Big( 2x - (-2x) \Big)}{\Big(\sqrt[3]{x^{2}(x+1)^4} + \sqrt[3]{x(x+1)^{2}x(x-1)^{2}} +  \sqrt[3]{x^{2}(x-1)^4}\Big)}
\end{equation*}
\begin{equation*}
    = \lim_{x \to + \infty} \frac{x \cdot 4x}{\Big(\sqrt[3]{x^{2}(x+1)^4} + \sqrt[3]{x(x+1)^{2}x(x-1)^{2}} +  \sqrt[3]{x^{2}(x-1)^4}\Big)}
\end{equation*}
\begin{equation*}
    = \lim_{x \to + \infty} \frac{4x^2}{\Big(\sqrt[3]{x^{2}(x+1)^4} + \sqrt[3]{x(x+1)^{2}x(x-1)^{2}} +  \sqrt[3]{x^{2}(x-1)^4}\Big)}
\end{equation*}
\begin{equation*}
    = \lim_{x \to + \infty} \frac{4x^2}{x^{\frac{2}{3}} \cdot \Big(\sqrt[3]{(x+1)^4} + \sqrt[3]{(x+1)^{2}(x-1)^{2}} +  \sqrt[3]{(x-1)^4}\Big)}
\end{equation*}
\begin{equation*}
    = \lim_{x \to + \infty} \frac{4 \sqrt[3]{x^4}}{\Big(\sqrt[3]{(x+1)^4} + \sqrt[3]{(x+1)^{2}(x-1)^{2}} +  \sqrt[3]{(x-1)^4}\Big)}
\end{equation*}
\begin{equation*}
    = \lim_{x \to + \infty} \frac{4 \sqrt[3]{x^4}}{\sqrt[3]{\Big(x \cdot (1 + \frac{1}{x})\Big)^4} + \sqrt[3]{\Big(x \cdot (1 + \frac{1}{x})\Big)^{2}\Big(x \cdot (1 - \frac{1}{x})\Big)^{2}} +  \sqrt[3]{\Big(x \cdot (1 - \frac{1}{x})\Big)^4}}
\end{equation*}
\begin{equation*}
    = \lim_{x \to + \infty} \frac{4 \sqrt[3]{x^4}}{\sqrt[3]{x^{4} (1 + \frac{1}{x}){^4}} + \sqrt[3]{x^{2} (1 + \frac{1}{x})^{2} x^{2} (1 - \frac{1}{x})^{2}} +  \sqrt[3]{x^{4} (1 - \frac{1}{x}){^4}}}
\end{equation*}
\begin{equation*}
    = \lim_{x \to + \infty} \frac{4 \sqrt[3]{x^4}}{\sqrt[3]{x^{4}} \Bigg(\sqrt[3]{(1 + \frac{1}{x}){^4}} + \sqrt[3]{(1 + \frac{1}{x})^{2}(1 - \frac{1}{x})^{2}} +  \sqrt[3]{(1 - \frac{1}{x}){^4}}\Bigg)}
\end{equation*}
\begin{equation*}
    = \lim_{x \to + \infty} \frac{4}{\Bigg(\sqrt[3]{(1 + \frac{1}{x}){^4}} + \sqrt[3]{(1 + \frac{1}{x})^{2}(1 - \frac{1}{x})^{2}} +  \sqrt[3]{(1 - \frac{1}{x}){^4}}\Bigg)}
\end{equation*}
\begin{equation*}
    = \frac{4}{\Bigg(\sqrt[3]{(1 + 0){^4}} + \sqrt[3]{(1 + 0)^{2}(1 - 0)^{2}} +  \sqrt[3]{(1 - 0){^4}}\Bigg)}
\end{equation*}
\begin{equation*}
    = \frac{4}{3}
\end{equation*}

\subsubsection*{10}
Mamy za zadanie policzyć granicę:

\begin{equation*}
    \lim_{x \to 2} \frac{\sqrt{x^3 - 3x^2 +4} -x + 2}{x^2 - 4} 
\end{equation*}


Zaczniemy od przekształcenia wyrażenia:

\begin{equation*}
    \frac{\sqrt{x^3 - 3x^2 +4} -x + 2}{x^2 - 4} = \frac{\sqrt{(x - 2)^{2}(x+1)} -x + 2}{x^2 - 4} =
\end{equation*}
\begin{equation*}
    = \frac{|x-2|\sqrt{x+1} - (x - 2)}{(x -2)(x+2)}
\end{equation*}

Jako, że występuje wartość bezwzględna, sprawdzimy czy granica istnieje licząc granicę prawostronną oraz lewostronną tj.

\begin{equation*}
    \lim_{x \to 2^+} \frac{|x-2|\sqrt{x+1} - (x - 2)}{(x -2)(x+2)}
\end{equation*}
\begin{equation*}
    \lim_{x \to 2^-} \frac{|x-2|\sqrt{x+1} - (x - 2)}{(x -2)(x+2)}
\end{equation*}

Sprawdzamy najpierw granicę lewostronną:

\begin{equation*}
    \lim_{x \to 2^-} \frac{|x-2|\sqrt{x+1} - (x - 2)}{(x -2)(x+2)} = \lim_{x \to 2^-} \frac{-(x - 2)\sqrt{x+1} - (x - 2)}{(x -2)(x+2)} = 
\end{equation*}
\begin{equation*}
    = \lim_{x \to 2^-} \frac{(2 - x)\sqrt{x+1} - (x - 2)}{(x -2)(x+2)} = \lim_{x \to 2^-} \frac{(2 - x)\sqrt{x+1} + (2 - x)}{(x -2)(x+2)}
\end{equation*}
\begin{equation*}
    = \lim_{x \to 2^-} \frac{(2 - x)\sqrt{x+1} + (2 - x)}{(x -2)(x+2)} = \lim_{x \to 2^-} \frac{(2 - x)\sqrt{x+1} + (2 - x)}{-(2 - x)(x+2)}
\end{equation*}
\begin{equation*}
    \lim_{x \to 2^-} \frac{(2 - x)\Big(\sqrt{x+1} + 1 \Big)}{-(2 - x)(x+2)} = \lim_{x \to 2^-} \frac{\Big(\sqrt{x+1} + 1 \Big)}{-(x+2)} = 
\end{equation*}
\begin{equation*}
    = \frac{\Big(\sqrt{2+1} + 1 \Big)}{-(2+2)} = \frac{\sqrt{3} + 1}{-4}
\end{equation*}

A następnie granicę prawostronną:

\begin{equation*}
    \lim_{x \to 2^+} \frac{|x-2|\sqrt{x+1} - (x - 2)}{(x -2)(x+2)} = \lim_{x \to 2^+} \frac{(x-2)\sqrt{x+1} - (x - 2)}{(x -2)(x+2)} = 
\end{equation*}
\begin{equation*}
    = \lim_{x \to 2^+} \frac{\sqrt{x+1} - 1}{(x + 2)} = \frac{\sqrt{2 + 1} -1}{2 + 2} = \frac{\sqrt{3} - 1}{4}
\end{equation*}

Zatem granica:

\begin{equation*}
    \lim_{x \to 2} \frac{\sqrt{x^3 - 3x^2 +4} -x + 2}{x^2 - 4} 
\end{equation*}

nie istnieje bo granice jednostronne nie są sobie równe.

\subsubsection*{11}
\begin{equation*}
    \lim_{x \to + \infty} x^{3}\Bigg(\sqrt{x^{2} + \sqrt{x^{4} + 1}} - x\sqrt{2}\Bigg) =
\end{equation*}
\begin{equation*}
    \lim_{x \to + \infty} x^{3}\Bigg(\sqrt{x^{2} + \sqrt{x^{4} + 1}} - x\sqrt{2}\Bigg) \Bigg(\frac{\sqrt{x^{2} + \sqrt{x^{4} + 1}} + x\sqrt{2}}{\sqrt{x^{2} + \sqrt{x^{4} + 1}} + x\sqrt{2}}\Bigg) =
\end{equation*}
\begin{equation*}
    \lim_{x \to + \infty} x^{3} \Bigg(\frac{x^{2} + \sqrt{x^{4} + 1} - 2x^{2}}{\sqrt{x^{2} + \sqrt{x^{4} + 1}} + x\sqrt{2}}\Bigg) =
\end{equation*}
\begin{equation*}
    \lim_{x \to + \infty} x^{3} \Bigg(\frac{ \sqrt{x^{4} + 1} - x^{2}}{\sqrt{x^{2} + \sqrt{x^{4} + 1}} + x\sqrt{2}}\Bigg) =
\end{equation*}
\begin{equation*}
    \lim_{x \to + \infty} x^{3} \Bigg(\frac{ \sqrt{x^{4}(1 + \frac{1}{x^4})} - x^{2}}{\sqrt{x^{2} + \sqrt{x^{4} + 1}} + x\sqrt{2}}\Bigg) =
\end{equation*}
\begin{equation*}
    \lim_{x \to + \infty} x^{3} \Bigg(\frac{ \sqrt{x^{4}} \sqrt{1 + \frac{1}{x^4}} - x^{2}}{\sqrt{x^{2} + \sqrt{x^{4} + 1}} + x\sqrt{2}}\Bigg) =
\end{equation*}
\begin{equation*}
    \lim_{x \to + \infty} x^{3} \Bigg(\frac{ x^2 \sqrt{1 + \frac{1}{x^4}} - x^{2}}{\sqrt{x^{2} + \sqrt{x^{4} + 1}} + x\sqrt{2}}\Bigg) =
\end{equation*}
\begin{equation*}
    \lim_{x \to + \infty} x^{3} \Bigg(\frac{ x^2 \Big(\sqrt{1 + \frac{1}{x^4}} - 1 \Big)}{\sqrt{x^{2} + \sqrt{x^{4} + 1}} + x\sqrt{2}}\Bigg) =
\end{equation*}
\begin{equation*}
    \lim_{x \to + \infty} x^{3} \Bigg(\frac{ x^2 \Big(\sqrt{1 + \frac{1}{x^4}} - 1 \Big)}{\sqrt{x^{2} + x^2 \sqrt{1 + \frac{1}{x^4}}} + x\sqrt{2}}\Bigg) =
\end{equation*}
\begin{equation*}
    \lim_{x \to + \infty} x^{3} \Bigg(\frac{ x^2 \Big(\sqrt{1 + \frac{1}{x^4}} - 1 \Big)}{\sqrt{x^{2} \Big(1 + \sqrt{1 + \frac{1}{x^4}}\Big)} + x\sqrt{2}}\Bigg) =
\end{equation*}
\begin{equation*}
    \lim_{x \to + \infty} x^{3} \Bigg(\frac{ x^2 \Big(\sqrt{1 + \frac{1}{x^4}} - 1 \Big)}{\sqrt{x^{2}} \sqrt{1 + \sqrt{1 + \frac{1}{x^4}}} + x\sqrt{2}}\Bigg) =
\end{equation*}
\begin{equation*}
    \lim_{x \to + \infty} x^{3} \Bigg(\frac{ x^2 \Big(\sqrt{1 + \frac{1}{x^4}} - 1 \Big)}{|x| \sqrt{1 + \sqrt{1 + \frac{1}{x^4}}} + x\sqrt{2}}\Bigg) =
\end{equation*}
\begin{equation*}
    \lim_{x \to + \infty} x^{3} \Bigg(\frac{ x^2 \Big(\sqrt{1 + \frac{1}{x^4}} - 1 \Big)}{x \sqrt{1 + \sqrt{1 + \frac{1}{x^4}}} + x\sqrt{2}}\Bigg) =
\end{equation*}
\begin{equation*}
    \lim_{x \to + \infty} x^{3} \Bigg(\frac{ x^2 \Big(\sqrt{1 + \frac{1}{x^4}} - 1 \Big)}{x \Bigg(\sqrt{1 + \sqrt{1 + \frac{1}{x^4}}} + \sqrt{2}\Bigg)}\Bigg) =
\end{equation*}
\begin{equation*}
    \lim_{x \to + \infty}  \frac{ x^{4} \Big(\sqrt{1 + \frac{1}{x^4}} - 1 \Big)}{\sqrt{1 + \sqrt{1 + \frac{1}{x^4}}} + \sqrt{2}} =
\end{equation*}
\begin{equation*}
    \lim_{x \to + \infty}  \frac{ x^{4} \Big(\sqrt{1 + \frac{1}{x^4}} - 1 \Big)}{\sqrt{1 + \sqrt{1 + \frac{1}{x^4}}} + \sqrt{2}}\frac{\Big(\sqrt{1 + \frac{1}{x^4}} + 1 \Big)}{\Big(\sqrt{1 + \frac{1}{x^4}} + 1 \Big)} =
\end{equation*}
\begin{equation*}
    \lim_{x \to + \infty}  \frac{ x^{4} \Big(1 + \frac{1}{x^4} - 1 \Big)}{\sqrt{1 + \sqrt{1 + \frac{1}{x^4}}} + \sqrt{2}}\frac{1}{\Big(\sqrt{1 + \frac{1}{x^4}} + 1 \Big)} =
\end{equation*}
\begin{equation*}
    \lim_{x \to + \infty}  \frac{ 1}{\Big(\sqrt{1 + \sqrt{1 + \frac{1}{x^4}}} + \sqrt{2}\Big) \Big(\sqrt{1 + \frac{1}{x^4}} + 1 \Big)} =
\end{equation*}
\begin{equation*}
    \frac{ 1}{\Big(\sqrt{1 + \sqrt{1 + 0}} + \sqrt{2}\Big) \Big(\sqrt{1 + 0} + 1 \Big)} = \frac{1}{(\sqrt{2} + \sqrt{2}) \cdot 2} = \frac{1}{4\sqrt{2}} 
\end{equation*}


\subsubsection*{13}

\begin{equation*}
    \lim_{x \to - \infty} \Big(\sqrt{x^2 +2x} + \sqrt[3]{x^3 + x^2}\Big) = \lim_{x \to - \infty} \Big(\sqrt{x^2 + 2x} + \sqrt[3]{- (-x^3 - x^2)}\Big) = 
\end{equation*}
\begin{equation*}
    \lim_{x \to - \infty} \Big(\sqrt{x^2 + 2x} + \sqrt[3]{-1}\sqrt[3]{(-x^3 - x^2)}\Big) = 
\end{equation*}
\begin{equation*}
    \lim_{x \to - \infty} \Big(\sqrt{x^2 + 2x} - \sqrt[3]{(-x^3 - x^2)}\Big) = 
\end{equation*}
\begin{equation*}
    \lim_{x \to - \infty} \Big(\sqrt[6]{(x^2 + 2x)^{3}} - \sqrt[6]{(-x^3 - x^2)^{2}}\Big) 
\end{equation*}


Niech \(G = \Big(\sqrt[6]{(x^2 + 2x)^{3}} - \sqrt[6]{(-x^3 - x^2)^{2}}\Big)\) oraz  niech \(S\) będzie:
\begin{equation*}
    S = \sqrt[6]{(x^2 +2x)^{15}} + \sqrt[6]{(x^2 +2x)^{12}} \cdot \sqrt[6]{(x^3 + x^2)^{2}} + \sqrt[6]{(x^2 + 2x)^{9}} \cdot \sqrt[6]{(x^3 + x^2)^{4}} +
\end{equation*}
\begin{equation*}
    + \sqrt[6]{(x^2 +2x)^{6}} \cdot \sqrt[6]{(x^3 + 2x)^{6}} + \sqrt[6]{(x^2 + 2x)^{3}} \cdot \sqrt[6]{(x^3 + x^2)^{8}} + \sqrt[6]{(x^3 + x^2)^{10}}
\end{equation*}

Wtedy:
\begin{equation*}
    \lim_{x \to - \infty} \Big(\sqrt[6]{(x^2 + 2x)^{3}} - \sqrt[6]{(-x^3 - x^2)^{2}}\Big) = \lim_{x \to - \infty} \frac{G \cdot S}{S} =
\end{equation*}
\begin{equation*}
    = \lim_{x \to - \infty} \Bigg(\Big((x^2 + 2x)^{3} - (-x^3 - x^2)^{2}\Big) \cdot \frac{1}{S}\Bigg) =
\end{equation*}
\begin{equation*}
    = \lim_{x \to - \infty} \Bigg(\Big((x^2 + 2x)^{3} - (-x^3 - x^2)^{2}\Big)\Bigg) \times % 
\end{equation*}
\begin{equation*}
    \times \Bigg(x^{5}\sqrt[6]{(1 + \frac{2}{x})^{15}} + x^{5}\sqrt[6]{(1 + \frac{2}{x})^{12}}\sqrt[6]{(1 + \frac{1}{x})^{2}}  + x^{5}\sqrt[6]{(1 + \frac{2}{x})^{9}}\sqrt[6]{(1 + \frac{1}{x})^{4}}
\end{equation*}
\begin{equation*}
    +  x^{5}\sqrt[6]{(1 + \frac{2}{x})^{6}}\sqrt[6]{(1 + \frac{1}{x})^{6}} + x^{5}\sqrt[6]{(1 + \frac{2}{x})^{3}}\sqrt[6]{(1 + \frac{1}{x})^{8}} + x^{5}\sqrt[6]{(1 + \frac{1}{x})^{10}}\Bigg)^{-1} = 
\end{equation*}
\begin{equation*}
    = \lim_{x \to - \infty} \Bigg(\Big((x^2 + 2x)^{3} - (-x^3 - x^2)^{2}\Big)\Bigg) \times % 
\end{equation*}
\begin{equation*}
    x^{5}\Bigg(\sqrt[6]{(1 + \frac{2}{x})^{15}} + \sqrt[6]{(1 + \frac{2}{x})^{12}}\sqrt[6]{(1 + \frac{1}{x})^{2}}  + \sqrt[6]{(1 + \frac{2}{x})^{9}}\sqrt[6]{(1 + \frac{1}{x})^{4}}
\end{equation*}
\begin{equation*}
    +  \sqrt[6]{(1 + \frac{2}{x})^{6}}\sqrt[6]{(1 + \frac{1}{x})^{6}} + \sqrt[6]{(1 + \frac{2}{x})^{3}}\sqrt[6]{(1 + \frac{1}{x})^{8}} + \sqrt[6]{(1 + \frac{1}{x})^{10}}\Bigg)^{-1} = 
\end{equation*}
\begin{equation*}
    = \lim_{x \to - \infty} \Bigg(\Big(4x^5 + 11x^4 + 8x^3\Big)\Bigg) \times % 
\end{equation*}
\begin{equation*}
    x^{5}\Bigg(\sqrt[6]{(1 + \frac{2}{x})^{15}} + \sqrt[6]{(1 + \frac{2}{x})^{12}}\sqrt[6]{(1 + \frac{1}{x})^{2}}  + \sqrt[6]{(1 + \frac{2}{x})^{9}}\sqrt[6]{(1 + \frac{1}{x})^{4}}
\end{equation*}
\begin{equation*}
    +  \sqrt[6]{(1 + \frac{2}{x})^{6}}\sqrt[6]{(1 + \frac{1}{x})^{6}} + \sqrt[6]{(1 + \frac{2}{x})^{3}}\sqrt[6]{(1 + \frac{1}{x})^{8}} + \sqrt[6]{(1 + \frac{1}{x})^{10}}\Bigg)^{-1} = 
\end{equation*}
\begin{equation*}
    = \lim_{x \to - \infty} \Bigg(x^{5} \Big(4 + \frac{11}{x} + \frac{8}{x^2}\Big)\Bigg) \times % 
\end{equation*}
\begin{equation*}
    x^{5}\Bigg(\sqrt[6]{(1 + \frac{2}{x})^{15}} + \sqrt[6]{(1 + \frac{2}{x})^{12}}\sqrt[6]{(1 + \frac{1}{x})^{2}}  + \sqrt[6]{(1 + \frac{2}{x})^{9}}\sqrt[6]{(1 + \frac{1}{x})^{4}}
\end{equation*}
\begin{equation*}
    +  \sqrt[6]{(1 + \frac{2}{x})^{6}}\sqrt[6]{(1 + \frac{1}{x})^{6}} + \sqrt[6]{(1 + \frac{2}{x})^{3}}\sqrt[6]{(1 + \frac{1}{x})^{8}} + \sqrt[6]{(1 + \frac{1}{x})^{10}}\Bigg)^{-1} = 
\end{equation*}

\begin{equation*}
    = \lim_{x \to - \infty} \Bigg(\Big(4 + \frac{11}{x} + \frac{8}{x^2}\Big)\Bigg) \times % 
\end{equation*}
\begin{equation*}
    \Bigg(\sqrt[6]{(1 + \frac{2}{x})^{15}} + \sqrt[6]{(1 + \frac{2}{x})^{12}}\sqrt[6]{(1 + \frac{1}{x})^{2}}  + \sqrt[6]{(1 + \frac{2}{x})^{9}}\sqrt[6]{(1 + \frac{1}{x})^{4}}
\end{equation*}
\begin{equation*}
    +  \sqrt[6]{(1 + \frac{2}{x})^{6}}\sqrt[6]{(1 + \frac{1}{x})^{6}} + \sqrt[6]{(1 + \frac{2}{x})^{3}}\sqrt[6]{(1 + \frac{1}{x})^{8}} + \sqrt[6]{(1 + \frac{1}{x})^{10}}\Bigg)^{-1} = 
\end{equation*}

\begin{equation*}
    = \Bigg(\Big(4 + 0 + 0\Big)\Bigg) \times % 
\end{equation*}
\begin{equation*}
    \Bigg(\sqrt[6]{(1 + 0)^{15}} + \sqrt[6]{(1 + 0)^{12}}\sqrt[6]{(1 + 0)^{2}}  + \sqrt[6]{(1 + 0)^{9}}\sqrt[6]{(1 + 0)^{4}}
\end{equation*}
\begin{equation*}
    +  \sqrt[6]{(1 + 0)^{6}}\sqrt[6]{(1 + 0)^{6}} + \sqrt[6]{(1 + 0)^{3}}\sqrt[6]{(1 + 0)^{8}} + \sqrt[6]{(1 + 0)^{10}}\Bigg)^{-1} = \frac{4}{6} = \frac{2}{3}
\end{equation*}




% ((x^2) + (2x))^(1/2) + (((x^3) + (x^2)))^(1/3)


\end{document}
