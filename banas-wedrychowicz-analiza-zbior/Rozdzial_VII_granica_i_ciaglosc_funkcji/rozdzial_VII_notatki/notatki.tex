\documentclass{article}
\usepackage{amsmath}
\usepackage{amssymb}
\usepackage{amsthm}
\usepackage[T1]{fontenc}


\newtheorem*{conjecture}{Przypuszczenie}
\newtheorem*{theorem}{Twierdzenie}
\newtheorem{lemma}{Lemat}
\renewcommand*{\proofname}{Dowód}


\makeatletter
\newcommand*{\rom}[1]{\expandafter\@slowromancap\romannumeral #1@}
\makeatother


\DeclareMathOperator{\interior}{int}

\begin{document}

\begin{theorem}
    Niech \(f, g : \mathbb{R} \to \mathbb{R}\) przy czym \(g\) jest funkcją odwracalną oraz ciągłą w \(x_0\), a \(y_0 := g(x_0)\).
    Niech \(h\) będzie takie, że \(f(x) = h(g(x))\) w pewnym otoczeniu \(x_0\) \((h = f \circ g^{-1})\).
    Wtedy:
    \begin{equation*}
        \lim_{x \to x_0} f(x) = \lim_{y \to y_0} h(y)
    \end{equation*}
\end{theorem}



Załóżmy, że istnieje \(\lim_{y \to y_0} h(y)\). Niech \(x_n\) będzie dowolnym ciągiem zbieżnym do \(x_0\) oraz \(x_n \neq x_0\).

Dla pewnego otoczenia \(x_0\) mamy \(f(x) = h(g(x))\). Na mocy definicji Heinego:

\begin{equation*}
    \lim_{x \to x_0} f(x) = \lim_{n \to + \infty} f(x_n) = \lim_{n \to + \infty} h(g(x_n))
\end{equation*}

Połóżmy \(y_n = g(x_n)\). Mamy wtedy:

\begin{equation*}
    \lim_{n \to + \infty} h(g(x_n)) = \lim_{n \to + \infty} h(y_n)
\end{equation*}

Z ciągłości \(g\) wynika, że \(y_n\) jest zbieżny do \(y_0\), a z założenia, istnieje \(\lim_{y \to y_0} h(y)\), zatem: 

\begin{equation*}
    \lim_{n \to + \infty} h(y_n) = \lim_{y \to y_0} h(y)
\end{equation*}

Stwierdzamy więc, że:

\begin{equation*}
    \lim_{x \to x_0} f(x) = \lim_{y \to y_0} h(y)
\end{equation*}




\end{document}