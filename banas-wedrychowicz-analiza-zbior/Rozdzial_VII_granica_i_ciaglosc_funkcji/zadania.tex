\documentclass{article}
\usepackage{amsmath}
\usepackage{amssymb}
\usepackage{amsthm}
\usepackage[T1]{fontenc}


\newtheorem*{conjecture}{Przypuszczenie}
\newtheorem*{theorem}{Twierdzenie}
\newtheorem{lemma}{Lemat}
\renewcommand*{\proofname}{Dowód}


\makeatletter
\newcommand*{\rom}[1]{\expandafter\@slowromancap\romannumeral #1@}
\makeatother


\DeclareMathOperator{\interior}{int}

\begin{document}

\section{Część A}

\subsection{Obliczyć granice}


\subsubsection*{1}

\begin{equation*}
    \lim_{x \to -1} \frac{x^4 + 3x^2 -4}{x + 1}
\end{equation*}

Używając schematu Hornera uzyskujemy \(x^4 + 3x^2 -4 = (x^2 + 4) (x - 1) (x + 1)\)

Zatem:
\begin{equation*}
    \lim_{x \to -1} \frac{x^4 + 3x^2 -4}{x + 1} = \lim_{x \to -1} \frac{(x^2 + 4) (x - 1) (x + 1)}{x + 1} = \lim_{x \to -1} (x^2 + 4)(x - 1) =
\end{equation*}
\begin{equation*}
    = (-1^2 + 4)(-1 - 1) = 5 \cdot -2 = -10  
\end{equation*}

\subsubsection*{2}

\begin{center}
    NOTATKA: jako że \(x \to +\infty\), to możemy założyć \(x>0\) i wtedy \(|x| = x\). Nie zmieni to wyniku granicy.
    Analogicznie dla \(x \to -\infty\) mamy \(|x| = -x\)
\end{center}
\begin{center}
    NOTATKA 2: Warto się przekonać formalnie o \(\lim_{x \to p} \sqrt{f(x)} = \sqrt{\lim_{x \to p} f(x)}\) (także \(f(x) >= 0\) bo pierwiastek)
\end{center}

\begin{equation*}
    \lim_{x \to +\infty} (\sqrt{x^2 + 1} - \sqrt{x^2 - 1}) = \lim_{x \to +\infty} (\sqrt{x^2 + 1} - \sqrt{x^2 - 1}) \frac{\sqrt{x^2 + 1} + \sqrt{x^2 - 1}}{\sqrt{x^2 + 1} + \sqrt{x^2 - 1}} =
\end{equation*}
\begin{equation*}
    = \lim_{x \to +\infty} \frac{x^2 + 1 -x^2 + 1}{\sqrt{x^2 + 1} + \sqrt{x^2 - 1}} = \lim_{x \to +\infty} \frac{2}{\sqrt{x^2 + 1} + \sqrt{x^2 - 1}} =
\end{equation*}
\begin{equation*}
    = \lim_{x \to +\infty} \frac{2}{\sqrt{x^2} \sqrt{1 + \frac{1}{x^2}} + \sqrt{x^2}\sqrt{1 - \frac{1}{x^2}}} = \lim_{x \to +\infty} \frac{2}{\sqrt{x^2}( \sqrt{1 + \frac{1}{x^2}} + \sqrt{1 - \frac{1}{x^2}})} = 
\end{equation*}
\begin{equation*}
    = \lim_{x \to +\infty} \frac{2}{|x|( \sqrt{1 + \frac{1}{x^2}} + \sqrt{1 - \frac{1}{x^2}})} = \lim_{x \to +\infty} \frac{2}{x( \sqrt{1 + \frac{1}{x^2}} + \sqrt{1 - \frac{1}{x^2}})} =
\end{equation*}
\begin{equation*}
    = \Bigg(\lim_{x \to +\infty} \frac{2}{x}\Bigg) \cdot  \Bigg(\lim_{x \to +\infty} \frac{1}{( \sqrt{1 + \frac{1}{x^2}} + \sqrt{1 - \frac{1}{x^2}})}\Bigg) = 0 \cdot \frac{1}{2} = 0
\end{equation*}

\subsubsection*{3}

\begin{center}
    NOTATKA: Podobno można szybciej, tip: wzór skróconego mnożenia na górze, wolfram po komendzie simplify
    potrafi zrobić z tego \( \sqrt{x} + 2 \cdot x^{1/4} + 4 \) a wtedy podstawienie \(12\) leci od razu.
\end{center}

\begin{equation*}
    \lim_{x \to 16} \frac{\sqrt{x \sqrt{x}} - 8}{\sqrt[4]{x} - 2} = \lim_{x \to 16} \frac{\sqrt[4]{x^3} - 8}{\sqrt[4]{x} - 2} =
\end{equation*}
\begin{equation*}
    = \lim_{x \to 16} \Bigg(\frac{\sqrt[4]{x^3} - 8}{\sqrt[4]{x} - 2}\Bigg) \Bigg(\frac{\sqrt[4]{x^3} + 2\sqrt[4]{x^2} + 2^{2}\sqrt[4]{x} + 2^3}{\sqrt[4]{x^3} + 2\sqrt[4]{x^2} + 2^{2}\sqrt[4]{x} + 2^3}\Bigg) = 
\end{equation*}
\begin{equation*}
    = \lim_{x \to 16} \frac{(\sqrt[4]{x^3} - 8)(\sqrt[4]{x^3} + 2\sqrt[4]{x^2} + 2^{2}\sqrt[4]{x} + 2^3)}{x - 16} = 
\end{equation*}
\begin{equation*}
    = \lim_{x \to 16} \Bigg(\frac{(\sqrt[4]{x^3} - 8)(\sqrt[4]{x^3} + 2\sqrt[4]{x^2} + 2^{2}\sqrt[4]{x} + 2^3)}{x - 16}\Bigg) \cdot 
\end{equation*}
\begin{equation*}
    \cdot \Bigg(\frac{\sqrt[4]{x^9} + 8\sqrt[4]{x^6} + 8^{2}\sqrt[4]{x^3} + 8^3}{\sqrt[4]{x^9} + 8\sqrt[4]{x^6} + 8^{2}\sqrt[4]{x^3} + 8^3}\Bigg) =
\end{equation*}
\begin{equation*}
    = \lim_{x \to 16} \frac{(x^3 - 8^4)}{(x - 16)}\frac{(\sqrt[4]{x^3} + 2\sqrt[4]{x^2} + 2^{2}\sqrt[4]{x} + 2^3)}{(\sqrt[4]{x^9} + 8\sqrt[4]{x^6} + 8^{2}\sqrt[4]{x^3} + 8^3)}
\end{equation*}

Dzieląc wielomian \(x^3 - 8^4 = x^3 - 16^3\) przez \(x-16\) otrzymujemy \(x^3 - 16^3 = (x-16)(x^2 + 16x + 256)\). Zatem:

\begin{equation*}
    = \lim_{x \to 16} \frac{(x-16)(x^2 + 16x + 256)}{(x - 16)}\frac{(\sqrt[4]{x^3} + 2\sqrt[4]{x^2} + 2^{2}\sqrt[4]{x} + 2^3)}{(\sqrt[4]{x^9} + 8\sqrt[4]{x^6} + 8^{2}\sqrt[4]{x^3} + 8^3)} =
\end{equation*}
\begin{equation*}
    = \lim_{x \to 16} \frac{(x^2 + 16x + 256)(\sqrt[4]{x^3} + 2\sqrt[4]{x^2} + 2^{2}\sqrt[4]{x} + 2^3)}{(\sqrt[4]{x^9} + 8\sqrt[4]{x^6} + 8^{2}\sqrt[4]{x^3} + 8^3)}
\end{equation*}
\begin{equation*}
    = \frac{(16^2 + 16^2 + 256)(\sqrt[4]{16^3} + 2\sqrt[4]{16^2} + 2^{2}\sqrt[4]{16} + 2^3)}{(\sqrt[4]{16^9} + 8\sqrt[4]{16^6} + 8^{2}\sqrt[4]{16^3} + 8^3)} =
\end{equation*}
\begin{equation*}
    = \frac{(16^2 + 16^2 + 16^2)(2^3 + 2^{1} \cdot 2^{2} + 2^{2} \cdot 2^{1} + 2^3)}{(2^9 + 2^{3} \cdot 2^{6} + 2^{6} \cdot 2^{3} + 2^{9})} = 
\end{equation*}
\begin{equation*}
    = \frac{(2^8 + 2^8 + 2^8)(2^3 + 2^3 + 2^3 + 2^3)}{(2^9 + 2^9 + 2^9 + 2^9)} = 
\end{equation*}
\begin{equation*}
    =  \frac{(3 \cdot 2^8)(4 \cdot 2^3)}{(4 \cdot 2^9)} = \frac{3 \cdot 2^8}{2^6} = 3 \cdot 2^2 = 12
\end{equation*}

\subsubsection*{4}

\begin{equation*}
    \lim_{x \to 1} \frac{1 - \sqrt[3]{x}}{1 - \sqrt[5]{x}} = \lim_{x \to 1} \frac{1 - \sqrt[3]{x}}{1 - \sqrt[5]{x}}\Bigg(\frac{(1 + \sqrt[3]{x} + \sqrt[3]{x}^2)}{(1 + \sqrt[3]{x} + \sqrt[3]{x}^2)}\Bigg) =
\end{equation*}
\begin{equation*}
    = \lim_{x \to 1} \frac{1 - \sqrt[3]{x}}{1 - \sqrt[5]{x}} \Bigg(\frac{(1 + \sqrt[3]{x} + \sqrt[3]{x}^2)}{(1 + \sqrt[3]{x} + \sqrt[3]{x}^2)}\Bigg) \Bigg(\frac{1 + \sqrt[5]{x} + \sqrt[5]{x}^2 + \sqrt[5]{x}^3 + \sqrt[5]{x}^4}{1 + \sqrt[5]{x} + \sqrt[5]{x}^2 + \sqrt[5]{x}^3 + \sqrt[5]{x}^4}\Bigg)
\end{equation*}
\begin{equation*}
    = \lim_{x \to 1} \frac{(1 - x)(1 + \sqrt[5]{x} + \sqrt[5]{x}^2 + \sqrt[5]{x}^3 + \sqrt[5]{x}^4)}{(1 - x)(1 + \sqrt[3]{x} + \sqrt[3]{x}^2)}
\end{equation*}
\begin{equation*}
    = \lim_{x \to 1} \frac{(1 + \sqrt[5]{x} + \sqrt[5]{x}^2 + \sqrt[5]{x}^3 + \sqrt[5]{x}^4)}{(1 + \sqrt[3]{x} + \sqrt[3]{x}^2)}
\end{equation*}
\begin{equation*}
    = \frac{1 + \sqrt[5]{1} + \sqrt[5]{1^2} + \sqrt[5]{1^3} + \sqrt[5]{1^4}}{1 + \sqrt[3]{1} + \sqrt[3]{1^2}} = \frac{5}{3}
\end{equation*}

\subsubsection*{5}

\begin{equation*}
    \lim_{x \to 0} \frac{\sqrt{x^2 + 1} - \sqrt{x + 1}}{1 - \sqrt{x + 1}} = \lim_{x \to 0} \Bigg(\frac{\sqrt{x^2 + 1} - \sqrt{x + 1}}{1 - \sqrt{x + 1}}\Bigg) \Bigg(\frac{1 + \sqrt{x + 1}}{1 + \sqrt{x + 1}}\Bigg)
\end{equation*}
\begin{equation*}
    = \lim_{x \to 0} \frac{\Big(\sqrt{x^2 + 1} - \sqrt{x + 1}\Big)(1 + \sqrt{x + 1})}{1 - x - 1}
\end{equation*}
\begin{equation*}
    = \lim_{x \to 0} \frac{\Big(\sqrt{x^2 + 1} - \sqrt{x + 1}\Big)(1 + \sqrt{x + 1})}{-x}
\end{equation*}
\begin{equation*}
    = \lim_{x \to 0} \frac{\Big(\sqrt{x^2 + 1} - \sqrt{x + 1}\Big)(1 + \sqrt{x + 1})}{-x} \frac{(\sqrt{x^2 + 1} + \sqrt{x + 1})}{(\sqrt{x^2 + 1} + \sqrt{x + 1})}
\end{equation*}
\begin{equation*}
    = \lim_{x \to 0} \frac{(x^2 + 1 -x - 1)(1 + \sqrt{x + 1})}{-x(\sqrt{x^2 + 1} + \sqrt{x + 1})}
\end{equation*}
\begin{equation*}
    = \lim_{x \to 0} \frac{-x(1 - x)(1 + \sqrt{x + 1})}{-x(\sqrt{x^2 + 1} + \sqrt{x + 1})}
\end{equation*}
\begin{equation*}
    = \lim_{x \to 0} \frac{(1 - x)(1 + \sqrt{x + 1})}{(\sqrt{x^2 + 1} + \sqrt{x + 1})}
\end{equation*}
\begin{equation*}
    = \frac{(1 - 0)(1 + \sqrt{0 + 1})}{(\sqrt{0^2 + 1} + \sqrt{0 + 1})} = \frac{(1 + 1)}{(1 + 1)} = \frac{2}{2} = 1
\end{equation*}


\subsubsection*{6}

\begin{center}
    NOTATKA: Warto przekonać się o istnieniu granicy (poza zerem) dla \(\frac{|x|}{x}\) i \(\frac{x}{|x|}\)
\end{center}

\begin{equation*}
    \lim_{x \to -\infty} (\sqrt{x^2 -8x + 3} - \sqrt{x^2 +11x}) 
\end{equation*}
\begin{equation*}
    = \lim_{x \to -\infty} (\sqrt{x^2 -8x + 3} - \sqrt{x^2 +11x}) \frac{(\sqrt{x^2 -8x + 3} + \sqrt{x^2 +11x})}{(\sqrt{x^2 -8x + 3} + \sqrt{x^2 +11x})} 
\end{equation*}
\begin{equation*}
    = \lim_{x \to -\infty} \frac{x^2 -8x + 3 -x^2 - 11x}{(\sqrt{x^2 -8x + 3} + \sqrt{x^2 +11x})}
\end{equation*}
\begin{equation*}
    = \lim_{x \to -\infty} \frac{-19x + 3}{(\sqrt{x^2 -8x + 3} + \sqrt{x^2 +11x})}
\end{equation*}
\begin{equation*}
    = \lim_{x \to -\infty} \frac{x(\frac{3}{x} - 19)}{\Big(\sqrt{x^2(1 - \frac{8}{x} + \frac{3}{x^2})} + \sqrt{x^2(1 + \frac{11}{x}})\Big)}
\end{equation*}
\begin{equation*}
    = \lim_{x \to -\infty} \frac{x(\frac{3}{x} - 19)}{\Big(\sqrt{x^2}\sqrt{(1 - \frac{8}{x} + \frac{3}{x^2})} + \sqrt{x^2}\sqrt{(1 + \frac{11}{x}})\Big)}
\end{equation*}
\begin{equation*}
    = \lim_{x \to -\infty} \frac{x(\frac{3}{x} - 19)}{\Big(|x|\sqrt{(1 - \frac{8}{x} + \frac{3}{x^2})} + |x|\sqrt{(1 + \frac{11}{x}})\Big)}
\end{equation*}
\begin{equation*}
    = \lim_{x \to -\infty} \frac{x(\frac{3}{x} - 19)}{|x| \Big(\sqrt{(1 - \frac{8}{x} + \frac{3}{x^2})} + \sqrt{(1 + \frac{11}{x}}\Big)}
\end{equation*}
\begin{equation*}
    = \Bigg(\lim_{x \to -\infty} \frac{x}{|x|}(\frac{3}{x} - 19)\Bigg) \cdot \Bigg(\lim_{x \to -\infty} \frac{1}{\Big(\sqrt{(1 - \frac{8}{x} + \frac{3}{x^2})} + \sqrt{(1 + \frac{11}{x}}\Big)}\Bigg)
\end{equation*}
\begin{equation*}
    = \Bigg(\lim_{x \to -\infty} \frac{x}{|x|}\Bigg) \cdot \Bigg(\lim_{x \to -\infty} \frac{3}{x} - 19 \Bigg) \cdot \frac{1}{\lim_{x \to -\infty} \Big(\sqrt{(1 - \frac{8}{x} + \frac{3}{x^2})} + \sqrt{(1 + \frac{11}{x}}\Big)}
\end{equation*}
\begin{equation*}
   = \Bigg(\lim_{x \to -\infty} \frac{x}{|x|}\Bigg) \cdot \Bigg(\lim_{x \to -\infty} \frac{3}{x} - 19 \Bigg) \cdot
\end{equation*}
\begin{equation*}
    \cdot \frac{1}{\Big(\lim_{x \to -\infty} \sqrt{(1 - \frac{8}{x} + \frac{3}{x^2})}\Big) + \Big(\lim_{x \to -\infty} \sqrt{(1 + \frac{11}{x}}\Big)}
\end{equation*}
\begin{equation*}
    = \Bigg(-1 \Bigg) \cdot \Bigg( 0 - 19 \Bigg) \cdot
\end{equation*}
\begin{equation*}
    \cdot \frac{1}{\Big( \sqrt{(1 - 0 + 0)}\Big) + \Big(\sqrt{(1 + 0}\Big)}
\end{equation*}
\begin{equation*}
    = \Bigg(-1 \Bigg) \cdot \Bigg( 0 - 19 \Bigg) \cdot \frac{1}{2} = \frac{19}{2}
\end{equation*}

\end{document}